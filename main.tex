\documentclass{article}
\usepackage[utf8]{inputenc}
\usepackage{graphicx}
\usepackage{booktabs}
\usepackage{caption}
\captionsetup[figure]{font=small,labelfont=small}

\title{Skill-bias and Wage Inequality in CEE: empirical investigation}
\author{Jan Pintera}
\date{}

\begin{document}

\maketitle
\section{Introduction}
The echoes of labour markets turmoil in the developed economies have been heard quite often in recent decades. Fears of unemployment, job-quality deterioration or, more specifically, the "hollowing-out" of the entire middle class appear in the latest government reports (e.g. Rodrik and Stantcheva (2020) in the case of France) and have been long discussed in the academic literature concentrated on effects of technological change in the United States (Acemoglu, 2012; Autor, 2014).

This paper attempts to compare these findings with experience of the so-called new EU member state countries of central and eastern Europe (CEE). These countries play rather different role in the global market value chains than a model developed country (Baldwin and Lopez-Gonzales, 2015)\footnote{The CEE (including Poland) are typical examples of the "Factory Economies" strongly linked to their headquarter economy - Germany. Note that the labour market theories discussed below are created and tested in context of the headquarter economies (US and Germany in our case) (Baldwin and Lopez-Gonzales, 2015)}. For their favourable unit labour cost and skilled workforce they seem to be an ideal recipients of offshoring from the high-wage economies, we can therefore assume rather opposite development than that in the old EU states. Indeed, empirical work in case of the income distribution shows signs of declining wage inequality in last decade or so (Magda et al., 2021). This work brings descriptive analysis of this issue using the EU-SILC survey data from 2005 and 2019 and attempts to put this data into context of research on the technological change impact and labour market polarization seen in the US and Western Europe.
\\
The literature generated several testable hypotheses about the impact of technological change on the functioning of the labour market, that despite their early origin (Katz and Murphy, 1992) and empirical critiques by Mishel, Schierholz and Schmitt (2013) and others seem to endure to these days (Aziz and Cortes, 2021; Goldin and Katz, 2020).

Among those is especially the STBC (Skill-biased technological change) hypothesis - assuming negative relationship between relative high and low skill labour supply and wages. Further refinement of this hypothesis postulate a job and wage polarization - seen in the US and elsewhere, a phenomena often connected with technological change and also an increase in relative supply of manual non-routine jobs. Using similar methods as statistics as seminal works on the US and German labour market (Katz and Murphy, 1992; Mishel, Schierholz and Schmitt, 2013), this work found that many of the phenomena mentioned above were not confirmed in case of the Central Europe. The general conclusion reached by the study is a good performance of the lower parts of the wage distribution. Perhaps most notably, we see relative decline of the highest earners in both wages relative to median and employment, contrary to the characteristic U-shaped behaviour documented by Acemoglu (2012) and interpreted as the job and wage polarization. This seems to be in line with the view of the new member states as a semi-periphery of the West, open economies with low unit labour costs which in environment of globalization leads to an inflow on relatively routine-intensive jobs, which drives demand for the low and middle type of jobs and has generally equalizing effect on the labour market. 
We further investigated the question of causality of labour market changes - we utilized a simple supply-demand framework used by Katz and Murphy (1992) and later studies and investigated significance of labour supply and wages relationship in regression inspired by the framework. Due to lack of observation that the micro-surveys can give us, we utilized a panel regression combining multiple CEE countries. One of the motivations of this approach was the significant skill upgrading seen in the region a phenomena visible in the US several decades earlier where the framework is found to perform well Hardy et al., 2018).
\\
The contribution of this work is an investigation of the main hypotheses about the development of labour market inequality in the new EU member states with particular emphasis given to the skill-biased framework and empirical hypotheses stemming directly from it. We also want to overview CEE labour market trends compared to the regularities observed in Germany and the United States. We consider this topic and the use of micro-data in this context as relatively under-researched, even more so as there is a good reason to believe that the observed development in CEE will be different, if not inverse to that found in the West. Compared with previous works on the topic, this work brings a direct test of the skill-biased framework for all the countries of interest instead of testing a particular subsection of the theory (routine-biased technological change as in case of Arendt and Grabowski (2019) or Hardy et al. (2018)) or concentrating on a single country. It will also use a different source of data (EU-SILC) that provides annual data for all countries of interest and due to their annual form and coverage of labour supply are in our view the best fit to the surveys used in the US studies such as Current Population Survey used in Katz and Murphy (1992). \\
The paper is organized as follows. First part introduces the skill bias framework and reviews labour market development in the CEE and US and the Germany. Second part outlines details of the Skill-Bias Technological change and discusses variable construction, third section presents and discusses the results and forth section concludes.

\section{Labour Market developments}
\subsection{Wage inequality hypotheses - the US case}

One of the key hypotheses emerging from the technology debate is so-called skill-biased technological change, which explains changes in relative wages using a simple supply-demand framework (the "Canonical model"; Acemoglu, 2012) focusing on different levels of skills/education. The Canonical model in its original form is a simple and straightforward model that uses relative high/low skill labour supply and time dependent "skill-biased" technological progress as a determinants of relative wages.\\
The skill-bias hypothesis was tested by Katz and Murphy (1992) in the form of regression of US skill premium (college/high-school relative wage) on a time trend and a relative supply of high/low-skilled labour. Despite its simplicity, the model was shown to perform rather well on the US data before 1990s (Katz and Murphy, 1992) and existence of the link between technology and education as a determining factor in wage setting in the long term seems evident (Piketty, 2015). However, Acemoglu and Autor (2011) shows that the Katz and Murphy's model overpredicts the skill premium in the 1990s and the 2000s. It also fails to account for several other stylized facts about the recent developments of wages in the US, most importantly, the job and wage polarization, i.e., strengthening of the tails of income/employment distribution. This process seems strongly connected to the automatization of middle-skill jobs. Based on these findings, Acemoglu (2012) proposes a comprehensive task-based framework (also routine-biased technological change, RBTC) focused on level of routine content of the tasks involved rather than on workers' skills. Formal representation of the task-based framework can be found in Acemoglu and Autor (2011). Most notably, this framework is capable of explaining the wage polarization phenomenon (increasing relative wages of bottom and top of the distribution with respect to the middle part) observed in the 1990s in the US.\\

Mishel, Schierholz and Schmitt (2013) postulate three testable hypotheses derived from the skill-biased technological change literature and its extensions. First, the labour supply and demand interactions determine wage formation. More concretely, technological change causes changes in labour demand which in turn affect wages. This causality can be considered a general feature of the framework common to both the original skill-biased technological change and its subsequent variant, the routinization-biased technological change. Second, from the empirical point of view, the skill-biased technological change leads to job polarization - a phenomenon highlighted by Acemoglu (2012), Howell and Kallenberg (2019) and others when discussing the developments of the Western labour markets in the last decades of the 20th century. Note, that in this point, both variants of the technological framework differ, with the original the SBTC hypothesis predicting a monotonic employment and wage development across the occupational distribution, a phenomena observed in the 1980s.\\
Third, the RBTC hypothesis implies a rise in both employment and wages in specific type of services - namely the low-wage service jobs characterized by manual non-routine content.

Note that the skill-bias framework always faced critiques such as Mishel, Schierholz and Schmitt (2013) who rather than explicitly denying the the underlying "job polarization" trend, map it to much earlier time and thus deny its causal link with inequality rise after 1970s (and thus questions link between job polarization and wage polarization). In their interpretation technological changes has significant impact on occupation composition, not on wage inequality. They also point to a general wage deficit - inability of wages to keep up with productivity growths and rising profits after 2000s.

Regarding the the key part of the framework - job-polarization, we should note that considering evidence outside the US and in longer time period, we can hardly consider it a stylized fact. As shown by Mishel and Shierholz and Schmitt (2013) job-polarization seem to be a phenomena linked firmly to the 1990s and already the early 2000s brought slowdown of education premium and high-occupation rise. We can therefore also formulate the difference between RBTC and SBTC as difference between 1980s and 1990s US labour market.

This work aims to provide an overview of the Central and Eastern European Labour market after the EU accession from the perspective of the above mentioned labour market theories about the impact of technological change in the developed economies. The work will mainly concentrate on labour market developments in the US and Germany as the most relevant labour markets for CEE and compare their development with labour market trends in the CEEs countries. The main theoretical frameworks investigated will be the skill-biased technological change both in its canonical version and its occupation-content oriented variant - routine based technological change. The work aims to provide close comparison of CEE and German trends, especially due to proximity and co-dependence of Germany and CEE in the global value-chains.
We should note that the CEE region has a specific characteristics when compared to the Western Europe, as its comparative advantage in labour costs place it to different side of the globalization demarcation line than countries of the Western Europe. At the same time, giving closer look at the relationship between the West and the CEE region give us even more fascinating picture with the CEE serving as a pool of relatively cheap and qualified labour to Germany, strongly influencing Germany's internal labour market in return (Marin 2004; 2018). Our comparison will mainly focus on Germany, as it is the most deeply connected neighbouring country (see Baldwin, 2015). Following sections outline labour market developments in Germany and CEE.


\subsection{CEE}
As noted by Tyrowicz and Smyk (2019) in their assessment of the impact of structural change on inequality, the micro-data on income and inequality have been used for assessment of the skill-bias hypothesis in a small number of developed countries only. For the CEE countries, these micro data were not available until relatively recently.
So far, there were only few attempts to investigate the role of skills in income inequality among the CEE countries. Arendt and Grabowski (2019) studied the wage premium in Poland, and Hardy et al. (2018) provide an analysis of task-content development in EU following Acemoglu and Autor's (2011) approach and provide analysis of labour supply development in EU-24 with emphasis on the CEE countries. Both papers exploit the task-content division of the labour force (classification of jobs according to a required level of cooperation and creativity), this work attempts for deeper examination of major labour market hypotheses in the CEE countries, it will use with the EU-SILC Eurostat database that provides micro-level information about income and living conditions in the EU countries on an annual basis. We believe this database is the closest to the Current Population Surveys data used by the key works in the field (Katz and Murphy (1992), Acemoglu (2012).
The results of both Hardy et al. (2018) and Arendt and Grabowski (2019)  point towards certain deviations of this region from the rest of Europe in terms of the task distribution. Namely, according to Hardy et al. (2018) we see increase in routine cognitive tasks in CEE countries, which is contrary to both the old-EU countries and routine-replacing technological change hypothesis, similarly Arendt and Grabowski (2019) find relative wages in routine manual jobs in Poland too high for the RBTC hypothesis to hold. Both studies then note significant educational upgrading in the region, especially the rapidly increasing tertiary education attainment (Hardy et al., 2018). We should note that at least in this aspect the CEE seem to differ significantly from the U.S. labour market, where as noted by Acemoglu (2012) high-school attainment is actually stagnant since 1960s and post-secondary attainment decelerated already in 1970s. Specificity of the CEE income inequality with respect to the West is also confirmed other works such as Magda et al. (2020) - who notes decrease of wage inequality in the CEE in 2002-2014 period. Before this period the CEE countries experienced significant inequality rise due to their economic transformation but the inequality leveled since then (Tyrowicz and Smyk, 2019) with evidence of wage inequality staying lower than in the developed countries Mysíková and Večerník (2018). From our point of view, this speak in favour of analyzing the skill-bias hypothesis in this region.



\subsection{Germany}

%Germany on the other hand - underwent a profound labour market transformation in recent decades (Marin et al., 2018). Dustmann et al. (2014) attribute its labour market resilience to a unique set of labour market institutions - most notably its decentralized and de-politicized wage bargaining process that allowed for labour market flexibility in face of adverse external macroeconomic conditions. Concretely, Dustmann et al. (2014) concentrate their analysis on wage restrains of German workers which can be reflected in behaviour of German unit labour costs with respect to the other Western countries. Dustman et. al (2014) also notes note decreasing real wages at the lower end of the wage distribution after mid-1990 but not before. He attributes this to the German interaction with the CEE countries - that served as a pool of comparatively cheap skilled labour that helped German businesses and in turn allowed to put pressure on German workers. Marin (2004) also interestingly notes that German were the offshoring skilled rather unskilled work to the new countries, in an attempt to solve its own low human capital endowment shortages.

In general, Germany has been, similarly to the US and other Western economies, experiencing rising income inequality at least since the 1980s (Biewen, Fitzenberger and Lazzer, 2017). However recent development point to certain specificity of the country's development. 

Biewen and Sturm (2021) find that the inequality has been stagnating since 2005 and attribute this phenomenon to recent labour market boom. Schank and Bossler (2020) concentrating on the lower tail of the distribution find a rising wage inequality in 2000s and declining trend after 2010s, furthermore they observe a sharp drop in inequality after 2014, that they attribute to minimum wage introduction.

Giving a closer look at the development of inequality in different parts of the income distribution Biewen and Seckler (2019) do not confirm wage polarization found in the US. They find much more monotonic development with the highest percentiles gaining relatively the most. This is confirmed by Biewen, Fitzenberger and Lazzer (2017) who document rise in inequality limited to the top part of the distribution (development in line with with the SBTC hypothesis) until mid 1990s with the labour market institutions preventing rise of the inequality at cost of higher unemployment  - later it rose across the entire distribution.

%Biewen and Sturm (2021) comments on German development after recent labour market boom (since 2005) and find it having an equalizing effect - it led to income gains across the distribution and the lower part of the distribution experienced bigger gains than the upper parts, despite institutional and external factor dampening this effect.


As far as the causes of the inequality rise is concerned the literature has so far not reached a conclusion. Yet among the most often mentioned reasons are labour market institutions (decline in collective bargaining) and composition changes (such as educational upgrading, labor market history, industry structure, and occupation) (Biewen, Fitzenberger and Lazzer, 2017). Emphasized is also a significant wage restraint by German workers reflected in behaviour of German unit labour costs with respect to the other Western countries (Dustmann et al., 2014) and the role played in the CEE countries that served as a pool of comparatively cheap skilled labour that helped German businesses and in turn allowed to put pressure on German workers (Marin 2004, 2018).
In terms of the SBTC, Biewen, Fitzenberger and Lazzer (2017) find strong influence of composition changes in explaining rising inequality - education (especially in the upper part of the distribution) and changes in recent labor market histories (lower part) and conclude that this finding is in line with SBTC hypothesis.
Last but not least, Glitz and Wissman (2021) show that after breaking German population to three education levels and two age groups and using the procedure developed by Katz and Murphy (1992) and Card and Lemieux (2001), they find that the labour supplies are to large extent able to explain the changes in skill premium in Germany. They find especially pronounced rise in skill premium of medium skilled to low skilled and link it to decline in the share of population with vocational training. Their findings are therefore very much in line with the original SBTC framework and can be seen as reaching a similar results as the seminal work of Goldin and Katz (2009). %TODO: Glitz and Wissmann have some conclusions about Polarization, add those here


\section{Methodology}
\subsection{The Canonical Model}
In the analysis below we will follow a modelling framework ("Canonical Model") developed first in Tinbergen and further elaborated in Katz and Murphy (1992) Goldin and Katz (2009), Card and Lemieux (2001), Acemoglu and Autor (2011) or Glitz and Wissmann (2021).
The fundamental assumption behind the framework is the "skill-bias" of the technological change that causes relative demand for high-skilled labour to permanently rise.
The model departs from an CES production function:
\begin{equation}
\label{eqn:STBC_prod_function}
Y = [\theta(A_{L}L)^{\frac{\gamma - 1}{\gamma}} + (1 - \theta)(A_{H}H)^{\frac{\gamma - 1}{\gamma}}]^\frac{\gamma}{\gamma - 1}
\end{equation}

In this is setting, $H$ denotes high-skilled (university) labour supply, $L$ low-skilled (non-university) labour supply, $\gamma$ is the elasticity of substitution between high skill and low skill labor and $\theta$ determines relative importance the two types of labour in the production function. Primary measure of inequality used is a (log) skill premium between these two types of labour. We can get this premium by first deriving wage for both $L$ and $H$ and obtaining their ratio:
\[\frac{w_{H}}{w_{L}} = \frac{(1 - \theta)}{\theta} \left(\frac{H}{L}\right)^{-\frac{1}{\gamma}}\left(\frac{A_{H}}{A_{L}}\right)^{\frac{\gamma - 1}{\gamma}}\]

and then linearizing the equation by taking logs:
\[\log(\frac{w_{H}}{w_{L}}) = c + \frac{\gamma - 1}{\gamma}\log(\frac{A_{H}}{A_{L}}) - \frac{1}{\gamma}\log(\frac{H}{L})\]
In this function the relative supply of skilled labour $\frac{H}{L}$ decreases the skill premium whereas the unobserved $\frac{A_{H}}{A_{L}}$ parameter measuring relative development of factor augmenting technology for high and low skilled labour represent the skill-biased technological change. We assume that this variable has a log linear trend. This assumption contains a key part of the model - permanently ongoing technological change increasing demand for skilled labour. Thus we obtaining the final version of the equation:
\begin{equation}
\label{eqn:STBC_regression}
\log(\frac{w_{H}}{w_{L}}) = c + \frac{\gamma - 1}{\gamma}\sigma_0 + \frac{\gamma - 1}{\gamma}\sigma_{1}t - \frac{1}{\gamma}\log(\frac{H}{L})
\end{equation}
OLS regression in form of is then estimated by Katz and Murphy (1992), Acemoglu (2012), Goldin and Katz (2020) and others in order to obtain estimate of the elasticity of substitution and an annual change in skilled labour demand.

Card and Lemieux (2001) and Glitz and Wissmann (2021) offer an extension of the model by incorporating middle skill category and distinguishing between young and old workers. Their framework result in a system of equations, allowing to obtain elasticities of substitution between different sub-group using a seemingly unrelated regression framework.

Routine Biased Technological change is then a further extension of the model above (in fact it nests the Canonical model as its specific case). The key idea of this framework is an economic activity primarily consisting of tasks that can be divided between routine and non-routine and further between cognitive and non-cognitive. The technological change is assumed to substitute the routine tasks and strengthen position of non-routine ones. The framework is thus able to explain polarization patterns visible in the US labour market in 1990s (Autor, 2014). This framework is formally elaborated by Acemoglu and Autor (2011).

\subsection{Variables construction}
We follow proceedings of Katz and Murphy (1992) and Glitz and Wissmann (2021) while calculating variables for equation \ref{eqn:STBC_regression}. The skill-premium is calculated. Whereas the relative labour supply is obtained using efficiency units.

% TODO: add estimation info such as efficiency unies, etc.; see Acemoglu and also GLitz and Wissmann

\section{Results}
In our analysis we used the Eurostat EU-SILC database for the CEE countries (Czech Republic, Slovakia, Poland, Hungary, Latvia, Estonia, Lithuania, Romania and Bulgaria) between years 2004 and 2019. EU-SILC collects information about income, poverty, social exclusion and living conditions across the EU countries. Our database contains records of more than 2 million individuals in total.

% Figure 1 - TODO: any interpretation of those numbers
Let us first sum-up general development of labour market inequality in Figure \ref{wage_gaps_CEE} which portrays 50/10 and 90/50 wage gap (ratio of percentiles of year log wage distribution) development for full-time workers in four CEE countries (so called Visegrad group). Despite certain differences among the countries we can see relatively unified development with a significant difference in comparison to the US data as found for example in (Mishel, Shierholz and Schmitt, 2013). Most notably, there is a clear decreasing tendency for the 90/50 wage gap, a finding contrary to the US and Western Europe evidence as well as the hypothesis of job polarization and "hollowing-out" of the middle class. We can also note mostly decreasing tendency of the 50/10 ratio a movement more in line with the US evidence. In general we can see strengthening of the lower part of the income distribution. The development also contrasts with German experience where we can see a general upward trend for both wage gaps (90/50 and 50/10) at least until 2015 (Biewen and Sturm, 2021).

%Figure 2
In figure \ref{wage_changes_percentiles} we can find changes in log wages of full-time workers relative to the median for the Visegrad countries between 2005 and 2019. Unlike the U-shape curved characteristic of wage polarization visible in the US (Acemoglu and Autor, 2011) we can instead observe a monotonic behaviour across the measured CEE countries with a clear tendency for decline of the highest percentiles. In general, the graphs confirm the conclusions from figure \ref{wage_gaps_CEE} and unlike the Western evidence show relative decline of the highest incomes, inverse behaviour than observed in the West.

% Figure 3
Figure \ref{employ_changes_percentiles} shows changes of employment shares for the ISCO-08 occupations skill rank between 2011-2019 it also depicts a locally-weighted smoothing regression curve. Compared to the US evidence, we most notably never see a characteristic U-shape found the Acemoglu (2012) for the 90s and 00s US labour market interpreted as the job-polarization. We can notice a rather different and diversified behaviour across the CEE countries, while we see a rise in the high-income occupations in case of Hungary - situation resembling the graph for the US market in the 1980s, the Czech republic shows the strongest rise around the median of the income distribution - direct opposite of the wage polarization logic and both Poland and Slovakia experience decline in the higher part of the skill-rank distribution.

% Panel Regression
In the next part we get to the crucial part of the canonical model doctrine - (presumably) negative relationship between relative labour supply and relative wages. As can be seen in the plots \ref{labour_supplies_cee} and \ref{high_low_log_wage_gap} we see rising tendency in relative high-skill supply rise across CEE on one hand (as high skill category we define individuals with the highest ISCED education level attained greater or equal to 5) and decline in university/college wage premium on the other. The negative correlation between the variables seems to be evident. We now take a look at the panel regression for years 2005-2019 to check if this relationship is statistically significant.\\
In tables \ref{RE_models_comparison} and \ref{FE_models_comparison}  we see the regression results of several regression specifications coming from the basic regression design proposed by Katz and Murphy (1992) in the form of equation \ref{eqn:STBC_regression}, we also add several other explanatory variables inspired by research on determinants of labour market inequality. As suggested by Farber (2018), we added union density for each year as a measure of labour market conditions. 
The H/L parameter in tables \ref{RE_models_comparison} and \ref{FE_models_comparison} is significant and negative and can be interpreted as an elasticity of substitution between high and low skilled workers. The time trend parameter is interpreted as an annual change in relative high skill demand caused by technological change.
Overall, the regression results suggest significantly higher elasticity of substitution in the CEE compared to the US and German case (Acemoglu, 2012; Glitz and Wissmann, 2021) and at the same time significantly lower pace of technological change. Also note that technological change seems to have negative sign despite a miniscule magnitude, this would imply that technological change has an equalizing effect (decrease in skill premium) in CEE.

Overall, we can identify a marked difference of the observed basic labour market patterns in the CEE compared to US and German evidence (and we can probably say Western experience in general) that to bigger or lesser degree experienced and job and wage polarization. We do not confirm such a phenomenon in the CEE and rather see strengthening of the middle and bottom part of the distribution compared to the highest quantiles.

\section{Conclusion}

\section{References}
\begin{enumerate}
\item Acemoglu, Daron. "What does human capital do? A review of Goldin and Katz's The race between education and technology." Journal of Economic Literature 50.2 (2012): 426-63.

\item Acemoglu, Daron, and David Autor. "Skills, tasks and technologies: Implications for employment and earnings." Handbook of labor economics. Vol. 4. Elsevier, 2011. 1043-1171.

\item Arendt, Łukasz, and Wojciech Grabowski. "Technical change and wage premium shifts among task-content groups in Poland." Economic research-Ekonomska istraživanja 32.1 (2019): 3392-3410

\item Autor, David. Polanyi's paradox and the shape of employment growth. Vol. 20485. Cambridge, MA: National Bureau of Economic Research, 2014.

\item Baldwin, Richard, and Javier Lopez‐Gonzalez. "Supply‐chain trade: A portrait of global patterns and several testable hypotheses." The world economy 38.11 (2015): 1682-1721.

\item Biewen, Martin, Bernd Fitzenberger, and Jakob De Lazzer. "Rising wage inequality in Germany: Increasing heterogeneity and changing selection into full-time work." ZEW-Centre for European Economic Research Discussion Paper 17-048 (2017).

\item Biewen, Martin, and Matthias Seckler. "Unions, internationalization, tasks, firms, and worker characteristics: A detailed decomposition analysis of rising wage inequality in Germany." The Journal of Economic Inequality 17.4 (2019): 461-498.

\item Biewen, Martin, and Miriam Sturm. Why a Labour Market Boom Does Not Necessarily Bring Down Inequality: Putting Together Germany's Inequality Puzzle. No. 14357. Institute of Labor Economics (IZA), 2021.

\item Card, David, and Thomas Lemieux. 2001. "Can Falling Supply Explain the Rising Return to College for Younger Men? A Cohort-Based Analysis." Quarterly Journal of Economics 116(2)

\item Schank, Thorsten, and Mario Bossler. "Wage inequality in Germany after the minimum wage introduction." VfS Annual Conference 2020 (Virtual Conference): Gender Economics. No. 224543. Verein für Socialpolitik/German Economic Association, 2020.

\item Rodrik, Dani, and Stefanie Stantcheva. "Economic Inequality and insecurity: Policies for an inclusive economy." Report for the Blanchard-Tirole Commission (2020).

\item Aziz, Imran, and Guido Matias Cortes. "Between-group inequality may decline despite a rising skill premium." Labour Economics 72 (2021)

\item Hardy, Wojciech, Roma Keister, and Piotr Lewandowski. "Educational upgrading, structural change and the task composition of jobs in Europe." Economics of Transition 26.2 (2018): 201-231.

\item Howell, David R., and Arne L. Kalleberg. "Declining job quality in the United States: Explanations and evidence." RSF: The Russell Sage Foundation Journal of the Social Sciences 5.4 (2019): 1-53.

\item Goldin, Claudia, and Lawrence F. Katz. "Extending the Race between Education and Technology." AEA Papers and Proceedings. Vol. 110. 2020

\item Glitz, Albrecht, and Daniel Wissmann. "Skill premiums and the supply of young workers in Germany." Labour Economics 72 (2021): 102034.

\item Dustmann, Christian, et al. "From sick man of Europe to economic superstar: Germany's resurgent economy." Journal of Economic Perspectives 28.1 (2014): 167-88

\item Marin, D. (2004): “A Nation of Poets and Thinkers – Less So With Eastern Enlargement? Austria and Germany,” Discussion Paper 4358, Centre for Economic Policy Research, London

\item Marin, Dalia. "Global Value Chains, Product Quality, and the Rise of Eastern Europe." Explaining Germany’s Exceptional Recovery (2018): 41

\item Magda, Iga, Jan Gromadzki, and Simone Moriconi. "Firms and wage inequality in Central and Eastern Europe." Journal of Comparative Economics 49.2 (2021): 499-552.

\item Mishel, Lawrence, Heidi Shierholz, and John Schmitt. 2013. “Don’t Blame the Robots: Assessing the Job Polarization Explanation of Growing Wage Inequality.”

\item Katz, Lawrence F., and Kevin M. Murphy. "Changes in relative wages, 1963–1987: supply and demand factors." The Quarterly Journal of Economics 107.1 (1992): 35-78.

\item Tyrowicz, Joanna, and Magdalena Smyk. "Wage inequality and structural change." Social Indicators Research 141.2 (2019): 503-538.

\item Mysíková, M., and Večerník, J. (2018). Personal Earnings Inequality and Polarization: The Czech Republic in Comparison with Austria and Poland. Eastern European Economics, 56(1), 57–80.

\end{enumerate}

\newpage
\section{Figures}

\begin{figure}[h]%
    \centering
    {\includegraphics[scale = 0.45]{wage_gaps.png} }
    \caption{Development of (log) wage gaps for fulltime workers in CEE, 2005–2019}
    \label{wage_gaps_CEE}
\end{figure}

\begin{figure}%
    \centering
    {\includegraphics[scale=0.5]{wage_changes_by_percentiles.png} }
    \caption{Changes in Log Hourly Wages by Percentile Relative to the Median}
    \label{wage_changes_percentiles}
\end{figure}

\begin{figure}%
    \centering
    {\includegraphics[scale=0.5]{employ_changes_by_percentiles.png} }
    \caption{Changes in employment by occupational skill percentile, 2011–2019. Mean log-wage in 2011 was used for obtaining the occupation skill rank}
    \label{employ_changes_percentiles}
\end{figure}

\begin{figure}%
        \centering 
        {\includegraphics[scale=0.7]{labour_supplies_cee.png}}
        \caption{Changes in relative high/low skill labour supply in CEE}
        \label{labour_supplies_cee}
\end{figure}

\begin{figure}%
    \centering 
    {\includegraphics[scale=0.4]{high_low_log_wage_gap.png}}
    \caption{Changes in composition adjusted high/low-skill log wage premium}
    \label{high_low_log_wage_gap}
\end{figure}

\begin{table}[!htbp]
\centering 
\caption{Random Effect model comparison - CEE and Visegrad countries}
\label{RE_models_comparison}

\resizebox{\textwidth}{!}{\begin{tabular}{lccccc}
\toprule
                               &   \textbf{CEE}  & \textbf{CEE - Unions} & \textbf{CEE $t^{2}$} & \textbf{Visegrad} & \textbf{Visegrad $t^{2}$}  \\
\midrule
\textbf{Dep. Variable}         &     logW\textbackslash HL    &        \textbackslash HL       &     logW\textbackslash HL     &      logW\textbackslash HL     &        logW\textbackslash HL        \\
\textbf{No. Observations}      &       118       &          118          &       118        &         60        &           60           \\
\textbf{Cov. Est.}             &    Unadjusted   &       Unadjusted      &    Unadjusted    &     Unadjusted    &       Unadjusted       \\
\textbf{R-squared}             &      0.7631     &         0.7703        &      0.7731      &       0.8058      &         0.8430         \\
\textbf{R-Squared (Within)}    &      0.4455     &         0.4624        &      0.4690      &       0.5993      &         0.6759         \\
\textbf{R-Squared (Between)}   &      1.0000     &         1.0000        &      1.0000      &       1.0000      &         1.0000         \\
\textbf{R-Squared (Overall)}   &      0.7631     &         0.7703        &      0.7731      &       0.8058      &         0.8430         \\
\textbf{F-statistic}           &      38.657     &         35.884        &      36.465      &       44.818      &         47.413         \\
\textbf{P-value (F-stat)}      &      0.0000     &         0.0000        &      0.0000      &       0.0000      &         0.0000         \\
\textbf{=====================} & =============== &    ===============    & ===============  &  ===============  &    ===============     \\
\textbf{log(H\textbackslash L)}             &     -0.1002     &        -0.0739        &     -0.1514      &      -0.1521      &        -0.2496         \\
\textbf{ }                     &    (-1.6702)    &       (-1.2114)       &    (-2.3842)     &     (-1.4363)     &       (-2.4955)        \\
\textbf{Time}                 &     -0.0078     &        -0.0054        &      3.2631      &      -0.0109      &         6.5808         \\
\textbf{ }                     &    (-2.5403)    &       (-1.6454)       &     (2.1696)     &     (-2.4631)     &        (3.5341)        \\
\textbf{EE}            &     -0.0310     &        -0.0052        &      0.0128      &                   &                        \\
\textbf{ }                     &    (-0.5516)    &       (-0.0901)       &     (0.2183)     &                   &                        \\
\textbf{HU}            &      0.1481     &         0.1532        &      0.1704      &       0.1707      &         0.2133         \\
\textbf{ }                     &     (4.2572)    &        (4.4386)       &     (4.7729)     &      (3.3059)     &        (4.4070)        \\
\textbf{LT}            &      0.1197     &         0.1289        &      0.1690      &                   &                        \\
\textbf{ }                     &     (1.9310)    &        (2.0948)       &     (2.5981)     &                   &                        \\
\textbf{LV}            &      0.1550     &         0.1393        &      0.1892      &                   &                        \\
\textbf{ }                     &     (3.1199)    &        (2.7909)       &     (3.6864)     &                   &                        \\
\textbf{PL}            &     -0.0265     &        -0.0436        &     -0.0070      &      -0.0067      &         0.0305         \\
\textbf{ }                     &    (-0.8184)    &       (-1.3087)       &    (-0.2105)     &     (-0.1436)     &        (0.7012)        \\
\textbf{SI}            &      0.1506     &         0.0648        &      0.1796      &                   &                        \\
\textbf{ }                     &     (3.6747)    &        (1.0454)       &     (4.2319)     &                   &                        \\
\textbf{SK}            &     -0.1583     &        -0.1592        &     -0.1559      &      -0.1559      &        -0.1513         \\
\textbf{ }                     &    (-6.8684)    &       (-6.9814)       &    (-6.8721)     &     (-6.6327)     &       (-7.0800)        \\
\textbf{constant}                 &      0.4610     &         0.3808        &      3262.3      &       0.4392      &         6575.2         \\
\textbf{ }                     &     (5.8769)    &        (4.2735)       &     (2.1751)     &      (3.3976)     &        (3.5402)        \\
\textbf{UD}                    &                 &         0.0058        &                  &                   &                        \\
\textbf{ }                     &                 &        (1.8311)       &                  &                   &                        \\
\textbf{$Time^2$}             &                 &                       &     -0.0008      &                   &        -0.0016         \\
\bottomrule
\end{tabular}}
\end{table}




\begin{table}[!htbp]
\centering 
\caption{Fixed effect model comparison - CEE and Visegrad countries (T-stats reported in parentheses)}
\label{FE_models_comparison}

\begin{center}
\resizebox{\textwidth}{!}{\begin{tabular}{lccccc}
\toprule
                               & \textbf{CEE} & \textbf{CEE - Unions} & \textbf{CEE $t^{2}$} & \textbf{Visegrad} & \textbf{Visegrad $t^{2}$}  \\
\midrule
\textbf{Dep. Variable}         &   logW\textbackslash HL   &        logW\textbackslash HL      &     logW\textbackslash HL    &      logW\textbackslash HL     &        logW\textbackslash HL        \\
\textbf{No. Observations}      &     118      &          118          &       118        &         60        &           60           \\
\textbf{Cov. Est.}             &  Unadjusted  &       Unadjusted      &    Unadjusted    &     Unadjusted    &       Unadjusted       \\
\textbf{R-squared}             &    0.2007    &         0.3376        &      0.2044      &       0.3862      &         0.4030         \\
\textbf{R-Squared (Within)}    &    0.4070    &         0.4173        &      0.4128      &       0.5134      &         0.5526         \\
\textbf{R-Squared (Between)}   &    0.0472    &         0.2787        &      0.0498      &       0.2666      &         0.2624         \\
\textbf{R-Squared (Overall)}   &    0.2007    &         0.3376        &      0.2044      &       0.3862      &         0.4030         \\
\textbf{F-statistic}           &    14.434    &         19.369        &      9.7651      &       17.932      &         12.603         \\
\textbf{P-value (F-stat)}      &    0.0000    &         0.0000        &      0.0000      &       0.0000      &         0.0000         \\
\textbf{=====================} & ============ &      ============     &   ============   &    ============   &      ============      \\
\textbf{log(H\textbackslash L)}             &    0.0630    &         0.1080        &      0.0603      &       0.2082      &         0.1996         \\
\textbf{ }                     &   (2.1143)   &        (3.7546)       &     (2.0059)     &      (3.0956)     &        (2.9685)        \\
\textbf{trend}                 &   -0.0149    &        -0.0121        &      1.8574      &      -0.0246      &         4.2630         \\
\textbf{ }                     &  (-5.3300)   &       (-4.6309)       &     (0.7300)     &     (-5.9448)     &        (1.2497)        \\
\textbf{const}                 &    0.6265    &         0.5107        &      1868.9      &       0.7966      &         4279.5         \\
\textbf{ }                     &   (19.494)   &        (13.496)       &     (0.7361)     &      (10.969)     &        (1.2571)        \\
\textbf{UD}                    &              &         0.0079        &                  &                   &                        \\
\textbf{ }                     &              &        (4.8551)       &                  &                   &                        \\
\textbf{$Time^2$}             &              &                       &     -0.0005      &                   &        -0.0011         \\
\bottomrule

\end{tabular}}
\end{center}
\end{table}

\end{document}
