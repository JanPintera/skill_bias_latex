\documentclass[11pt]{article}
\usepackage[utf8]{inputenc}
\usepackage{graphicx}
\usepackage{booktabs}
\usepackage{caption}
\captionsetup[figure]{font=small,labelfont=small, width=.8\linewidth, name=Fig., labelfont=bf} %\footnotesize
\captionsetup[table]{font=small,labelfont=small, labelfont=bf}
%\usepackage[a4paper, total={170mm,257mm}, left=20mm,top=20mm]{geometry}
\usepackage[a4paper, total={6in, 8in}]{geometry}
\usepackage{placeins}
\usepackage{natbib}
\setcitestyle{authoryear,open={(},close={)}} %Citation-related commands
\usepackage{amsmath}



\makeatletter
\renewcommand{\maketitle}{\bgroup\setlength{\parindent}{0pt}
\begin{flushright}
  \textbf{\@title}\\
  \vspace{5mm}
  \@author\\
  \vspace{5mm}
  \@date
\end{flushright}\egroup
}
\makeatother

\renewenvironment{abstract}
 {\small
  \begin{flushleft}
  \bfseries \abstractname\vspace{-.5em}\vspace{0pt}
  \end{flushleft}
  \list{}{%
    \setlength{\leftmargin}{0mm}% <---------- CHANGE HERE
    \setlength{\rightmargin}{\leftmargin}%
  }%
  \item\relax}
 {\endlist}




\title{\begin{LARGE}Skill-bias and Wage Inequality in the EU New Member States: Empirical Investigation\end{LARGE}}
\author{    
    \begin{large}Jan Pintera\end{large} \\\vspace{5mm} \begin{small} Institute of Economic Studies, Faculty of Social Sciences, Charles University,\\ Prague, Czech Republic\\
    Email (corresponding author): jan.pintera@fsv.cuni.cz \end{small}
}
\date{October 2024}
\def \jel {	J30, J31, O14, O31, O33}
\def \Keywords {Labor Markets, Technological Change, Polarization, Skills}

\begin{document}


\maketitle

\thispagestyle{empty}
\begin{abstract}
We use individual-level data on income and education level from the EU-SILC database to investigate the trends in income distribution and wage polarization in the EU New Member States. We do not confirm the existence of job polarization in wages and employment that has been observed in the United States or other developed countries. We find a decreasing skill premium, particularly in Czechia, Poland, Hungary, and Slovakia, in the context of educational upgrading. This is contrary to the Skill-Biased Technological Change hypothesis and suggests that the substitution effect linked to the growing relative supply of skilled labor is primary in setting the wage dynamics. These results also imply a different impact of globalization on the labour markets in the EU New Member States than in other countries. However, it remains unclear whether these differences are temporary or will prevail in the future.
\bigskip

\textbf{JEL Classification:} \jel

\textbf{Keywords:}  \Keywords

\bigskip
\textbf{Statements and Declarations:}
This work was supported by the Grant Agency of the Czech Republic, Grant No. GACR no. 20-14990S.
\\

\textbf{Acknowledgements:}
We thank Bernd Fitzenberger for his helpful comments and suggestions.

\end{abstract}
\clearpage
\setcounter{page}{1}



\section{Introduction}
The echoes of labour market turmoil in developed economies have been heard quite often in recent decades. Fears of unemployment, job-quality deterioration, or, more specifically, the "hollowing-out" of the entire middle class appear in the latest government reports \citetext{e.g., \citealt{rodrik2020economic}} and can be documented in the declining relative position of the Western middle class in the world income distribution \citep{milanovic2020elephant}. These fears are often associated with rising populism and declining trust in democratic institutions.

There are two primary interpretations of the Western malaise that have emerged in the literature. The first interpretation focuses on the impact of international trade, as exemplified by the work of Autor and Dorn (2013). The second interpretation concentrates on the impact of technological change, viewed through the lens of relative demand shifts for labor. 

This paper focuses on the situation on the EU's Eastern flank, which presents both similarities and differences compared to the Western scenario. While these countries are gradually converging structurally with the West, they played a distinct role in the globalization push after 1990. The integration of Central and Eastern Europe (CEE) had an impact on the West comparable to that of China, as demonstrated by Dauth et al. (2013). On the other hand, the region's reliance on labor cost advantages may place it in a unique position regarding vulnerability to automation. Furthermore, the region as a whole is somewhat underperforming in terms of innovation, as evidenced by its subpar ranking on the Regional Innovation Scorecard (RIS).


Therefore, we use the individual-level data from the EU-SILC database to examine the evolution of wage inequality, wage polarisation, and the elasticity of substitution between low and high skill labour to evaluate the demand shifts on the labor markets in the CEE. Unlike their Western counterparts, the CEE countries have shown signs of declining wage inequality over the last decade \citep{magda2021firms}, often accompanied by low unemployment and overheating labour market.

In addition, there are at least two conflicting phenomena present in the CEE that are likely to produce outcomes different from those of developed economies. First, there is an educational upgrading happening in the region \citep{hardy2018educational}, suggesting significant structural change toward a knowledge-based economy. These developments contrast with US economy which is experiencing an educational slowdown \citep{goldin2010race}.

Second, the CEE countries, for their favourable unit labour cost and skilled workforce, seem to be ideal recipients of offshoring from high-wage economies. They also play a rather different role in the global value chains than developed countries \citep{baldwin2015supply}. The CEECs are typical examples of the "Factory Economies" strongly linked to their headquarter economy, Germany. 
A different position in the global value chains may also imply a different impact of global technological change than in the developed world, where the routine-intensive occupations decline can be linked to offshoring \citep{acemoglu2012does}.


Despite these dissimilarities, a populist drive comparable to the developed countries in the West can be seen in the new EU member states, with the populist parties' vote share tripling between 2000 and 2017. The agenda of these parties also bear similarities to the populists in the old EU countries \citep{orenstein2022work}. In Western countries, this rise of populist movement is often linked to the effects of globalisation and labour market polarisation. Given different labour market dynamics, the CEE countries provide an interesting case for understanding the link between income inequality and populism. 



From a methodological perspective, this work builds on the Skill-Bias Technological change hypothesis \citep{katz1992changes} that is based on the interplay of supply and demand for skill. The latter is driven by technological change, and the former by investment in human capital. This work will concentrate on potential causes of labour market inequality outlined by the Skill-biased Technological Change (STBC) hypothesis that despite its early origin \citep{katz1992changes} and empirical critiques seem to endure to these days \citep{aziz2021between, goldin2020extending}.

% REVIEW - add the comment to the paragraph
Further refinement of the SBTC hypothesis, the Routine-biased technological change (RBTC)  argues that technological changes in recent years has increased demand for high-skilled workers performing non-routine cognitive work while decreasing demand for workers performing routine jobs. As many of these workers are middle-skilled, RTBC allows for  job and wage polarization, phenomena seen in the US and other developed economies \citep{rodrik2020economic, temin2018vanishing}. Using the EU-SILC survey microlevel data from 2005 to 2019, this work brings descriptive and regression analysis of labour market polarization and other key labour market trends in the context of Central and Eastern Europe.


We further investigated the elasticity of substitution between high and low skill labour, which is one of the key features of the STBC framework \citep{katz1992changes}. One of the motivations for this approach was the significant skill upgrade seen in the region, a phenomenon visible in the US several decades earlier when the framework was found to perform well \citep{hardy2018educational}.

We aim to complement previous research on wage inequality in the CEE region by focusing on the relative supply and demand for high-low skill labor and technological change, which forms the core of the SBTC hypothesis. Our contribution lies in bringing a systematic study of the SBTC hypothesis to the relatively understudied CEE region. While existing literature on the SBTC in the region, such as \citet{tyrowicz2019wage}, concentrates on the countries' behavior during the economic transition or elaborates on routine-based task theories, we intend to study the Canonical model proposed by \citet{katz1992changes} during the pre-COVID EU experience of these countries. This will be done in two steps: descriptive analysis of the wage premia and examination of the model itself. We believe that the SBTC model is appropriate for this goal as it allows us to comment on key features of these economies, such as wage dispersion, education system performance, and the level of technological development.

Our main contribution to the literature is the direct application of the skill-biased framework to all CEE countries, instead of testing a particular subsection of the theory (such as routine-biased technological change, as in the case of \citet{arendt2019technical} or \citet{hardy2018educational}) or concentrating on a single country. Additionally, we utilize a different source of data, EU-SILC, which provides annual data for all countries of interest. We argue that EU-SILC, due to its annual form and coverage of labor supply, is the best fit for our purposes, similar to surveys used in US studies such as the Current Population Survey used by \citet{katz1992changes}.

To anticipate our results, we found that many of the phenomena identified in developed economies are not confirmed in the case of Central Europe. The general conclusion reached by our study is a good performance of the lower parts of the wage distribution. Perhaps most notably, we see a relative decline of the ninth decile in wages relative to median and mostly flat or monotonic behaviour of relative employment. This result contrasts with the characteristic U-shaped behaviour documented by \citet{acemoglu2012does} and interpreted as the job and wage polarization. Secondly, we did not find an evidence for skill-biased technological change in form of demand shifts, instead increasing relative supply drives the dynamics of the labor market. %We however also discover heterogeneity between the CEE countries.
%This seems to be in line with the view of the new member states as open economies with low unit labour costs which in environment of globalization leads to an inflow on relatively routine-intensive jobs, which drives demand for the low and middle type of jobs and has generally equalizing effect on the labour market.

The paper is organized as follows. The first part reviews the labour market development in the CEE, US and Germany. The second part discusses descriptive evidence on the CEE labour market, including evidence on polarization. The third part introduces the Skill Bias framework and outlines the details of the Canonical model. This part also presents the elasticity of substitution estimates and discusses the construction of variables necessary to obtain them. The fourth section concludes.

\section{Labour Market Developments}
\subsection{Wage Inequality Hypotheses - Case of Developed Economies}

The utilization of microdata for the investigation of long-run labour market trends dates mainly to the empirical estimation of the link between the U.S. skill premium and the relative supply of high/low-skilled labour by \citet{katz1992changes}. Their results led to the formulation of the so-called skill-biased technological change hypothesis, which explains changes in relative wages using a simple supply-demand framework (the "Canonical model"; \citealt{acemoglu2012does}) focusing on different levels of skills/education. The Canonical model in its original form is a simple and straightforward model that uses relative high/low skill labour supply and time-dependent "skill-biased" technological progress as a determinant of relative wages. Despite its simplicity, the existence of the link between technology and education as a determining factor in wage setting in the long term seems evident \citep{piketty2018capital}, and the Canonical model was shown to perform rather well in the US data before the 1990s \citep{katz1992changes}.

%%%%%%%%% REVIEW
Key complements of the skill-biased technological debate are the models of directed technical change \citep{acemoglu2002directed, acemoglu2002technical}, which assume an endogenous reaction of the demand curve in response to a supply shift. The main motivation of these models is to explain the change in the speed of skill-biased technological change happening in the environment of growing supply of skills, as seen in the US in the late 1970s \citep{acemoglu2002directed}.

Despite the model's performance in earlier decades, \citet{acemoglu2011skills} show that the Katz and Murphy's model overpredicts the skill premium in the 1990s and the 2000s. It also fails to account for several other stylized facts about the recent developments of wages in the US, most importantly, the job and wage polarization represented by strengthening tails of income/employment distribution. This process seems strongly connected to the automation of middle-skill jobs. Based on these findings, \citet{acemoglu2012does} proposes a comprehensive task-based framework (also routine-biased technological change, RBTC) focused on the level of routine content of the tasks involved and mapping between workers' skill and tasks performed by them. A formal representation of the task-based framework can be found in \citet{acemoglu2011skills}. This framework is capable of explaining the wage and job polarization phenomena observed in the 1990s in the US. 
%%%%=% REVIEW:
Canonical model representing SBTC can be understood and specific version of the the task-based framework assuming fixed mapping between skills and tasks.
On the other hand, as RBTC allows for changes in skill-task mapping, it gives it an ability to incorporate several styilzed facts observed in the US labor market since 1990s - apart from wage and labor polarization the RBTC is also able to incorporate real wage declines while the canonical model implies increases of real wages of both skilled and unskilled in results of factor-augmenting technological change.

\citet{mishel2013assessing} postulate three testable hypotheses derived from the skill-biased technological change literature and its extensions. First, the labour supply and demand interactions determine wage formation. More concretely, technological change causes shifts in labour demand which, in turn, affect wages. This causality can be considered a general feature of the framework common to both the original skill-biased technological change and its subsequent variant, the routinization-biased technological change. Second, from the empirical point of view, the skill-biased technological change leads to job and wage polarization - phenomena highlighted by \citet{acemoglu2012does}, \citet{howell2019declining} and others when discussing the developments of the Western labour markets in the last decades of the 20th century. Note that at this point, both variants of the technological framework differ, with the original SBTC hypothesis predicting monotonic employment and wage development across the occupational distribution, a phenomenon observed in the 1980s. Third, the RBTC hypothesis implies a rise in both employment and wages in a specific type of services - namely, the low-wage service jobs characterized by manual non-routine content.

%Note that the skill-bias framework always faced critiques such as \citet{mishel2013assessing}, who, rather than explicitly denying the underlying "job polarization" trend, map it to a much earlier time and thus deny its causal link with inequality rise after the 1970s. The link between job polarization and wage polarization is therefore in question. In the interpretation of \citeauthor{mishel2013assessing}, technological changes have a significant impact on occupation composition, not on wage inequality. They also point to a general wage deficit - the inability of wages to keep up with productivity growth and rising profits after the 2000s.

 Over time, job polarization has been identified in many developed economies \citep{rodrik2020economic, oecd2017}. Consequently, it has become accepted as one of the defining features of the developed economies' labour market \citep{howell2019declining}. On the other hand, as shown by \citet{mishel2013assessing}, job polarization in the US economy seems to be a phenomenon linked firmly to the 1990s, and already the early 2000s brought a slowdown in both education premium and high-occupation rise. Therefore, we can also formulate the difference between RBTC and SBTC as the difference between the 1980s and 1990s US labour market. On the other hand, both the polarization itself and the declining position of the middle class in general are not limited to the US or a single time period \citep{temin2018vanishing, rodrik2020economic}. %Moreover, deeper troubles in Western labour markets can be seen in declining job quality \citep{howell2019declining}, disappearing middle-class \citep{temin2018vanishing}, as well as a relative decline in the position of the Western medians in the world income distribution \citep{milanovic2020elephant}.



\subsection{Labour Market Inequality in Central and Eastern Europe}
% TODO: they really speak about SBTC?
Until recently, the micro-data on income and inequality have been used for assessment of income inequality in a small number of developed countries.

With newly available micro-data, there have been several attempts to investigate the role of skills in income inequality among the CEE countries so far. \citet{arendt2019technical} studied the wage premium in Poland using a multilevel model, and \citet{hardy2018educational} provide an analysis of task content development in the EU following \citeauthor{acemoglu2011skills}'s (2011) approach and provide analysis of labour supply development in the EU-24 with the emphasis on the CEE countries. Both papers investigate the validity of RBTC theory by focusing on the task-content division of the labour force (classification of jobs according to a required level of cooperation and creativity).

The results of both Hardy et al. (2018) and Arendt and Grabowski (2019) point toward certain deviations of this region from the rest of Europe in terms of the task distribution. Namely, according to Hardy et al. (2018) we see an increase in routine cognitive tasks in CEE countries, which is contrary to both the old EU countries and routine-replacing technological change hypothesis, similarly Arendt and Grabowski (2019) find relative wages in routine manual jobs in Poland too high for the RBTC hypothesis to hold. 

Both, RBTC focused studies then note significant educational upgrading in the region, especially the rapidly increasing tertiary education attainment (Hardy et al., 2018). We should note that at least in this aspect the CEE countries seem to differ significantly from the U.S. labour market, where as noted by \citet{acemoglu2012does}, high-school attainment is actually stagnant since the 1960s and post-secondary attainment decelerated already in the 1970s.

The specificity of the CEE income inequality with respect to the West is also confirmed by  other inequality focused studies. \citet{magda2021firms} notes the decrease of wage inequality in the CEE in the 2002-2014 period. Before this period, the CEE countries experienced significant inequality rise due to their economic transformation, but the inequality leveled since then \citep{tyrowicz2019wage} with evidence of wage inequality staying lower than in the developed countries \citep{mysikova2018personal}. This conclusion is also confirmed by a recent study by Magda et al. (2021), who find generally decreasing levels of wage inequality in the CEE using the EU-SES database for 2002-2014, with the only country with a slight increase in wage inequality being the Czech Republic. The authors note that this finding stands in contrast to the development found in Western countries.% From our point of view, this speaks in favour of analyzing the validity of the skill-bias hypothesis in this region.

We should also note that the CEE region has specific characteristics compared to Western Europe. Above all, thanks to their comparative advantage in labour costs, the CEE countries have a different position in the globalization process than countries of Western Europe. Giving a closer look at the relationship between the West and the CEE region shows that the region primarily serves as a pool of relatively cheap and qualified labour to Germany, strongly influencing Germany's internal labour market in return 
\citep{marin2004nation, marin2018global}. 
%%%%%%%%%%% REVIEW - commented out
%Notably, \citet{glitz2021skill} has shown that after breaking German the population to three education levels and two age groups and using the procedure developed by \citet{katz1992changes} and \citet{card2001can}, they find that the labour supplies are to large extent able to explain the changes in skill premium in Germany. They find especially pronounced rise in skill premium of medium skilled to low skilled and link it to decline in the share of population with vocational training. Their findings are therefore very much in line with the original SBTC framework and can be seen as reaching a similar result as the seminal work of \citet{goldin2010race}.

%%%%%%%%%%%%%%%%%%%%%%%%%% REVIEW
This study brings twofold contribution to the CEE-focused literature. First, it complements the studies concentrated on RBTC. In light of the deviations from the theory found by \citet{arendt2019technical} and \citet{hardy2018educational}, we investigated job and wage polarization. These phenomena are not accounted for by the Canonical model and their appearance in the US led to the development of RBTC and served as main motivation for developing this version of the hypothesis \citep{acemoglu2011skills}. Looking at CEE, evidence on polarization is limited, with the main contribution being \cite{mysikova2018personal} investigating the phenomenon in context of the Financial crisis (2004-2010) in Poland and Czech republic, while not confirming the polarization hypothesis.

Secondly, we study the impact of the educational upgrading using the supply and demand framework supplied by \cite{katz1992changes} and others.  Increasing supply of university graduates, is considered a key phenomenon in the regional labor market \citep{arendt2019technical, hardy2018educational} with \cite{magda2021firms} note that educational upgrading is a key factor in decreasing returns to tertiary education, contributing to reduction in wage inequality. We believe that the canonical model is the ideal framework for evaluation the interplay of relative supply and demand as crucial explanatory factor for skill premium in CEE. It can also suggest the extent of the directed technological change is present in the region. While increase in relative supply of skills has been documented in the US, the evidence so far does not suggest that it produces similar - skill premium increasing effect in the CEE, there is a question to which demand has responded to the increase in relative supply \cite{acemoglu2002directed}. 

We believe that testing SBTC is especially relevant in the region given deviations from the RBTC mentioned by \cite{arendt2019technical, hardy2018educational} but also relevant from the policymaker viewpoint as it directly links the outcome of education system, which is under direct influence of policimakers, to labor market inequality (see \citep{goldin2010race}). 

The SBTC supply-demand framework is still widely in use as documented by recent works of \cite{glitz2021skill}, \cite{aziz2021between} or \cite{farber2021unions}, who use the model for empirical investigation or just develop its extensions. To our best knowledge, the canonical model in the form of the underlying supply-demand (canonical) framework has never been evaluated in the region to full extent.


%%%%%%%%%%%%%%%%%%%%%%%%%%%%%%%%%%%%

%\subsection{Germany}

%Germany on the other hand - underwent a profound labour market transformation in recent decades (Marin et al., 2018). Dustmann et al. (2014) attribute its labour market resilience to a unique set of labour market institutions - most notably its decentralized and de-politicized wage bargaining process that allowed for labour market flexibility in face of adverse external macroeconomic conditions. Concretely, Dustmann et al. (2014) concentrate their analysis on wage restrains of German workers which can be reflected in behaviour of German unit labour costs with respect to the other Western countries. Dustman et. al (2014) also note decreasing real wages at the lower end of the wage distribution after mid-1990 but not before. He attributes this to the German interaction with the CEE countries - that served as a pool of comparatively cheap skilled labour that helped German businesses and in turn allowed to put pressure on German workers. Marin (2004) also interestingly notes that German were the offshoring skilled rather unskilled work to the new countries, in an attempt to solve its own low human capital endowment shortages.

%In general, Germany has been, similarly to the US and other Western economies, experiencing rising income inequality at least since the 1980s (Biewen, Fitzenberger and Lazzer, 2017). However recent development point to certain specificity of the country's development. 

%Biewen and Sturm (2021) find that the inequality has been stagnating since 2005 and attribute this phenomenon to recent labour market boom. Schank and Bossler (2020) concentrating on the lower tail of the distribution find a rising wage inequality in 2000s and declining trend after 2010s, furthermore they observe a sharp drop in inequality after 2014, that they attribute to minimum wage introduction.

%Giving a closer look at the development of inequality in different parts of the income distribution Biewen and Seckler (2019) do not confirm wage polarization found in the US. They find much more monotonic development with the highest percentiles gaining relatively the most. This is confirmed by Biewen, Fitzenberger and Lazzer (2017) who document rise in inequality limited to the top part of the distribution (development in line with with the SBTC hypothesis) until mid 1990s with the labour market institutions preventing rise of the inequality at cost of higher unemployment  - later it rose across the entire distribution.

%Biewen and Sturm (2021) comments on German development after recent labour market boom (since 2005) and find it having an equalizing effect - it led to income gains across the distribution and the lower part of the distribution experienced bigger gains than the upper parts, despite institutional and external factor dampening this effect.


%As far as the causes of the inequality rise is concerned the literature has so far not reached a conclusion. Yet among the most often mentioned reasons are labour market institutions (decline in collective bargaining) and composition changes (such as educational upgrading, labor market history, industry structure, and occupation) (Biewen, Fitzenberger and Lazzer, 2017). Emphasized is also a significant wage restraint by German workers reflected in behaviour of German unit labour costs with respect to the other Western countries (Dustmann et al., 2014) and the role played in the CEE countries that served as a pool of comparatively cheap skilled labour that helped German businesses and in turn allowed to put pressure on German workers (Marin 2004, 2018).

%In terms of the SBTC, Biewen, Fitzenberger and Lazzer (2017) find strong influence of composition changes in explaining rising inequality - education (especially in the upper part of the distribution) and changes in recent labor market histories (lower part) and conclude that this finding is in line with SBTC hypothesis.
%Last but not least, Glitz and Wissman (2021) show that after breaking German population to three education levels and two age groups and using the procedure developed by Katz and Murphy (1992) and Card and Lemieux (2001), they find that the labour supplies are to large extent able to explain the changes in skill premium in Germany. They find especially pronounced rise in skill premium of medium skilled to low skilled and link it to decline in the share of population with vocational training. Their findings are therefore very much in line with the original SBTC framework and can be seen as reaching a similar results as the seminal work of Goldin and Katz (2009). %TODO: Glitz and Wissmann have some conclusions about Polarization, add those here


\section{Trends in Wage and Employment Development}
\subsection{Labour Market Polarization?}\label{wage_analysis}
To analyse trends in labour market developments, we rely on the Eurostat EU-SILC database with data also from the CEE countries (Czech Republic, Slovakia, Poland, Hungary, Latvia, Estonia, Lithuania, Romania, and Bulgaria) between 2005 and 2019. The EU-SILC collects information about income, poverty, social exclusion and living conditions across the EU countries. Our dataset contains more than 2 million observations in total.
%%%%%% NEW - START: 
Compared to other potential data sources, such as the EU-SES survey, we selected SILC because it provides both income and wage data of sufficient quality, as well as information on the amount of labor performed, which are necessary for the estimation of the SBTC model. 
%%%%%%% NEW - END

For presentation purposes, we aggregated the countries into three regional blocks, the Central Europe (the "Visegrad" countries), the Baltics (Latvia, Estonia and Lithuania) and two Balkan countries, Romania and Bulgaria. This division was inspired by the geographical and socioeconomic proximity of the countries.\footnote{We add Slovenia to the Central European countries, as its macroeconomic performance is much closer to them than Romania and Bulgaria.} The results below refer to these regional blocks, results for individual countries are presented in the Appendix.\footnote{Note that we calculate the aggregated metrics by pooling all observations in a region together and then treating it as a single territory, i.e. the statistics below are not averages of individual countries' statistics in the Appendix.} We choose 2011-2019 as our main period of interest due to data limitations (data for Bulgaria and Romania and various variable changes) and also in order to concentrate on the period after the Great Recession.


% Aggregate - Table 1
\begin{table}[!htbp]
\centering 
\caption{Changes in Real Wages for Different Groups of Countries}
\label{real_wage_changes_agg}
\begin{center}
\resizebox{\textwidth}{!}{


\begin{tabular}{lrrrrrr}
\toprule
{} &     RO \& BG &     RO \& BG &   Cent. Europe &   Cent. Europe &     Baltics &     Baltics \\
{} &  2010/2007 &  2019/2011 &  2010/2007 &  2019/2011 &  2010/2007 &  2019/2011 \\
%country\_group &     balkan &     balkan &   visegrad &   visegrad &     baltic &     baltic \\
Groups                                &            &            &            &            &            &            \\
\midrule
%(upper) secondary education           &   9.310017 &  59.969078 &  11.661339 &  13.946453 &  10.444225 &  35.929474 \\
$<$5                                    &  14.581693 &  65.260985 &   7.309054 &   8.120131 &  -2.965103 &  36.511753 \\
5-15                                  &   7.861526 &  63.277764 &   9.162507 &  -0.852100 &  13.541349 &  31.962220 \\
15-25                                 &   8.865883 &  66.529665 &   9.287660 &   7.086426 &  16.848990 &  38.034179 \\
25-35                                 &   8.085102 &  58.722765 &   9.636790 &   0.469503 &  20.252799 &  33.171526 \\
35-45                                 &   5.013863 &  61.656307 &   9.998737 &   2.844017 &  18.044866 &  28.720680 \\
%5-15                                  &   7.861526 &  63.277764 &   9.162507 &  -0.852100 &  13.541349 &  31.962220 \\
%<5                                    &  14.581693 &  65.260985 &   7.309054 &   8.120131 &  -2.965103 &  36.511753 \\
$>$45                                   & -64.982239 &  91.008855 &   1.551521 &   3.060651 &  20.611585 &  35.019766 \\
Female                                &   8.017022 &  66.750833 &   7.859075 &   2.333832 &  16.875744 &  32.899632 \\
Male                                  &   8.006975 &  60.959955 &   9.803801 &   4.163552 &  10.660589 &  35.193105 \\
Primary Education                     &   2.588869 &  70.827861 &  -4.752946 & -45.600881 &  23.304905 &  70.928672 \\
Lower Secondary Education             &   1.118504 &  55.677731 &  13.446450 &  15.638981 &   6.002626 &  27.930662 \\
(upper) Secondary Education           &   9.310017 &  59.969078 &  11.661339 &  13.946453 &  10.444225 &  35.929474 \\
Post-secondary Non-tertiary Education &   5.591391 &  77.938633 &   9.465074 &   2.105854 &  16.559132 &  32.032608 \\
%primary education                     &   2.588869 &  70.827861 &  -4.752946 & -45.600881 &  23.304905 &  70.928672 \\
Tertiary education                    &   8.442430 &  65.136792 &   7.749198 &  -5.664850 &  14.562024 &  34.033161 \\
\bottomrule
\end{tabular}

}
\caption*{\footnotesize The table presents log changes in real monthly wages of full-time workers between the given years. We use the mean wages of the sex-education-experience groups defined above. The aggregated categories displayed are then weighted averages of relevant groups using the groups' average employment shares over the entire sample period as weights. We calculate the real wages by deflating nominal wages in each period by the country's Harmonised index of consumer prices obtained from Eurostat. To get the results for broader regional groups (as displayed above), we first calculate an average of respective countries' real wages for each sex-education-experience group. }
\end{center}
\end{table}




% Table 1 and Table 2 - wage and LS changes tables with commentary
Table \ref{real_wage_changes_agg} shows log changes in real wages for different groups of full-time workers in two time periods, 2007$/$2010 and 2011$/$2019.\footnote{Tables \ref{real_wage_changes_ce} and \ref{real_wage_changes_bb} describe the changes for individual countries.} The changes are calculated for males and females, five education categories, and six experience groups. First, we observe quite strong real wage growth for the highest education category (tertiary education or higher) in all regions apart from Central Europe, where we find a slight decline between 2011 and 2019. Table \ref{real_wage_changes_agg}, however, shows that in none of the regions are the wages of the tertiary educated the fastest growing. In all regions, the wage growth of the secondary educated surpasses the growth in the tertiary education categories.

%Notably, we observe robust growth across all categories in the Baltics and Balkan countries. On the other hand, the situation in Central Europe is somewhat different. The wages of both the highest and lowest education categories decline, which contrasts to wage increases across secondary education.\footnote{Note, however, that the results in the primary education category are driven by a relatively low number of observations and only two countries - Poland and Slovakia.} The CEE experience, therefore, contrasts with the US experience, where the less-educated workers experienced real wage declines \citep{acemoglu2011skills}.

Similarly, Table \ref{labour_supply_changes_agg} shows changes in relative labour supply.\footnote{Tables \ref{labour_supply_changes_bb} and \ref{labour_supply_changes_ce} describe the results for individual countries} Concretely, we depict log changes in each group's share of total labour supply measured in efficiency units.\footnote{Efficiency units are defined in section \ref{KM_vars}.} The results show a steady rise in female share in the labour supply and a similar increase for the highest education categories across the regional groups. Among the experience categories, we see a rising labour supply for the higher experience groups and declines in the case of workers with less than five years of working experience, probably a sign of population ageing.

%%% Fitzenberger's review %%%%% Moved below
%Table \ref{labour_supply_changes_agg} therefore points to an increasing relative supply of tertiary educated while Table \ref{real_wage_changes_agg} shows evidence of stagnating or even declining relative wage premiums. This evidence does not exclude the possibility of relative supply changes as a primary factor of the relative wage dynamics between high and low skill population (as exhaustively examined by \citet{katz1992changes}).

%%%%%%%%
% Aggregate - Table 2
% TODO - rows ordering
\begin{table}[!htbp]
\centering 
\caption{Changes of Labour Supply for Different Groups of Countries.}
\label{labour_supply_changes_agg}
\begin{center}
\resizebox{\textwidth}{!}{


\begin{tabular}{lrrrrrr}
\toprule
{} &  2010/2007 &  2019/2011 &  2010/2007 &  2019/2011 &   2010/2007 &  2019/2011 \\
{} &     RO \& BG &     RO \& BG &   Cent. Europe &   Cent. Europe &      Baltics &     Baltics \\
Groups                                &            &            &            &            &             &            \\
\midrule
%(upper) secondary education           &  -0.856678 &  -1.930573 &  -5.771289 &  -2.719726 &  -12.278409 & -17.813729 \\
$<$5                                    &  -2.383018 & -31.097151 &   2.249316 & -43.438851 &   -4.427445 & -16.999040 \\
5-15                                  &  -4.671684 &  -3.833148 &   0.796531 &  -2.034299 &    5.413153 &  17.826651 \\
15-25                                 &   6.985925 &  -3.933818 &  -7.453282 &   8.174225 &   -8.145433 & -18.880740 \\
25-35                                 & -10.210762 &   8.558053 &  -1.978402 &  -4.703112 &    4.217927 & -15.607869 \\
35-45                                 &  25.112974 &  12.351141 &  21.275877 &  32.029868 &   10.256839 &  38.920874 \\
%5-15                                  &  -4.671684 &  -3.833148 &   0.796531 &  -2.034299 &    5.413153 &  17.826651 \\
%$<$5                                    &  -2.383018 & -31.097151 &   2.249316 & -43.438851 &   -4.427445 & -16.999040 \\
$>$45                                   & -36.640996 &  59.656172 &  14.065841 &  71.537131 &  -19.632433 &  48.323612 \\
Female                                &   1.296522 &  -1.930132 &   4.091550 &   2.007874 &    8.494020 &  -4.176115 \\
Male                                  &  -0.795805 &   1.223618 &  -2.551984 &  -1.325431 &   -6.453240 &   3.177369 \\
Primary Education                     & -50.254843 & -14.738936 & -20.863042 & -50.979049 &  -12.331598 & -32.205389 \\
Lower Secondary Education             &  -7.603790 & -26.757482 &  -1.851176 &  24.034607 &  -12.826862 & -32.896778 \\
(upper) Secondary Education           &  -0.856678 &  -1.930573 &  -5.771289 &  -2.719726 &  -12.278409 & -17.813729 \\
Post-secondary Non-tertiary Education &  -1.729285 &   5.109752 &   0.797851 & -35.876316 &  -30.872638 &  12.805771 \\
%pre-primary education                 &  59.345873 &       -inf & -91.783450 &       -inf & -117.468738 &       -inf \\
%primary education                     & -50.254843 & -14.738936 & -20.863042 & -50.979049 &  -12.331598 & -32.205389 \\
Tertiary education                    &   8.452129 &  10.272401 &  11.712725 &   7.441955 &   20.472786 &  10.584279 \\
\bottomrule
\end{tabular}

}
\caption*{ \footnotesize The table presents log changes in the share of total labour supply provided by a given group in a specified period. The labour supply is measured in the efficiency units.}
\end{center}
\end{table}

% Figure 1 and 2 - here % REVIEW STRECO: Commented-out
%Figures \ref{low_deciles_vs_min_w} and \ref{high_deciles_vs_meam_w} compare selected sample wage percentiles with the growth of minimum and average wage in the economy. The pictures show the contrast between the volatile 1st percentile and the steadily growing rest of the distribution without significant divergence or convergence.

%\begin{figure}[!htbp]%
%    \centering
%    \caption{Minimum Wage against the Lowest Percentiles}
%    {\includegraphics[scale=0.5]{agg_min_wages.png} }
%    \label{low_deciles_vs_min_w}
%    \caption*{\footnotesize Deciles of monthly log wages of full-time workers. The minimum wage statistic is obtained from Eurostat, the rest is calculated from the EU-SILC survey data. We used an average of the official minimum wage figure for the countries in each region as the final minimum wage.}
%\end{figure}

%More concretely, Figure \ref{low_deciles_vs_min_w} shows the development of the log minimum wage against the first percentile and first decile of the wage distribution (we use monthly log wages of full-time workers). We can see that the minimum wage closely copies the first decile of our sample. Figure \ref{high_deciles_vs_meam_w} then portrays a similar picture for the upper segments of the distribution and log of the average wage for a given region. We see the depicted lines going mostly in parallel with the exception of mean and median converging around 2008 in Central Europe and Baltics and in the most recent period in Romania and Bulgaria. Overall, the comparison of both figures shows a significantly higher variation of the lowest percentile compared to the higher ones. This is especially visible in the case of the Baltics, as well as in Romania and Bulgaria.

%\begin{figure}[!htbp]%
%    \centering
%    \caption{Average Wage against the Highest Percentiles}
%    {\includegraphics[scale=0.5]{agg_high_deciles_against_mean.png} }
%    \label{high_deciles_vs_meam_w}
%    \caption*{\footnotesize Deciles of monthly log wages of full-time workers. The average wage statistic is obtained from Eurostat, the rest is calculated from the EU-SILC survey data. We used an average of the official mean wage figure for the countries in each region as the final mean wage.}
%\end{figure}

% Figure 3 
% TODO: 50/10 seems to be significantly more volatile than 90/50 
To sum up the general development of labour market inequality, Figure \ref{agg_wage_gaps_CEE} depicts the development of  50/10 and 90/50 wage gaps (ratio of percentiles of a monthly log wage distribution) for full-time workers in the three regions defined above.
Figure \ref{agg_wage_gaps_CEE} shows that the pay gaps developed differently in the three regions, despite all being generally different from the US data as found, for example, in \citet{mishel2013assessing}. 

\begin{figure}[!htbp]%
    \centering
    \caption{Development of (Log) Wage Gaps for Full-time Workers,  2005–2019}
    {\includegraphics[scale = 0.5]{wage_gaps_agg.png} }
    \label{agg_wage_gaps_CEE}
    \caption*{\footnotesize Development of ratios of different deciles of log monthly wage distribution for full-time workers. }
\end{figure}

In the case of Central European countries, there is a tendency for a relatively long-term and monotonic decrease in the 90/50 wage gap, which implies decreasing income inequality between the 90th decile and the median, a finding that contradicts the US and Western European evidence as well as the hypothesis of wage polarization. We can also note the mostly decreasing tendency of the 50/10 ratio, a movement more in line with the US evidence as the least paid jobs seem to be catching up with the median. A similar tendency can be observed for the Baltic countries, even though the 90/50 curve is significantly flatter in this case and the 50/10 curve more volatile, which is exemplified by a steep rise in wage inequality after the crisis in 2008. The impact of the crisis is also visible, yet less pronounced, in Central Europe. The Balkan countries, on the other hand, experienced a rather flat 90/50 ratio after 2012 and a reversed trend in the 50/10 ratio in recent years. Nevertheless, there was no negative reaction to the 2008 crisis, with the ratios continuing to decline around the year 2010.

Increasing real wages and decreasing wage gaps imply that, especially in Central Europe, the growth of the recent decades led to improved incomes of the median earners as well as people from the 10th income percentile. The same development is to a lesser degree present in the other two regions. The development also contrasts with Germany, where a general upward trend for both wage gaps (90/50 and 50/10) was visible at least until 2015 \citep{biewen2021labour}. Whereas the median seems to be either gaining or at least keeping its position with respect to the top, the lower part of the distribution is much more volatile and seems to react more to changes in the business cycle.


%Figure 4 - look at Acemoglu, 2012 and so on I consider extending the graph description if necessary.
Next, we focus on changes in log wage percentiles of full-time workers relative to the median between 2011 and 2019, to check for polarization pattern in the region. Figure \ref{agg_wage_changes_percentiles_11_19} shows monotonic behaviour for the Visegrad counties with a clear tendency for a decline of the highest percentiles relative to the median. A Similar yet less pronounced picture is visible for the Baltics, whereas the same plot for Romania and Bulgaria shows a rather contrasting picture with declining lowest percentiles and a tendency to increase for the two highest deciles.\footnote{Figures in the Appendix show that the growing inequality seems to be driven by development in Bulgaria (see Figure \ref{wage_changes_percentiles}), whereas Romania resembles development in Central Europe.} If we compare our results to Western evidence, we can notice that development in Romania and Bulgaria seems to be closest to the German scenario and the US scenario in the 1980s. Predominately monotonic behaviour is, however, common to all three regions.\footnote{Note that Figure \ref{agg_wage_changes_percentiles} shows that for the 2007-2019 period, the same graph for the Balkan countries shows a declining tendency for the lowest percentiles and a rather flat behaviour afterwards. Development in the Baltics is then more flat than in Figure \ref{agg_wage_changes_percentiles_11_19}} This is an important conclusion as a similar plot for the US shows a characteristic U-shaped curve with the increases concentrated at the ends of the distribution, such behaviour is interpreted as wage polarization \citep{acemoglu2011skills}. 

\begin{figure}[!htbp]%
    \centering
    \caption{Changes in Log Wages by Percentile Relative to the Median (2011-2019)}
    {\includegraphics[scale=0.5]{agg_wage_changes_percentiles_11_19.png} }
    \label{agg_wage_changes_percentiles_11_19}
    \caption*{\footnotesize The figure shows how given percentile of log monthly wage distribution changed relative to the median between 2011 and 2019. The data are for full-time workers. Formally, we depict $\log(\frac{P_{2019}^{n}}{P_{2019}^{50}}) - \log(\frac{P_{2011}^{n}}{P_{2011}^{50}})$ for each percentile $n$ of the distribution. Note that in line with \citet{acemoglu2011skills} we depict the 5th-95th percentile.} 
\end{figure}

% Figure 5
\begin{figure}[!htbp]%
    \centering
    \caption{Changes in Employment by Occupational Skill Percentile, 2011–2019}
    {\includegraphics[scale=0.5]%{agg_employ_changes_percentiles.png} }
    {agg_employ_changes_percentiles_no_weighting.png} }
    \label{agg_employ_changes_percentiles}
    \caption*{\footnotesize The vertical axis depicts a change of employment (annual hours worked) in each occupational percentile as a share of total employment in a given region. The occupations are ranked by skill percentiles obtained using mean log wage for each occupation in 2011. A Line representing a locally weighted smoothing regression is also depicted. All employment share changes are multiplied by 100. Also, note that the y-axis range omits some extreme values to better display the smoothed regression.   }
\end{figure}
%Figure \ref{agg_employ_changes_percentiles} comments on a crucial polarizing behaviour in the CEEC. It shows changes in employment shares for the ISCO-08 occupations skill rank between 2011-2019\footnote{The starting year of the analysis is chosen due to changes in ISCO classification in the EU-SILC dataset.}, it also depicts a locally weighted smoothing regression curve. We can notice a rather different and diversified, yet mostly monotonic, behaviour across the CEE countries. There is a declining tendency for the employment share of high-income occupations, especially in the case of Central Europe. To a lesser degree, we see this behaviour in the Baltic states, with most of the percentiles below the eighth decile being predominately flat. Romania and Bulgaria again represent a certain outlier showing rising employment in high-income occupations. Note that there is also a tendency for employment shares to increase for the lowest percentiles in the two countries, which would indicate the existence of a certain level of polarization. However, the magnitude of these changes is low compared to the changes in the highest percentiles. We can also note an increasing variance of the estimates in Balkan and Baltics in the upper half of the distribution - this makes the conclusions for high percentiles for Baltics and the two Balkan countries less reliable.\footnote{This is also supported the by development visible in Figure \ref{employ_changes_percentiles} where the highest percentiles for both Bulgaria and Romania seems to be rather flat.}


%%%% Fitzenberger's review --- Fig.5 paragraph rewrite

Figure \ref{agg_employ_changes_percentiles} investigates the posibility of job polarization in the CEECs. It presents changes in employment shares for the International Standard Classification of Occupations (ISCO-08) skill rank between 2011 and 2019\footnote{The analysis begins in 2011 due to changes in the ISCO classification in the EU-SILC dataset.} and depicts a locally weighted smoothing regression curve. The behavior of the employment shares across the CEECs is quite diverse. Central Europe stands out with a monotonous increase in the relative employment share of high-income occupations. In contrast, the locally-smoothed regression curves in the remaining two regions are relatively flat, with minor increases around the third decile. However, the magnitude of these changes is small, and there is considerable variance in the estimates in the middle of the distribution, particularly in the Baltics. Furthermore, it is important to note that our results may conceal significant heterogeneity within the country groups\footnote{This is evident in Figure \ref{employ_changes_percentiles}, with some of the Visegrad countries displaying minor declines in high-income categories and diverse results for the Baltics.}.
%%%% Fitzenberger's review --- Fig.5 paragraph rewrite

Last but not least, compared to the US evidence, we again do not see the  U-shape found in \citet{acemoglu2012does} for the 1990s and 2000s US labour market, interpreted as the job polarization.


%%%%% Fitzenberger's review - end
%%%%%% REVIEW
While there are other dimensions of inequality to be discussed within the skill groups, we believe our analysis has shown differences between the CEE labour markets and the key stylized facts found in developed countries. In particular, we do not confirm either job or wage polarization in the CEE, and we never see a monotonic rise in inequality similar to certain periods of the US development. Yet, at the same time, there is rather a diversified behaviour between the investigated regions themselves. In general, we can contrast declining measures of inequality in Central European countries with the mostly stagnating situation in the Baltic and signs of increasing inequality in the case of the Balkan countries.
%The difference between the investigated countries comes as a certain surprise to us. All the countries are relatively low-wage and should have a similar position in the global supply chain relative to the "Headquarter economies" such as Germany or the US \citep{baldwin2015supply}. We should also note that the Central European and Baltic countries have a very similar labour market type. According to \citet{/content/publication/1fd2da34-en}, their bargaining systems can, in all cases, with the exception of Slovenia, be classified as fully or largely decentralized (Romania and Bulgaria are not covered). Overall, our results show that the labour income inequality alone cannot explain populist movements in the CEE.

\subsection{Sectoral and Occupational Analysis}
% Figures 6 - 9 describing NACE/ILO codes decomposition for wages and employment
Figures \ref{wage_changes_nace} and \ref{employ_changes_nace} show relative changes in wages and employment for the NACE classification of economic activities. Particularly, robust wage growth in manufacturing and construction in Central European countries stands out and seems to be in line with the strong position of these countries in the European manufacturing core \citep{stollinger2016structural}. On the other hand, the key public sectors are falling behind in this region. In the other two regions, we see strong performance of Finance and ICT in Baltic countries, while in Romania and Bulgaria, the public sector's relative wages are rising. These findings point to significant structural differences among the economies, \citet{stollinger2016structural}, for example, notes  that the Manufacturing core includes central European states, yet not other countries in our data.  Figure \ref{employ_changes_nace} then documents that changes in relative wages in the NACE categories are often associated with moves in relative employment in the opposite direction (see Construction in Central Europe or Finance in the Baltics).


% some sources: https://www.oecd.org/economy/surveys/bulgaria-2021-OECD-economic-survey-overview.pdf
%https://www.oecd-ilibrary.org/sites/bf4a7892-en/index.html?itemId=/content/component/bf4a7892-en#chapter-d1e1343
% overview: https://ec.europa.eu/eurostat/web/products-eurostat-news/-/DDN-20190917-1
%Gini: https://data.worldbank.org/indicator/SI.POV.GINI?end=2018&locations=BG-CZ-RO-HU&start=2002&fbclid=IwAR1PFRKBUVf-Gmq4NdR1r4bjuQrNCrfQXIst_SQcG97p1dcpmO7WNy5heag

\begin{figure}[!htbp]%
\centering
    \caption{Changes in Log Wages Relative to the Median by NACE category (2011-2019)}
    {\includegraphics[scale=0.5]{wage_changes_nace.png} }
    \caption*{\footnotesize The Figure shows changes in mean log monthly wages in NACE categories relative to the median wage. We use wages of full-time workers and NACE Rev. 2 categories (sections) of economic activity.}
\label{wage_changes_nace}
\end{figure}

\begin{figure}[!htbp]%
\centering
    \caption{Employment Share Changes between 2011-2019 by NACE Category}
    {\includegraphics[scale=0.5]{employ_changes_nace.png} }
    \caption*{\footnotesize The Figure shows changes in labour supply (using hours worked) shares of different NACE Rev. 2 categories in total labour supply. All workers who worked at least one month in a given year were used.}
\label{employ_changes_nace}
\end{figure}


When we compare similar plots for changes in employment share using major ILO employment categories in Figures \ref{wage_changes_ilo} and \ref{employ_changes_ilo}, we again do not find an unified picture among the regions. Figure \ref{employ_changes_ilo} allows for a basic comparison with trends both in the US and Western Europe, where we, in line with the job polarization hypothesis, find high- (Managers, Professionals and Technicians) and low-education occupations (Elementary and Services \& Sales) growing at the expense of the middle-education occupations such as clerks, machine operators, and crafts and trade jobs (Acemoglu and Autor, 2011). We see similar behaviour for Baltic states with Managers and Professionals growing in the relative employment share together with Elementary occupations on the other side of the spectrum. However, in the two remaining regions, there is less evidence suggesting the presence of employment polarization. Employment shares of occupations with the same level of education are rarely increasing/falling simultaneously (e.g., growth of Professionals versus the decline of Managers and Technicians in Central Europe) and rise of Professionals looks as by far the most consistent feature among the regions.


%%%%% Fitzenberger's Revision: Link to Katz and Murphy (esp. Section V)
When compared to the original Skill-Biased Technological Change (SBTC) hypothesis presented by \cite{katz1992changes}, it is clear that, despite regional and country differences, two of the most prominent and consistent phenomena arising from the industrial and occupational analyses in Figures \ref{employ_changes_nace} and \ref{employ_changes_ilo} are the decline in relative employment in manufacturing and agriculture, and the significant rise in the "administrative" category, which is typically characterized by professional, scientific, and technical activities. On the occupational level, as shown in Figure \ref{employ_changes_ilo}, we can already see an increase in relative employment of professionals combined with mostly declining relative employment of middle-skilled categories. Therefore, our findings from the industrial and occupational employment division analysis are in line with those of \cite{katz1992changes}, as they both indicate a shift toward more professional and high-skilled categories in terms of employment.

However, in contrast to \citeauthor{katz1992changes}, our analysis of wages reveals a different picture. The education premium appears to be shrinking, as suggested by the real wage changes, which tend to peak in secondary education categories, as discussed in Section \ref{wage_analysis}. This is in contrast to the monotonically increasing wage changes in education level found by \citeauthor{katz1992changes}. Later in section \ref{skill_bias_cee}, we will show that the skill premium is actually declining. The evidence so far, therefore does not exclude the possibility of relative supply changes as a primary factor of the relative wage dynamics between high and low skill population (as exhaustively examined by \citet{katz1992changes}).  It is worth noting that this negative co-movement inside the individual employment categories (note a strong wage performance of Craft and Trade workers in Central Europe together with their decreasing employment share).

%%%%% Fitzenberger's Revision - end


%Moreover, our data frequently show changes in relative employment going in the opposite direction to changes in relative wages (note a strong wage performance of Craft and Trade workers in Central Europe together with their decreasing employment share). This suggest a possibility of negative covariance between changes in relative wages and employment. As thoroughly discussed by \citep{katz1992changes},  such negative covariance could be a result of supply-side causality for the observed development of relative wages. The movement of wages and employment in opposite directions can also be found in other cases.
%%%%% Fitzenberger's Revision - Table III
%In the case of the Balkan economies, the most prominent employment change seems to be a growth in Services $\&$ Sales, which are considered low-education by \citep{acemoglu2011skills}\footnote{However, this classification is up to debate as the service jobs are here aggregated with sales occupations.}, combined with a mild decrease in relative wages in this category.

\subsection{Covariance Analysis}\label{covariance_analysis}

In the next section, we attempt to formally evaluate the hypothesis of a negative covariance of relative wages and relative employment changes, as suggested above. To do so, we adopt the approach of \cite{katz1992changes} and evaluate the inner products of changes in relative wages and relative supplies within demographic groups\footnote{The groups are defined by sex, education level, and experience and are further described in Section \ref{KM_vars}}, in three roughly equal time periods. Under the assumption that labor supply shifts are the dominant factor in the development of relative wages, we expect the inner products to be negative. Our results in Table \ref{inner_products_ls_w} indicate that while the majority of the periods show negative inner products, we see also positive numbers in every country group for at least one period. This suggests that shifts in relative demand might also play a role in the development of relative wages.

\begin{table}[!htbp]
\centering 
\caption{Inner Products of changes in relative supplies with changes in relative wages}
\label{inner_products_ls_w}
\begin{center}
%\resizebox{\textwidth}{!}{

\begin{tabular}{lrrr}
\toprule
{} &  2007-2011 &  2007-2016 &  2011-2016 \\
\midrule

Cent. Europe &  -0.002735 &   0.006626 &   0.004317 \\
Baltics   &   0.000477 &  -0.005630 &  -0.002432 \\
RO \& BG   &  -0.005488 &  -0.001376 &   0.001183 \\
\midrule
Detrended data\\
\midrule
Cent. Europe &  -0.000785 &  -0.000071 &  -0.000998 \\
Baltics   &   0.000351 &  -0.000011 &   0.000494 \\
RO \& BG   &  -0.002006 &  -0.000080 &  -0.001667 \\
\bottomrule
\end{tabular}
%}
\caption*{\footnotesize  The numbers above represent the inner products of relative supply and wage changes between three time periods (2007-2010, 2011-2015, 2016 and 2019). We calculate the average relative supply and wages within each of the time periods for each of our demographic groups defined by education, sex, and experience. Relative supply is defined as hours worked supplied by a given group divided by total employment measured in efficiency units, while relative wages by group are created using average employment shares as fixed weights the same way as described in section \ref{KM_vars}. The reported numbers are inner products of the changes in these measures of wages and supplies between each pair of the three time intervals. The procedure is repeated for each of the country groups. The columns above indicate the start years of each compared time interval.}
\end{center}
\end{table}

Detrending the wage and supply measures for each demographic group by a linear trend simulating smooth linear demand growth, such as by skill-biased technological change, largely makes the inner products negative. However, the inner products are very small are largely indistinguishable from zero.\footnote{To check this, we conducted a series of regressions using the metrics from Table \ref{inner_products_ls_w} as exogenous and endogenous variables, and found that the estimated parameters were not significantly different from zero.}

%%%%%%%%%% REVIEW STRECO
As noted by \citeauthor{katz1992changes}, non-positivity of the inner product is the key condition for the relative wages to be explained entirely by changes in relative supply . Detrending performed above suggest that this condition is fulfilled ex-linear trend demand element and the demand side seems to be otherwise stable. Relative supply will likely be a key factor in the skill premium development in the region.\footnote{ The results of the analysis above is complemented by Figure \ref{agg_km_vars_detrended} in the Appendix, showing the detrended series of relative labor supply and skill premium, the inputs of Canonical model, discussed in section \ref{skill_bias_cee}. %Although the picture is not completely clear, the two series are frequently going in the opposite direction to changes in relative wages plotted against each other, this demonstrates the strength of the relationship between relative supply and wages 
}

We shall return to this question in the regression analysis below.
%%%%%%%%%% REVIEW STRECO - END

%%%%% Fitzenberger's Revision - Table III

%%%%% Fitzenberger's Revision - Table VI
%Results in previous table does not give us firm persuation that our supply side shifts are the dominant force in the CEE as some of the inner products are not negative. Therefore, we investigate demand side shift, to achieve this, we construct demand shift measure similar to the one in KM. Demand shift here are expressed by  changes in sectoral employment weighted by ... 


%Figure 8
\begin{figure}[!htbp]%
\centering
    \caption{Changes in Log Wages Relative to the Median by ILO Major Category (2011-2019)}
    {\includegraphics[scale=0.5]{wage_changes_ilo.png} }
    \caption*{\footnotesize The Figure shows changes in mean log monthly wages in ILO major categories relative to the median wage. We use wages of full-time workers and ISCO-08 major groups for the classification of the occupations. The categories displayed correspond to ISCO major groups 1-9. The names were abbreviated.}
\label{wage_changes_ilo}
\end{figure}


\begin{figure}[!htbp]%
\centering
    \caption{Employment Share Changes between 2011-2019 by ILO Major Category}
    {\includegraphics[scale=0.5]{employ_changes_ilo.png} }
    \caption*{\footnotesize Changes in different ISCO-08 major groups' shares of total labour supply (measured in hours worked). All workers who worked at least one month in a given year were used.}
\label{employ_changes_ilo}
\end{figure}


%Despite bringing an interesting view of the economies, our results so far do not tell us any decisive conclusion about the SBTC hypothesis. We can only tell that our results, especially those for Central Europe (which are inverse to the US scenario), differ from those of the West. They, however, still could be in line with either SBTC or RBTC theory, given that either high skill labour supply growth is high enough or the countries are recipients of routine jobs from abroad thanks to globalization.

\newpage
\section{Skill-biased Technological Change and the CEE}\label{skill_bias_cee}

The following section discusses the development of labour supply and skill premium, which are the key variables of the Canonical Model — a formalisation of the skill-biased technological change (SBTC) hypothesis developed by Katz and Murphy (1992) and subsequent works. It allows us to investigate the relationship between relative labour supply and skill premium, as outlined in Equation \ref{eqn:STBC_regression}, which is interpreted as the elasticity of substitution between high- and low-skill labor. The model and the construction of the variables are formally introduced in the Appendix (sections \ref{canonical} and \ref{KM_vars}, respectively). 

\subsection{Elasticity of Substitution Estimation}


We start our analysis by looking at the labor supply. As can be seen in Figure \ref{agg_labour_supplies}, there is a rising tendency in relative high-skill supply across CEE (as a high skill category, we define individuals with the highest ISCED education level attained greater than or equal to 5). However, for the university wage premium depicted in Figure \ref{agg_high_low_log_wage_gap} the picture looks more complex.%%%%%%%%%%%%%%%%% REVIEW
\footnote{Figures \ref{regional_high_low_log_wage_gap} and  \ref{regional_labour_supplies} suggest further dynamics on the regional level. Key aspect of the region is large difference between economic performance of the capital cities and main metropolitan ares and the rest of the countries - the figure show that despite similar/same trend, with increasing relative supply downward trending premiums, there are also long run differences between capital regions and the poorest regions. Noteworthy is especially larger wage premium in the capital cities. Note that this analysis is incomplete due to deficiencies of the data which prevent us from drawing much conclusion from the analysis.} 
A decline in the skill premium can be seen in the case of Central Europe. On the other hand, the Baltics show a rather volatile development with a pronounced increase in the premium after 2008 and a downward trend afterwards. Balkan countries experience a decreasing trend before 2015 and a sharp rise in the premium after. These results thus confirm some of the conclusions from the descriptive part, especially concerning the recent development of Romania and Bulgaria.
%%%% REVIEW : 
Overall, the skill premium development does not align with the sharp rise in skill premium seen in the "Canonical" example of the US \citep{acemoglu2002directed}. 

% Figure 10
\begin{figure}[!htbp]%
    \centering
    \caption{Changes in Composition Adjusted High/Low-skill Log Wage Premium}
    {\includegraphics[scale=0.6]{agg_high_low_log_wage_gap.png} }
    \label{agg_high_low_log_wage_gap}
    \caption*{\footnotesize The Figure displays the logarithm of the skill premium described in section \ref{KM_vars}}
\end{figure}

The original Skill-biased Technological change hypothesis assumes a demand-driven change in the labour market (which in turn results from exogenously given technological change). Corresponding to such change should then be an increase in both skill premium and relative supply of skills, as was empirically documented in the case of the US \citep{acemoglu2011skills}. The two key variables of this model are therefore positively correlated. 

% NOTE with reference to Fig.3: it may not be obvious at first sight but the inner product quite match fluctuations in Fig.10 and Fig.11 (If you imagine taking mean in each periods and check correlation of the changes between periods). Also the time series are not fully equivalent (e.g. we ignore the Secondary/Tertiary distinction in Fig.3), so they may differ sometimes.
The data for CEEC presented in Figures \ref{agg_high_low_log_wage_gap} and \ref{agg_labour_supplies} showing university wage premium and relative supply by year. The Figures complement analysis in Table \ref{inner_products_ls_w} and suggest a prevailing negative correlation over the 2007-2019 period between the skill premium and the relative skill supply. However, closer examination of relationship between the two series suggest significant fluctuations leading to temporary reversals in the relationship, consistently with the developments presented in Table \ref{inner_products_ls_w}.  % see Katz and Murphy - detrended / KM 92 - inner products in the ntbs


% Figure 11
\begin{figure}[!htbp]%
    \centering
    \caption{Changes in Relative High/Low Skill Labour Supply}
    {\includegraphics[scale=0.6]{agg_labour_supplies.png} }
    \label{agg_labour_supplies}
    \caption*{\footnotesize The Figure displays the logarithm of the relative labour supply described in section \ref{KM_vars}.}
\end{figure}

A negative covariance between wage premia and relative labour supply would contradict one of the fundamental assumptions of the SBTC model. According to this model, demand-driven change would imply a simultaneous increase/decrease in both variables, as it would cause a movement along the supply curve. However, the situation in CEE appears to be consistent with a movement along the labour demand curve. This suggests that supply changes are setting the market trends or at least playing a more prominent role than the standard SBTC hypothesis would imply. This finding is in line with the overheating  CEE labor market, where the low unemployment environment implies high labour demand for both high- and low- skill categories, making it possible for changes in relative supplies to play a more significant role,  without endogenous demand reaction to the supply shift as suggested by \citep{acemoglu2002directed}. %This was documented above in the analysis of Figures \ref{wage_changes_ilo} and \ref{employ_changes_ilo}.


% Figure ... in the appendix shows that detrending resulting, despite evidence of negative correlation from Table 3, Fig. ... shows the results is not extremely strong with relative wages being rather flat - the negative values in Table 3 are indeed often vary small and hardly distiguishible from zero.
% To remove the demand factors proxied by a linear trend and demonstrate the strength of the relationship between relative supply and wages, 

%%%%%%%%% REVIEW:
Covariance analysis above provided some support for the dominant role of supply side factors, role of the linear trend growth was not excluded. For further investigation of the two variables, we now look at the panel regression for 2005-2019.\footnote{Unlike some of the follow-up works on SBTC \citep{card2001can}, we do not distinguish age categories (young and old workers). We can observe different behavior of the education premium in CEE compared to the original US example, which was motivated by the contrast of rapidly increasing education premium for younger workers and stagnation premium for older workers. In Figure \ref{wage_premiums_by_age}, we can see mostly parallel movements in both young and old education skill premia. The only significant difference is in the Baltics, with declining premia for older workers and stagnating premia for younger ones. Investigation of an advanced framework can be a topic for further research. To further illustrate the dynamics of inequality in the region, Figures \ref{agg_high_low_log_wage_gap_sex} and \ref{agg_high_low_log_wage_gap_exp} also show skill premiums based on gender and experience level. These figures seem to suggest that there is a convergence in skill premiums between the gender groups, this is to a somewhat lesser degree true for experience levels, where especially in visegrad countries more experienced workers keep higher skill premiums than new entrants.}   %%%%% Fitzenberger's revision - Age,  maybe mention young unemployment compared to southern periphery

%%%% Regression results
Following the key works in the field, such as \citet{katz1992changes} or \citet{acemoglu2012does}, we perform regression according to Equation \ref{eqn:STBC_regression} separately for each country - the results confirm the conclusions of the detrended series analysis above - the estimates are insignificant with the exception of Romania and Bulgaria that have significantly negative coefficients with magnitude suggesting higher elasticity than in the US case (-0.27 and -0.34, respectively).\footnote{Overview of the results can be found in Table \ref{regression_individual_countries}. We have also done this exercise for the regional groups - elasticities are again insignificant except for the Balkans.}

However, since the regressions above work with short time series, we decided to utilize panel regression estimation. In Table \ref{panel_regression_comparison}, we see the regression results of several specifications coming from the basic regression design proposed by \citet{katz1992changes} in the form of Equation \ref{eqn:STBC_regression}. This regression equation has been used extensively in studies predominantly concentrated on developed economies in the last decades (Havranek et al., 2021). We also add other explanatory variables inspired by research on determinants of labour market inequality. As suggested by \citet{farber2021unions}, we added union density for each year, average minimum wage and unemployment rate in each of the countries as measures of labour market conditions. Union density was obtained from the ICTWSS database, while minimum wages and unemployment data were sourced from Eurostat. %Note the impact of the minimum wage on labour market inequality found in Germany \citep{bossler2020wage}. 
%%%%%%%%%%%%%%%%%%%%%%%%%%%%%%%%%%%%%%%%%%%%%%%%%%%%%%%
%REVIEW STRECO: 

Table \ref{panel_regression_comparison} shows that union density, despite 
being insignificant, shows constantly positive impact on the skill premium. This suggests that membership in unions could actually have been inequality-increasing, which contrasts with recent evidence focused on the US, most notably by \citet{farber2021unions}, who find union density to be inequality-decreasing in a similar context. Although our findings cannot be interpreted as causal, we believe this result could point to the specific position of unions in the region. The level of de-unionization was higher compared to Western Europe and the US, and fragmented collective bargaining rarely extended to the uncovered sector. Association with the former regime translates into low prestige for the organizations and ideological obstacles to unions and collective bargaining. Existing studies find a wage premium for union memberships, yet the membership composition of the organizations remains an open question, with suggestions that the membership is formed by high-earners, possibly even accepting wage restraints in exchange for job protection \citep{magda2017trade}, potentially eliminating the equalizing effect found elsewhere. \citet{farber2021unions} connect union activity closely with the push for a higher minimum wage, so the two variables are likely strongly linked. Our results indeed show that the coefficient for the minimum wage is indistinguishable form zero.

%%%%%%%%%%%%%%%%%%%%%REVIEW STRECO:
We have also included variables describing the region's position in the Global value chains (GVC), namely participation and position indices as discussed in \citet{coveri2024global}. The former measures overall involvement of a country in the GVCs, while the latter measures the level of 'upstreamness' (i.e., proximity from primary production inputs) of a given country in the Global value chains. \footnote{Formally, the position index calculates relative intensity of backward versus forward linkages defined by foreign value added embedded in country's exports and domestic value added in foreign gross export respectively (see \cite{coveri2024global} for further details).} As \cite{coveri2024global} notes, high-income economies' position index is generally negative because they are more involved in downstream industries. We sourced these variables using the OECD TiVa database.

Table \ref{panel_regression_comparison} suggests that higher participation index decreases the skill premium in the CEE countries, documenting the role and position of the region as and outsourcing destination for the Western Europe. The position index has a negative sign, which implies that higher involvement in downstream production implies higher skill premium. This finding shows that CEE resembles other high-income economies on a global scale and is consistent with \citet{coveri2024global}, who associate it with higher levels of outsourcing linked to downstreamness of the economy. We see that the region is characterized by gradually increasing GVC participation while the backward linkages become more dominant. CEE are generally on the receiving end of globalization, yet they are themselves not immune to outsourcing and closer these countries are to the high-income countries in their industry composition the higher is the pressure on outsourcing associated with more downstream industries.


%%%%%%%%%%%%% REVIEW Table 4:
\begin{table}[!htbp]
\centering 
\caption{Determinants of Skill Premium}
\label{panel_regression_comparison}
\begin{center}
\resizebox{\textwidth}{!}{


\begin{tabular}{lcccc}
\toprule
                                     & \textbf{FE} & \textbf{FE} &   \textbf{RE}    &  \textbf{RE}  \\
\midrule
\textbf{Dependent variable}               &           Skill premium               &     Skill premium     &      Skill premium     &      Skill premium      \\
%\textbf{Estimator}                   &           PanelOLS           &     PanelOLS     &   RandomEffects   &   RandomEffects    \\
%\textbf{Cov. Est.}                   &          Clustered           &    Clustered     &     Clustered     &     Clustered      \\
%\textbf{R-squared}                   &            0.3416            &      0.4321      &       0.3150      &       0.4039       \\
%\textbf{R-Squared (Within)}          &            0.3416            &      0.4321      &       0.3366      &       0.4233       \\
%\textbf{R-Squared (Between)}         &            0.1772            &      0.1872      &      -0.3251      &      -0.3289       \\
%\textbf{R-Squared (Overall)}         &            0.1808            &      0.1964      &      -0.0177      &       0.0291       \\
%\textbf{F-statistic}                 &            34.244            &      13.805      &       32.422      &       13.164       \\
%\textbf{P-value (F-stat)}            &            0.0000            &      0.0000      &       0.0000      &       0.0000       \\
\textbf{=========================}   &         ===========          &   ===========    &  ===============  &  ===============   \\
\textbf{Relative supply}             &           -0.1685            &     -0.1713      &      -0.1086      &      -0.0948       \\
\textbf{ }                           &          (-2.2977)           &    (-1.2869)     &     (-1.6910)     &     (-0.8832)      \\
\textbf{Trend}                       &           -0.0029            &     0.0005      &      -0.0057      &      -0.0012       \\
\textbf{ }                           &          (-0.4728)           &     (0.0359)     &     (-1.0212)     &     (-0.1347)      \\
\textbf{Union density}               &                              &      0.0068      &                   &       0.0069       \\
\textbf{ }                           &                              &     (1.1506)     &                   &      (1.5335)      \\
\textbf{Min. wage}                   &                              &      0.0002      &                   &       0.0001       \\
\textbf{ }                           &                              &     (0.6671)     &                   &      (0.6443)      \\
\textbf{Unemp. rate}                 &                              &      0.0079      &                   &       0.0053       \\
\textbf{ }                           &                              &     (1.5691)     &                   &      (1.1800)      \\
\textbf{GVC participation index}     &                              &     -0.0060      &                   &      -0.0072       \\
\textbf{ }                           &                              &    (-1.5628)     &                   &     (-1.8942)      \\
\textbf{GVC Position Index}          &                              &     -0.8779      &                   &      -0.7433       \\
\textbf{ }                           &                              &    (-1.3188)     &                   &     (-1.3377)      \\
\textbf{Constant}                    &                              &                  &       0.4819      &       0.5879       \\
\textbf{ }                           &                              &                  &      (5.6607)     &      (3.8771)      \\
\textbf{===========================} &        =============         &  =============   & ================= & =================  \\
\textbf{Observations}            &             144              &       144        &        144        &        144         \\
\textbf{Country effects}                 &            Yes           &      Yes      &    No               &         No         \\
\textbf{Time effects}                 &            No           &      No      &    No               &         No         \\
\textbf{Cov. Est.}               &          Clustered           &    Clustered     &     Clustered     &     Clustered      \\
\textbf{R$^{2}$}  &            0.3416            &      0.4321      &       0.3150      &       0.4039       \\
\bottomrule
\end{tabular}

}
\caption*{\footnotesize Clustered Standard Errors reported, t-statistics in parentheses}
\end{center}
\end{table}

The H/L parameter in Table \ref{panel_regression_comparison} can be interpreted as the elasticity of substitution between high and low skilled workers and is therefore of primary interest. Both models also contain a time trend parameter, interpreted as an annual change in relative high skill demand caused by technological change.\footnote{We estimate the equation with the university (tertiary) education representing the high-skill category and all other categories considered low-skill. Another common specification - where we compare university and high school (secondary) education is in the Appendix in Table \ref{panel_regression_comparison_high_school}.}
% REVIEW: Both models contain time trend parameter interpreted as ...

The results show that the relative supply coefficient is negative, significant and between -0.2 and -0.1. As this coefficient represents a negative inverse of the elasticity $\gamma$, we get an elasticity of substitution around 5 or 10. %\footnote{We also performed the Hausmann test in order to choose the preferred model variant. With p-value 0.04551 (for the model including Union density variable) we reject the null hypothesis of RE model.} NOTE: 
% moved to the paragraph below
%%%%%%%%%%%%%REVIEW: 
It is worth noting that the estimates are model choice dependent with Fixed effect model giving a range between 5 to 6, while random effect model estimates being higher around 10.\footnote{While we consider FE model more intuitive in this case as we have little interest in the between effects between the investigated countires, we have also included the RE model, as we failed to reject Hausman test for some of the more complex specifications of the model.}
%%%%%%%%%%%%%REVIEW END

These findings suggest that high- and low-skill labor are gross substitutes. This implies that high- and low-skilled workers are relatively interchangeable, and crucially, an increase in the supply of high-skill workers decreases the demand for low-skilled ones \citep{havranek2020elasticity}. The results also indicate that the estimate is higher than the US consensus estimate for the elasticity of substitution, yet not entirely out of proportion of studies working on Western Europe, as shown by \cite{havranek2020elasticity}. The elasticity of substitution estimates should be viewed in light of the substantial educational upgrading in the region, as highlighted in the previous literature, such as \citet{arendt2019technical} and \citet{hardy2018educational}, who also found high demand for routine (even though cognitive) jobs, a result of CEE being an offshoring destination. Our results suggest there might be an excessive supply of high-skilled population that is subsequently being pushed into less high-skilled occupations than its formal level of education would suggest, resulting in downward pressure to the relative wages.

The regression results also show the time trend parameters, interpreted as a pace of technological change and, more importantly, representing the demand shift. Our results show that this parameter is not significantly different from zero in the CEE. We therefore do not confirm existence a trend demand shift in the CEE. A Possible interpretation of this result is a significantly slower pace of technological change in the CEE countries resulting in less pronounced labour demand changes. Note that such interpretation is in line with previously suggested labour supply shifts - technologically driven labour demand shifts are not strong enough to surpass pressure from the supply side caused by the overheated labour market.


%%%%%%%%%%%%%%%%%%%%%%%%%%%% REVIEW STRECO:
Combining the evidence from the covariance analysis in section \ref{covariance_analysis} and Table \ref{panel_regression_comparison} we see that  it is the substantial relative supply shift in form of the education upgrading that dominated the dynamics of the relative wages. In context of the directed directed technological change framework \citep{acemoglu2002directed}, it is the substitution effect between skilled and unskilled that dominates the picture, pushing effectively the skill premium downward. While the situation in Western economies is characterized by a skill-biased demand shift, in reaction to an increase of relative supply, which pushes the demand curve and has an ability to overcome the substitution effect and push the skill premium up \citep{acemoglu2002directed}. The regression above does not find evidence for this type of endogenous demand shift in the CEE. Therefore, if preset in the CEE, the demand shifts happen to be clearly weaker than the substitution effect caused by educational upgrading.

%The model above suggest that despite demand increasing demand in the region, as documented by Table ..., it is relatively inconsequential. 

The lack of evidence for demand shift and simple comparison of relative supply and skill premia in Table \ref{agg_labour_supplies} and \ref{wage_premiums_by_age} suggest that contrary to the US evidence - we do not see skill-bias in the region. We therefore do not confirm strong version of induced bias of technology, elaborated in \cite{acemoglu2002directed}, where the supply shift leads to upwards sloping demand curve. We also did not find evidence for its weaker form represented by trend demand shifts as shown by the regression above, so the evidence points to movement along the stable demand curve.

We might speculate why there is a lack of evidence for an endogenous demand shift in the CEE. As results in Table \ref{panel_regression_comparison} suggest, there is likely a strong international element present in demand determination in the region, likely being inequality decreasing during the observed period. Arguably, it could be primarily the international factors that could determine the demand for both high- and low-skilled workers during the countries' membership in the EU and could be also linked to the lack of endogenous demand response to labor supply shift through specific aspects of the economies such as high levels of foreign ownership, relatively small market size, or export orientation. These characteristics might arguably move CEE to a position similarly to the proverbial "South" in models of directed technical change such as \cite{acemoglu2002directed}, with technology likely being adopted from abroad in the CEE. Details of the international impact on the relative demand could be the topic of further  research.


%%%%%%%%%%%%%%%%%%%%%%%%%%%%%%%%%%%%%%%%%%

%Nevertheless, note that the model has a relatively poor fit compared to the results from seminal works such as \citet{katz1992changes} and \citet{glitz2021skill} despite R$^{2}$ for individual countries being sometimes around 0.9 (see Table \ref{regression_individual_countries}).
%However, the in many model specifications the the coefficients are insignificant or even positive. This results, contrast with expectations given from the theory -that implies non-negativity of the estimates, as well as many empirical studies (Katz and Murphy has ... for the US), yet seem to be in line with tendency of the literature to present upward biased estimates (Havranek et al., 2021).


%Overall, we can identify a marked difference of the observed basic labour market patterns in the CEE compared to US and German evidence (and we can probably say Western experience in general) that to bigger or lesser degree experienced job and wage polarizations. We do not confirm such a phenomenon in the CEE and rather see strengthening of the middle and bottom part of the distribution compared to the highest quantiles.

\section{Conclusion}
We investigate CEE labour markets during almost the entire period of the countries' EU membership using EU-SILC micro survey data to check key labour market hypotheses found in developed economies, namely the Skill-biased technological change (SBTC) and its newer Routine-biased variant (RBTC). The former is represented by the Canonical model by \citet{katz1992changes}, the latter by the job and wage polarization that represent a key finding of the literature, leading to the formulation of the RBTC.

Moreover, we attempt to complement the literature on labour market inequality in the set of countries closely linked to the developed economies for which the inequality is usually measured but at the same time having a very different position in the labour market chains. We find the EU-SILC a good option for emulating the key STBC literature such as Katz and Murphy (1992) in the CEE context and, at the same time, relatively little used dataset in the context of inequality estimation.

First, our findings indicate a decreasing labour market inequality for most workers, which can be illustrated by the relative position of the median to the top 10 \% and the increasing real wages of workers with secondary education. Our data also suggest that volatility and changes in reaction to the business cycle are concentrated in the lower part of the income distribution.

Next, we do not find much evidence for polarization in terms of wages or occupation. Most importantly, the characteristic U-shaped curve that would suggest growing relative employment and wages on the extremes of the income distribution found in the literature centered around the US is absent in the data of the CEE countries. Our results suggest rather flat or monotonic behaviour. The development of relative wages resembles much more an inverted version of the monotonic behaviour found in the US in the 1980s. Therefore, there does not seem to be any particular reason to adopt the routine-based technological hypothesis instead of SBTC in the context of these countries.

Second, we have also found differences between the countries investigated, embodied by a recent rise in inequality in Romania and Bulgaria. The phenomenon is puzzling to some extent and requires further investigation. However, one should remember that these countries are significantly different in their overall macroeconomic performance from the rest of the sample.\footnote{
Also, note that this process seems to be driven by development in a single country (Bulgaria).}

 We have investigated SBTC hypothesis using the supply-demand framework suggested by \citeauthor{katz1992changes}, we also investigated the hypothesis of negative covariance between relative labor and relative wages, which would result from changes in relative supply primarily setting wage dynamics. In CEE, educational upgrading combined with a downward trending education premium favors this hypothesis. 
 We do not confirm existence of skill-bias neither in the form of increasing skill-premium, under the condition of labor supply shift, nor in the form of substantial shifts in labor demand, envisioned by the SBTC. Our results suggest the wage premium development could be explained primarily by a substitution effect linked with increasing relative supply of high-skilled workers. In addition, we estimated the elasticity of substitution using the estimates of the skill premium of tertiary-educated individuals and their relative labor supply. To deal with the limited number of observations available for individual countries, we used a procedure developed by Katz and Murphy (1992) in a panel model framework. Our estimates suggest the elasticity of substitution in the CEE is between 4 and 10, which is higher than in the case of the US.


The equalizing and generally pro-worker and pro-median situation in the labour market seems to confirm our assumption that the CEE and the West are on different sides of the globalization dividing line. However, this finding also brings a certain puzzle, as these countries experience their own populist surge with the populist parties' vote share tripling between 2000 and 2017 and many of these parties participating in or even leading the governments in the region. However, the sources of these political tendencies seem to be somewhat different from those of developed countries. In the case of the CEE, the populist surge appears to be occurring despite the development of the labour market rather than as a consequence of it. In our view, this paradox offers an opportunity for further investigation of the material causes of dissatisfaction in the region.
%and therefore further dimensions of inequality.

\newpage

\section*{Declaration of generative AI and AI-assisted technologies in the writing process}
During the preparation of this work the author(s) used ChatGPT in order to improve readability of the manuscript. After using this tool/service, the author(s) reviewed and edited the content as needed and take(s) full responsibility for the content of the publication.









\newpage

\bibliographystyle{apalike}
\bibliography{references}


%\section{References}
%\begin{enumerate}
%\item Acemoglu, Daron. "What does human capital do? A review of Goldin and Katz's The race between education and technology." Journal of Economic Literature 50.2 (2012): 426-63.

%\item Acemoglu, Daron, and David Autor. "Skills, tasks and technologies: Implications for employment and earnings." Handbook of labor economics. Vol. 4. Elsevier, 2011. 1043-1171.

%\item Arendt, Łukasz, and Wojciech Grabowski. "Technical change and wage premium shifts among task-content groups in Poland." Economic research-Ekonomska istraživanja 32.1 (2019): 3392-3410

%\item Autor, David. "Polanyi's paradox and the shape of employment growth." Vol. 20485. Cambridge, MA: National Bureau of Economic Research, 2014.

%\item Aziz, Imran, and Guido Matias Cortes. "Between-group inequality may decline despite a rising skill premium." Labour Economics 72 (2021): 102063.

%\item Baldwin, Richard, and Javier Lopez‐Gonzalez. "Supply‐chain trade: A portrait of global patterns and several testable hypotheses." The world economy 38.11 (2015): 1682-1721.

%\item Biewen, Martin, Bernd Fitzenberger, and Jakob De Lazzer. "Rising wage inequality in Germany: Increasing heterogeneity and changing selection into full-time work." ZEW-Centre for European Economic Research Discussion Paper 17-048 (2017).

%\item Biewen, Martin, and Matthias Seckler. "Unions, internationalization, tasks, firms, and worker characteristics: A detailed decomposition analysis of rising wage inequality in Germany." The Journal of Economic Inequality 17.4 (2019): 461-498.

%\item Biewen, Martin, and Miriam Sturm. "Why a Labour Market Boom Does Not Necessarily Bring Down Inequality: Putting Together Germany's Inequality Puzzle." No. 14357. Institute of Labor Economics (IZA), 2021.

%\item Card, David, and Thomas Lemieux. 2001. "Can Falling Supply Explain the Rising Return to College for Younger Men? A Cohort-Based Analysis." Quarterly Journal of Economics 116(2)

%\item Dustmann, Christian, et al. "From sick man of Europe to economic superstar: Germany's resurgent economy." Journal of Economic Perspectives 28.1 (2014): 167-88

%\item Goldin, Claudia, and Lawrence F. Katz. "Extending the Race between Education and Technology." AEA Papers and Proceedings. Vol. 110. 2020

%\item Glitz, Albrecht, and Daniel Wissmann. "Skill premiums and the supply of young workers in Germany." Labour Economics 72 (2021): 102034.

%\item Havranek, Tomas, et al. "The elasticity of substitution between skilled and unskilled labor: A meta-analysis." (2020).

%\item Hardy, Wojciech, Roma Keister, and Piotr Lewandowski. "Educational upgrading, structural change and the task composition of jobs in Europe." Economics of Transition 26.2 (2018): 201-231.

%\item Howell, David R., and Arne L. Kalleberg. "Declining job quality in the United States: Explanations and evidence." RSF: The Russell Sage Foundation Journal of the Social Sciences 5.4 (2019): 1-53.

%\item Katz, Lawrence F., and Kevin M. Murphy. "Changes in relative wages, 1963–1987: supply and demand factors." The Quarterly Journal of Economics 107.1 (1992): 35-78.

%\item Magda, Iga, Jan Gromadzki, and Simone Moriconi. "Firms and wage inequality in Central and Eastern Europe." Journal of Comparative Economics 49.2 (2021): 499-552.

%\item Marin, Dalia. “A Nation of Poets and Thinkers – Less So With Eastern Enlargement? Austria and Germany,” Discussion Paper 4358, Centre for Economic Policy Research, London (2004)

%\item Marin, Dalia. "Global Value Chains, Product Quality, and the Rise of Eastern Europe." Explaining Germany’s Exceptional Recovery (2018): 4

%\item Milanovic, Branko. "Elephant who lost its trunk: Continued growth in Asia, but the slowdown in top 1\% growth after the financial crisis." VoxEU. org 6 (2020).

%\item Mishel, Lawrence, Heidi Shierholz, and John Schmitt. 2013. “Don’t Blame the Robots: Assessing the Job Polarization Explanation of Growing Wage Inequality.”

%\item Mysíková, M., and Večerník, J. (2018). Personal Earnings Inequality and Polarization: The Czech Republic in Comparison with Austria and Poland. Eastern European Economics, 56(1), 57–80.

%\item OECD. 2017. “How Technology and Globalization Are Transforming the Labour Market.” In The Employment Outlook 2017. Paris: Organization for Economic Cooperation and Development.

%\item OECD (2019), Negotiating Our Way Up: Collective Bargaining in a Changing World of Work

%\item Orenstein, Mitchell A., and Bojan Bugarič. "Work, family, fatherland: The political economy of populism in central and Eastern Europe." Journal of European Public Policy 29.2 (2022): 176-195.

%\item Rodrik, Dani, and Stefanie Stantcheva. "Economic Inequality and insecurity: Policies for an inclusive economy." Report for the Blanchard-Tirole Commission (2020).

%\item Schank, Thorsten, and Mario Bossler. "Wage inequality in Germany after the minimum wage introduction." VfS Annual Conference 2020 (Virtual Conference): Gender Economics. No. 224543. Verein für Socialpolitik/German Economic Association, 2020.

%\item Stöllinger, Roman. "Structural change and global value chains in the EU." Empirica 43.4 (2016): 801-829.

%\item Temin, Peter. The Vanishing Middle Class, new epilogue: Prejudice and Power in a Dual Economy. MIT press, 2018.

%\item Tyrowicz, Joanna, and Magdalena Smyk. "Wage inequality and structural change." Social Indicators Research 141.2 (2019): 503-538.













%\end{enumerate}
\newpage
\appendix
\section{Appendix}
\numberwithin{equation}{section}
\setcounter{table}{0}
\renewcommand{\thetable}{A\arabic{table}}
\renewcommand{\thefigure}{A\arabic{figure}}
\setcounter{figure}{0}

\subsection{The Canonical Model}\label{canonical}
In the analysis below, we will follow a modelling framework (Canonical Model) developed first in Tinbergen and further elaborated in \citet{katz1992changes}, \citet{goldin2010race}, \citet{card2001can}, \citet{acemoglu2011skills}, or \citet{glitz2021skill}.
The fundamental assumption behind the framework is the "skill bias" of the technological change that causes relative demand for high-skilled labour to rise permanently.
The model departs from a CES production function:
\begin{equation}
\label{eqn:STBC_prod_function}
Y = [\theta(A_{L}L)^{\frac{\gamma - 1}{\gamma}} + (1 - \theta)(A_{H}H)^{\frac{\gamma - 1}{\gamma}}]^\frac{\gamma}{\gamma - 1}
\end{equation}

In this setting, $H$ denotes high-skilled (university) labour supply, $L$ low-skilled (non-university) labour supply, $\gamma$ is the elasticity of substitution between high skill and low skill labour, and $\theta$ determines the relative importance of the two types of labour in the production function. The primary measure of inequality used is a (log) skill premium between these two types of labour. We can get this premium by first deriving wages for both $L$ and $H$ and obtaining their ratio:
\[\frac{w_{H}}{w_{L}} = \frac{(1 - \theta)}{\theta} \left(\frac{H}{L}\right)^{-\frac{1}{\gamma}}\left(\frac{A_{H}}{A_{L}}\right)^{\frac{\gamma - 1}{\gamma}}\]

and then linearizing the equation by taking logs:
\[\log(\frac{w_{H}}{w_{L}}) = c + \frac{\gamma - 1}{\gamma}\log(\frac{A_{H}}{A_{L}}) - \frac{1}{\gamma}\log(\frac{H}{L})\]
In this function, the relative supply of skilled labour $\frac{H}{L}$ decreases the skill premium, whereas the unobserved $\frac{A_{H}}{A_{L}}$ parameter increases it. $\frac{A_{H}}{A_{L}}$ can be interpreted as a relative development of factor augmenting technology for high and low skilled labour and represents the skill-biased technological change. We assume that this variable has a log-linear trend. This assumption contains a key part of the model - permanently ongoing technological change increasing demand for skilled labour. Thus, we obtain the final version of the equation:
\begin{equation}
\label{eqn:STBC_regression}
\log(\frac{w_{H}}{w_{L}}) = c + \frac{\gamma - 1}{\gamma}\sigma_0 + \frac{\gamma - 1}{\gamma}\sigma_{1}t - \frac{1}{\gamma}\log(\frac{H}{L})
\end{equation}
OLS regression in the form of is then estimated by Katz and Murphy (1992), \citep{acemoglu2012does}, Goldin and Katz (2020) and others in order to obtain an estimate of the elasticity of substitution and an annual change in skilled labour demand.

Card and Lemieux (2001) and Glitz and Wissmann (2021) offer an extension of the model by incorporating middle skill category and distinguishing between young and old workers. Their framework results in a system of equations, allowing to obtain elasticities of substitution between different subgroups using a seemingly unrelated regression framework.

Routine Biased Technological change is then a further extension of the model above (in fact, it nests the Canonical model as its specific case). The key idea of this framework is an economic activity primarily consisting of tasks that can be divided between routine and non-routine and further between cognitive and non-cognitive. The technological change is assumed to substitute routine tasks and strengthen the position of non-routine ones. The framework can thus explain polarization patterns visible in the US labour market in the 1990s \citep{autor2014polanyi}. This framework was formally elaborated by Acemoglu and Autor (2011).

\subsection{Variables Construction} \label{KM_vars}
We follow the proceedings of \citet{katz1992changes} and \citet{glitz2021skill} while calculating the variables used in equation \ref{eqn:STBC_regression}. We first divide our data into groups defined by sex, experience level (6 categories based on time after finishing the highest level of education) and the highest attained ISCED education category.
We compute an estimate of total hours worked in each group and each year as weeks worked times usual weekly hours and personal sample weight from the survey. Similarly, we compute average weekly wages for full-time workers for each of the education-experience-gender groups defined above.\footnote{As the survey does not contain information about worked weeks but only worked months, we assume a person worked 4 weeks every month to get our estimate. In the rest of the work, we prefer using variables with monthly frequency (e.g., monthly wages)}

% Obtaining average relative wage per group
We start by calculating the statistics used further in the process. We sum over all individuals in a group to get a total labour supply in a given group and year (so-called count sample). Subsequently, we compute the relative share of each group in total labour supply in a given year and use this measure to calculate fixed weights defined by a vector of average employment shares for each group over all available years in the sample. We use these fixed weights together with the groups' average wages matrix to calculate time series of relative wages by groups.\footnote{In a given year, all groups' average weekly wages are weighted by its respective fixed weights and summed, creating a wage index for each year of the sample. %result of N'W
We then deflate wages for all groups in a given year to create relative wages.} An average of this time series through time can be interpreted as an estimate of the average relative wage of a given group.

% Efficiency units 
The relative labour supply and other descriptive statistics below are obtained using efficiency units. Efficiency units are essentially the labour supplies (hours worked) for each education-experience-gender-year group multiplied by the group's average relative wage estimate defined above. The result of this operation is then labour supply for each group in each year. We construct more aggregated measures of the labour supply by summing over these groups in each skill group (education category, high and low).

% Labour Supply
We calculate the final supply of high ($H$) and low ($L$) skill labour (more precisely of tertiary vs non-tertiary education) as a sum of respective cells for tertiary and lower than tertiary education groups. Similarly, the changes in labour supply in Table \ref{labour_supply_changes_agg} are also sums of the corresponding groups.

% Wage premiums = weighted average of education-age-gender-year group's wages
To get the skill wage premium, we use the average wages for each education-age-gender-year group mentioned above (the so-called wage sample). We calculated the high/low skill group's composition-adjusted wage as a weighted average of the respective groups' wages with weights defined as each group's average share of the respective (high/low) skill group's total labour supply over all observed years (i.e. we give the highest weight to the wages which provided the highest share of labour supply in the given skill group). As before, the groups included in the high skill category are those with tertiary or higher education, the rest of the groups are considered low-skilled. The skill premium is then a ratio of the composition-adjusted wages of the two groups ($\frac{w_H}{w_L}$).


% check this, primary source KM Table II!
In line with Equation \ref{eqn:STBC_regression}, we then use logs of the skill premium and of the relative labour supply ($H/L$) in the regressions below to obtain the estimate of the elasticity of substitution. Efficiency units are also used in descriptive statistics.



\subsection{Figures}

\begin{figure}[!htbp]%
    \centering
    \caption{Detrended Logarithm of Skill Wage Premium and Relative Labour Supply}
    {\includegraphics[scale=0.5]{agg_km_vars_detrended.png} }
    \label{agg_km_vars_detrended}
\end{figure}




\begin{figure}[!htbp]%
    \centering
    \caption{Changes in Log Wages by Percentile Relative to the Median (2007-2019)}
    {\includegraphics[scale=0.5]{agg_wage_changes_percentiles.png} }
    \label{agg_wage_changes_percentiles}
\end{figure}



%\begin{figure}[!htbp]%
%    \centering
%    \caption{Minimum  Wage Against the Lowest Percentiles}
%    {\includegraphics[scale=0.5]{min_wages_cee.png} }
%    \label{low_deciles_vs_min_w_cee}
%    \caption*{Deciles of monthly log wages of full-time workers. The minimun wage statistic is obtained from Eurostat, rest is calculated from the EU-SILC survey data - individual countries}
%\end{figure}


%\begin{figure}[!htbp]%
%    \centering
%    \caption{Average Wage Against the Highest Percentiles}
%    {\includegraphics[scale=0.5]{high_deciles_against_mean.png} }
%    \label{high_deciles_vs_meam_w_cee}
%    \caption*{\footnotesize Deciles of monthly log wages of full-time workers. The average wage statistic is obtained from Eurostat, the rest is calculated from the EU-SILC survey data. }
%\end{figure}



\begin{figure}[!htbp]%
    \centering
    \caption{Development of (Log) Wage Gaps for Full-time Workers in CEE, 2005–2019}
    {\includegraphics[scale = 0.5]{wage_gaps.png} }
    \label{wage_gaps_CEE}
    \caption*{\footnotesize Development of ratios of different deciles of log monthly wage distribution for full-time workers. }
\end{figure}

\begin{figure}[!htbp]%
    \centering
    \caption{Changes in Log Hourly Wages by Percentile Relative to the Median (2011 - 2019) - Individual Countries}
    {\includegraphics[scale=0.5]{wage_changes_percentiles.png} }
    \label{wage_changes_percentiles}
\end{figure}

\begin{figure}[!htbp]%
    \centering
    \caption{Changes in Employment by Occupational Skill Percentile, 2011–2019.}
    {\includegraphics[scale=0.5]{employ_changes_by_percentiles.png}
    %\includegraphics[scale=0.5]{employ_changes_by_percentiles_without_weighting.png} 
    }
    \label{employ_changes_percentiles}
    \caption*{\footnotesize Mean log-wage in 2011 was used for obtaining the occupation skill rank. Slovenia was excluded from this graph due to low number of observations. }
\end{figure}

\begin{figure}[!htbp]%
        \centering
        \caption{Changes in Relative High/Low Skill Labour Supply in CEE}
        {\includegraphics[scale=0.5]{labour_supplies_cee.png}}
        \label{labour_supplies_cee}
        \caption*{\footnotesize The Figure displays the logarithm of the relative labour supply described in section \ref{KM_vars}.}
\end{figure}

\begin{figure}[!htbp]%
    \centering
    \caption{Changes in Composition Adjusted High/Low-skill Log Wage Premium}
    {\includegraphics[scale=0.5]{high_low_log_wage_gap.png}}
    \label{high_low_log_wage_gap}
    \caption*{\footnotesize The Figure displays the logarithm of the skill premium described in section \ref{KM_vars}}
\end{figure}


\begin{figure}[!htbp]%
    \centering
    \caption{Detrended Skill Wage Premium Against Relative Labour Supply}
    {\includegraphics[scale=0.5]{km_vars_detrended.png} }
    \label{km_vars_detrended}
\end{figure}


\begin{figure}[!htbp]%
    \centering
    \caption{Composition Adjusted High/Low Skill Wage Premium -  Younger and Older Workers}
    {\includegraphics[scale=0.5]{wage_premiums_by_age.png} }
    \label{wage_premiums_by_age}
\end{figure}


\FloatBarrier

%% Table 1 - real wage comparisons by country
\begin{table}[!htbp]
\centering 
\caption{Changes in Real Wages by Country - Central Europe}
\label{real_wage_changes_ce}
\resizebox{\textwidth}{!}{


\begin{tabular}{lrrrrrrrrrr}
\toprule
{} &  2010/2007 &  2019/2011 &  2010/2007 &  2019/2011 &  2010/2007 &  2019/2011 &  2010/2007 &  2019/2011 &  2010/2007 &  2019/2011 \\
Country &         CZ &         CZ &         HU &         HU &         PL &         PL &         SK &         SK &         SI &         SI \\
Groups                                &            &            &            &            &            &            &            &            &            &            \\
\midrule
%(upper) secondary education           &  16.595227 &  17.727918 &  -2.792900 &  17.749656 &   2.079740 &  18.991410 &  37.263530 &  18.773899 &   9.003570 &   4.400152 \\
$<$5                                    &  13.292203 &  12.484792 & -13.301797 &   1.992760 &   1.404850 &   2.158435 &  43.029078 &  14.963351 &   2.532791 &  -4.005816 \\
5-15                                  &  15.490846 &  12.065493 &  -9.345043 &  -4.194209 &   0.206450 &  -6.249807 &  44.631900 &  14.671115 &   6.545352 &  -8.444205 \\
15-25                                 &  15.974291 &  15.734025 &  -5.772473 &   5.305644 &  -2.669173 &   7.435828 &  36.098234 &  13.677924 &   8.370083 &  -4.945813 \\
25-35                                 &  12.657331 &  13.878316 &  -6.565035 &  -4.745099 &  -0.645663 &  -2.863376 &  33.044754 &  10.769683 &   9.746011 &  -2.121099 \\
35-45                                 &  14.679464 &   8.743428 & -10.412439 &  -6.204726 &  -4.694525 &   3.389281 &  34.904562 &  17.388578 &  13.090370 &  -3.692145 \\
%5-15                                  &  15.490846 &  12.065493 &  -9.345043 &  -4.194209 &   0.206450 &  -6.249807 &  44.631900 &  14.671115 &   6.545352 &  -8.444205 \\
%$<$5                                    &  13.292203 &  12.484792 & -13.301797 &   1.992760 &   1.404850 &   2.158435 &  43.029078 &  14.963351 &   2.532791 &  -4.005816 \\
$>$45                                   &  16.969656 &  22.386638 &  40.081575 & -57.108594 & -20.501749 & -13.079941 &  86.422321 &   1.240834 &  89.739821 &   2.103319 \\
Female                                &  16.466768 &  12.904817 & -10.231480 &  -2.504522 &   0.592534 &   1.988888 &  39.512075 &  16.257187 &   4.963890 &  -7.061669 \\
Male                                  &  13.726690 &  13.509837 &  -6.801153 &  -1.191443 &  -2.056478 &  -0.829993 &  38.315302 &  12.038894 &   9.684599 &  -3.691119 \\
primary education                     &        NaN &        NaN &        NaN &        NaN &  -6.431957 &  16.742316 &  14.892138 & -19.080876 &        NaN &        NaN \\
lower secondary education             &  12.661607 &  21.167355 &  -4.838441 &  17.668417 &   1.545489 &  36.688322 &  41.083019 &  19.807928 &   9.676708 &   8.913304 \\
(upper) secondary education           &  16.595227 &  17.727918 &  -2.792900 &  17.749656 &   2.079740 &  18.991410 &  37.263530 &  18.773899 &   9.003570 &   4.400152 \\
post-secondary non tertiary education &        NaN &        NaN &  -6.564271 &  11.965689 &   3.763429 &   8.793980 &        NaN &        NaN &        NaN &        NaN \\
%primary education                     &        NaN &        NaN &        NaN &        NaN &  -6.431957 &  16.742316 &  14.892138 & -19.080876 &        NaN &        NaN \\
tertiary education                    &  13.503349 &  10.458081 & -10.876622 & -13.326144 &  -3.078349 & -17.315706 &  39.555843 &  10.173994 &   6.793411 & -11.944414 \\
\bottomrule
\end{tabular}


}
\end{table}


\begin{table}[!htbp]
\centering 
\caption{Changes in Real Wages by Country - Balkans and Baltics}
\label{real_wage_changes_bb}
\resizebox{\textwidth}{!}{
\begin{tabular}{lrrrrrrrrrr}
\toprule
{} &  2010/2007 &  2019/2011 &  2010/2007 &  2019/2011 &  2010/2007 &  2019/2011 &  2010/2007 &  2019/2011 &   2010/2007 &   2019/2011 \\
Country &         BG &         BG &         EE &         EE &         LT &         LT &         LV &         LV &          RO &          RO \\
Groups                                &            &            &            &            &            &            &            &            &             &             \\
\midrule
%(upper) secondary education           &  34.169631 &  48.082543 &  15.294322 &  28.030395 &  -9.759740 &  46.361597 &  20.016842 &  39.274502 &   -9.949148 &   71.059771 \\
$<$5                                    &  47.330489 &  46.999261 &  -2.786063 &  27.692451 & -21.782520 &  43.721896 &  12.934346 &  40.860507 &  -10.227069 &   80.734811 \\
5-15                                  &  39.495721 &  58.990249 &  17.017189 &  23.508816 &  -9.050369 &  44.872473 &  27.472710 &  33.046852 &  -15.907555 &   66.301508 \\
15-25                                 &  38.971626 &  62.226653 &  24.511844 &  29.143427 &   3.417652 &  43.966947 &  19.711511 &  45.794507 &  -11.472192 &   70.141590 \\
25-35                                 &  41.241964 &  47.981114 &  25.221330 &  28.599171 &   0.019808 &  37.385062 &  29.161357 &  37.007500 &  -14.133517 &   67.328592 \\
35-45                                 &  48.925777 &  59.413050 &  23.044023 &  23.929130 &   6.100906 &  25.092660 &  23.943484 &  37.257713 &  -27.240479 &   64.900808 \\
%5-15                                  &  39.495721 &  58.990249 &  17.017189 &  23.508816 &  -9.050369 &  44.872473 &  27.472710 &  33.046852 &  -15.907555 &   66.301508 \\
%$<$5                                    &  47.330489 &  46.999261 &  -2.786063 &  27.692451 & -21.782520 &  43.721896 &  12.934346 &  40.860507 &  -10.227069 &   80.734811 \\
$<$45                                   &  16.741057 &  80.343981 &  45.896240 &  51.709741 & -24.018872 &  23.309385 &  28.600122 &  28.044531 & -170.377774 &  104.299231 \\
Female                                &  42.140465 &  59.076116 &  21.635277 &  30.356567 &   0.618485 &  33.349480 &  26.489896 &  34.992677 &  -14.005371 &   73.227794 \\
Male                                  &  40.929483 &  54.063529 &  16.225103 &  24.603861 &  -9.107269 &  46.648602 &  20.246848 &  41.156555 &  -15.908893 &   66.045212 \\
primary education                     &  18.277370 &  35.193310 &  13.006796 &  52.001774 & -10.855238 &  49.018434 &  75.204162 &  87.603874 &   -8.930614 &   94.121130 \\
lower secondary education             &  25.414493 &  29.361815 &  12.579510 &  15.265605 & -13.902503 &  34.663328 &  14.897043 &  39.491236 &  -18.434239 &   77.152514 \\
(upper) secondary education           &  34.169631 &  48.082543 &  15.294322 &  28.030395 &  -9.759740 &  46.361597 &  20.016842 &  39.274502 &   -9.949148 &   71.059771 \\
post-secondary non tertiary education &  19.765816 &  79.095391 &  18.719176 &  26.865995 &  -4.276447 &  40.013376 &  20.818130 &  38.422394 &   -6.907480 &   73.697912 \\
%primary education                     &  18.277370 &  35.193310 &  13.006796 &  52.001774 & -10.855238 &  49.018434 &  75.204162 &  87.603874 &   -8.930614 &   94.121130 \\
tertiary education                    &  47.536173 &  62.580966 &  20.829847 &  27.525643 &  -2.795276 &  38.959689 &  25.199548 &  37.588198 &  -17.014110 &   67.006217 \\
\bottomrule
\end{tabular}
}
\end{table}





% Table 2 - individual countries a)
\begin{table}[!htbp]
\centering 
\caption{Relative Labour Supply Changes by Country - Central Europe}
\label{labour_supply_changes_ce}
\resizebox{\textwidth}{!}{
\begin{tabular}{lrrrrrrrrrr}
\toprule
{} &  2010/2007 &  2019/2011 &  2010/2007 &   2019/2011 &  2010/2007 &  2019/2011 &  2010/2007 &   2019/2011 &  2010/2007 &   2019/2011 \\
Country &         CZ &         CZ &         HU &          HU &         PL &         PL &         SK &          SK &         SI &          SI \\
Groups                                &            &            &            &             &            &            &            &             &            &             \\
\midrule
%(upper) secondary education           &  -4.072809 &  -6.978868 &  -4.258382 &  -14.639163 &  -5.887259 & -12.735159 &  -7.592798 &    0.478960 &  -1.317696 &  -16.957616 \\
$<$5                                    &   8.226937 &  -8.964951 & -10.410914 &   -7.212603 &   1.465125 & -42.976057 &  15.683775 &  -42.556512 &   1.303569 &  -38.138896 \\
15-25                                 &   1.231744 & -12.722249 &  -4.766265 &   -7.567406 & -11.520447 &  18.913838 &  -7.891195 &    4.435811 & -11.531289 &    1.571734 \\
25-35                                 &   2.367041 &   9.069147 &   7.254298 &    1.364161 &  -5.938776 & -24.437601 &  -2.540541 &  -12.859471 &   3.443666 &  -13.001034 \\
35-45                                 &  -2.820841 &  10.661964 &  40.908923 &   28.621185 &  30.080666 &  27.337031 &  14.292898 &   21.478962 &  12.836356 &   64.505852 \\
5-15                                  &  -6.422063 &  -0.039276 & -14.599780 &  -15.640995 &   5.180733 &   8.675418 &  -1.798339 &   16.635153 &   4.815126 &    3.894283 \\
%$<$5                                    &   8.226937 &  -8.964951 & -10.410914 &   -7.212603 &   1.465125 & -42.976057 &  15.683775 &  -42.556512 &   1.303569 &  -38.138896 \\
$>$45                                   &  49.114285 &  53.040895 &  47.969350 &  151.887733 &   4.788782 &  17.825691 &  33.106098 &   41.217003 &  50.342057 &  145.376930 \\
Female                                &   1.896840 &   5.510090 &   2.314666 &   -0.537905 &   6.180807 &   6.427731 &   1.206245 &   -1.029751 &   2.865350 &    1.253471 \\
Male                                  &  -0.930675 &  -2.870336 &  -1.592832 &    0.402593 &  -3.707420 &  -4.259695 &  -0.772012 &    0.659910 &  -2.048722 &   -0.937857 \\
primary education                     &        inf &        inf &        NaN &         NaN & -20.218391 & -53.699785 & -81.802288 & -105.235923 & -62.171101 &        -inf \\
lower secondary education             &  -7.342903 & -14.269409 &  -0.214999 &   17.948349 &  59.969552 &  66.649061 &   5.393534 &    6.398527 & -16.042602 &  -54.422373 \\
post-secondary non tertiary education &  -1.215507 &       -inf &  -6.539873 &    6.072272 &  -0.771252 & -61.832275 &        inf &   -5.867509 &        NaN &         NaN \\
(upper) secondary education           &  -4.072809 &  -6.978868 &  -4.258382 &  -14.639163 &  -5.887259 & -12.735159 &  -7.592798 &    0.478960 &  -1.317696 &  -16.957616 \\
%pre-primary education                 &        NaN &        NaN &        NaN &         NaN & -81.038886 &       -inf &        NaN &         NaN &        NaN &         NaN \\
%primary education                     &        inf &        inf &        NaN &         NaN & -20.218391 & -53.699785 & -81.802288 & -105.235923 & -62.171101 &        -inf \\
tertiary education                    &  11.830884 &  20.047701 &   7.377572 &   12.992954 &  13.578050 &  25.139745 &  11.664222 &   -1.072560 &   8.347106 &   26.038308 \\
\bottomrule
\end{tabular}

}
\end{table}

% Table 2 - individual countries b)
\begin{table}[!htbp]
\centering 
\caption{Relative Labour Supply Changes by Country - Balkans and Baltics}
\label{labour_supply_changes_bb}
\resizebox{\textwidth}{!}{
\begin{tabular}{lrrrrrrrrrr}
\toprule
{} &   2010/2007 &  2019/2011 &   2010/2007 &  2019/2011 &  2010/2007 &  2019/2011 &   2010/2007 &  2019/2011 &  2010/2007 &  2019/2011 \\
Country &          BG &         BG &          EE &         EE &         LT &         LT &          LV &         LV &         RO &         RO \\
Groups                                &             &            &             &            &            &            &             &            &            &            \\
\midrule
%(upper) secondary education           &    1.270990 &  -7.472449 &   -4.219765 & -24.563782 & -17.017731 & -11.006235 &  -11.814197 & -21.036536 &  -2.075079 &  -0.143069 \\
$<$5                                    &   -0.951242 &  10.550018 &   -1.107954 &  -3.477220 & -11.578134 & -18.084308 &    5.888761 & -27.601205 &  -3.182683 & -48.302524 \\
15-25                                 &   -0.552386 &  -0.697178 &   -3.466915 & -14.004675 & -11.787495 & -28.160306 &   -8.130503 &  -8.438484 &   9.563295 &  -5.306868 \\
25-35                                 &   -3.163274 &  -3.337190 &    0.920876 &  -9.686528 &   5.593757 & -19.990798 &    1.250872 & -13.213702 & -13.230831 &  12.441937 \\
35-45                                 &   21.015362 &   9.278204 &    5.498243 &  21.481766 &  24.406446 &  57.209064 &   -5.444474 &  26.575962 &  31.079075 &  15.631660 \\
5-15                                  &   -7.183673 &  -8.973120 &    2.123712 &   9.307710 &   8.432220 &  24.372235 &    7.235132 &  16.005090 &  -4.018921 &  -2.376742 \\
%$<$5                                    &   -0.951242 &  10.550018 &   -1.107954 &  -3.477220 & -11.578134 & -18.084308 &    5.888761 & -27.601205 &  -3.182683 & -48.302524 \\
$>$45                                   &  117.220952 &  85.395334 &   -9.337257 &  40.949785 &  -8.184769 &  58.456756 &  -33.811906 &  46.149722 & -71.209834 &  49.650510 \\
Female                                &    2.279623 &  -2.792326 &    7.251265 &  -5.924975 &   7.980956 &  -3.975438 &    9.192610 &  -3.627072 &   1.804030 &  -0.504534 \\
Male                                  &   -1.511782 &   1.978517 &   -4.949785 &   4.096353 &  -6.727155 &   3.277389 &   -7.089635 &   2.914129 &  -1.088929 &   0.314719 \\
primary education                     &  -28.277864 &  -6.329743 &   43.808599 & -38.810912 &   4.774837 & -66.305449 & -134.875227 &  30.864208 & -58.023669 & -20.616426 \\
lower secondary education             &    7.138331 & -16.992489 &  -15.246012 &  16.672097 &   9.909781 & -47.284491 &  -25.265803 & -61.291948 & -14.709289 & -30.352261 \\
(upper) secondary education           &    1.270990 &  -7.472449 &   -4.219765 & -24.563782 & -17.017731 & -11.006235 &  -11.814197 & -21.036536 &  -2.075079 &  -0.143069 \\
post-secondary non tertiary education &  -72.709787 & -18.049389 & -104.662174 &  71.492996 & -20.440135 &  -1.336479 &  -52.989267 &  41.567394 &  -0.935263 &   3.934652 \\
%pre-primary education                 &   72.507726 &       -inf &         NaN &        NaN &        NaN &        NaN & -112.070892 &       -inf &        NaN &        NaN \\
%primary education                     &  -28.277864 &  -6.329743 &   43.808599 & -38.810912 &   4.774837 & -66.305449 & -134.875227 &  30.864208 & -58.023669 & -20.616426 \\
tertiary education                    &   -1.393563 &  14.118566 &   15.750078 &   9.996584 &  17.107347 &   7.413791 &   28.973587 &  13.829629 &  19.051972 &   9.350157 \\
\bottomrule
\end{tabular}
}
\end{table}


\begin{table}[!htbp]
\centering 
\caption{Panel Regression Comparison - Individual Countries}
\label{regression_individual_countries}
\begin{center}
%\resizebox{\textwidth}{!}{


\begin{tabular}{lrrrrr}
\toprule
Country &  Relat. Supply &   P-value &    $t$-statistic &  $R^{2}$ \\
\midrule
     CZ &    0.029875 &  0.828226 &  0.221766 &   0.700503 \\
     BG &   -0.348470 &  0.051787 & -2.207408 &   0.747515 \\
     EE &    0.074138 &  0.759536 &  0.313159 &   0.343227 \\
     HU &    0.110438 &  0.432270 &  0.812607 &   0.964073 \\
     LT &   -0.020593 &  0.857309 & -0.183709 &   0.610137 \\
     LV &    0.084884 &  0.432266 &  0.818276 &   0.489621 \\
     PL &    0.989898 &  0.227472 &  1.271998 &   0.735357 \\
     RO &   -0.267396 &  0.012218 & -3.051628 &   0.905163 \\
     SI &   -0.048920 &  0.571804 & -0.581287 &   0.914788 \\
     SK &    0.114771 &  0.206714 &  1.334802 &   0.741803 \\
\bottomrule
\end{tabular}

%}
\end{center}
\end{table}


%%%%%%% REVIEW:
\begin{table}[!htbp]
\centering 
\caption{Panel Regression Comparison - Using Secondary Education as the Low-Skill Category}
\label{panel_regression_comparison_high_school}
\begin{center}
\resizebox{\textwidth}{!}{

\begin{tabular}{lcc}
\toprule
                                 & \textbf{FE} &    \textbf{RE}     \\
\midrule
\textbf{Dep. Variable}           &       Skill Premium       &      Skill Premium       \\
%\textbf{Estimator}               &       PanelOLS       &   RandomEffects    \\
%\textbf{No. Observations}        &         144          &        144         \\
%\textbf{Cov. Est.}               &      Clustered       &     Clustered      \\
%\textbf{R-squared}               &        0.3406        &       0.3154       \\
%\textbf{R-Squared (Within)}      &        0.3406        &       0.3351       \\
%\textbf{R-Squared (Between)}     &        0.0099        &      -0.2642       \\
%\textbf{R-Squared (Overall)}     &        0.0197        &       0.0284       \\
%\textbf{F-statistic}             &        34.094        &       32.475       \\
%\textbf{P-value (F-stat)}        &        0.0000        &       0.0000       \\
\textbf{=====================}   &     ===========      &  ===============   \\
\textbf{Relative Supply}               &       -0.1326        &      -0.0715       \\
\textbf{ }                       &      (-1.9844)       &     (-1.2777)      \\
\textbf{Trend}                   &       -0.0046        &      -0.0073       \\
\textbf{ }                       &      (-0.8475)       &     (-1.4830)      \\
\textbf{Constant}                   &                      &       0.4930       \\
\textbf{ }                       &                      &      (7.3184)      \\
\textbf{=======================} &    =============     & =================  \\
\textbf{Country FE}              &        Yes           &                    \\
\textbf{No. Observations}        &         144          &        144         \\
\textbf{Cov. Est.}               &      Clustered       &     Clustered      \\
\textbf{R-squared}               &        0.3406        &       0.3154       \\
\bottomrule
\end{tabular}

}
\caption*{\footnotesize The high school category was defined as containing ISCED level of the highest attained education 3 and 4.}
\end{center}
\end{table}



% Regression results - Low skill as High School educated
%\begin{table}[!htbp]
%\centering 
%\caption{Panel Regression Comparison - Using Secondary Education as the Low-Skill Category}
%\label{panel_regression_comparison_high_school}
%\begin{center}
%\resizebox{\textwidth}{!}{



%\begin{tabular}{lccc}
%\toprule
%                                 & \textbf{FE} & \textbf{FE} &    \textbf{RE}     \\
%\midrule
%\textbf{Dependent Variable}           &       skill premium       &         skill premium        &      skill premium      \\
%\textbf{Dep. Variable}           &       logW\_HL       &         logW\_HL        &      logW\_HL      \\
%\textbf{Estimator}               &       PanelOLS       &         PanelOLS        &   RandomEffects    \\
%\textbf{No. Observations}        &         144          &           144           &        144         \\
%\textbf{Cov. Est.}               &      Clustered       &        Clustered        &     Clustered      \\
%\textbf{R-squared}               &        0.3279        &          0.0626         &       0.3154       \\
%\textbf{R-Squared (Within)}      &        0.3279        &          0.3237         &       0.3351       \\
%\textbf{R-Squared (Between)}     &        0.2461        &          0.2232         &      -0.2642       \\
%\textbf{R-Squared (Overall)}     &        0.2486        &          0.2263         &       0.0284       \\
%\textbf{F-statistic}             &        64.884        &          7.9480         &       32.475       \\
%\textbf{P-value (F-stat)}        &        0.0000        &          0.0056         &       0.0000       \\
%\textbf{=====================}   &     ===========      &       ===========       &  ===============   \\
%\textbf{Relative Supply}               &       -0.2155        &         -0.1912         &      -0.0715       \\
%\textbf{ }                       &      (-3.3988)       &        (-2.0002)        &     (-1.2777)      \\
%\textbf{Trend}                   &                      &                         &      -0.0073       \\
%\textbf{ }                       &                      &                         &     (-1.4830)      \\
%\textbf{Constant}                   &                      &                         &       0.4930       \\
%\textbf{ }                       &                      &                         &      (7.3184)      \\
%\textbf{=======================} &    =============     &      =============      & =================  \\
%\textbf{No. Observations}        &         144          &           144           &        144         \\
%\textbf{Country FE}                 &        Yes        &          Yes         &                    \\
%\textbf{Time Effects}                 &        No        &          Yes         &                    \\
%\textbf{Cov. Est.}               &      Clustered       &        Clustered        &     Clustered      \\
%\textbf{$R^{2}$}               &        0.3279        &          0.0626         &       0.3154       \\
%\bottomrule
%\end{tabular}


%}
%\caption*{\footnotesize The high school category was defined as containing ISCED level of the highest attained education 3 and 4.}
%\end{center}
%\end{table}



\begin{table}[!htbp]
\centering 
\caption{Basic Income Statistics (logs)}
\resizebox{\textwidth}{!}{

\begin{tabular}{llrrrrrr}
\toprule
     &          &    Count &      Mean &       Std. &       25\% &       50\% &       75\% \\
Year & Country Group &          &           &           &           &           &           \\
\midrule
2005 & Baltics &   7936 &  5.759445 &  0.696044 &  5.342908 &  5.759542 &  6.149943 \\
     & Cent. Europe &  32713 &  6.147041 &  0.845683 &  5.659424 &  6.096721 &  6.645361 \\
2006 & Baltics &   9882 &  5.902042 &  0.706217 &  5.491138 &  5.898741 &  6.307205 \\
     & Cent. Europe &  39671 &  6.271106 &  0.975203 &  5.824943 &  6.288126 &  6.786192 \\
2007 & RO \& BG &   7136 &  5.291471 &  0.682239 &  4.928569 &  5.295346 &  5.645978 \\
     & Baltics &  14030 &  6.034773 &  0.706098 &  5.627518 &  6.051653 &  6.440065 \\
     & Cent. Europe &  42221 &  6.326558 &  0.956770 &  5.906392 &  6.342669 &  6.811998 \\
2008 & RO \& BG &   9087 &  5.519398 &  0.738633 &  5.192519 &  5.543809 &  5.893378 \\
     & Baltics &  14025 &  6.261341 &  0.709582 &  5.860931 &  6.287629 &  6.671028 \\
     & Cent. Europe &  44834 &  6.439026 &  0.948994 &  6.073611 &  6.474315 &  6.899268 \\
2009 & RO \& BG &  10225 &  5.690156 &  0.581562 &  5.341389 &  5.663007 &  6.013813 \\
     & Baltics &  13698 &  6.388316 &  0.762803 &  5.966948 &  6.397537 &  6.819962 \\
     & Cent. Europe &  43023 &  6.636470 &  0.694342 &  6.222516 &  6.616336 &  7.038087 \\
2010 & RO \& BG &  10427 &  5.648773 &  0.539679 &  5.320666 &  5.620769 &  5.955757 \\
     & Baltic &  12909 &  6.253656 &  0.929514 &  5.859363 &  6.298158 &  6.735506 \\
     & Cent. Europe &  41529 &  6.590091 &  0.720055 &  6.155336 &  6.570671 &  6.997807 \\
2011 & RO \& BG &  10444 &  5.644617 &  0.545742 &  5.327945 &  5.629986 &  5.933982 \\
     & Baltic &  13250 &  6.225751 &  0.927470 &  5.834358 &  6.289585 &  6.708138 \\
     & Cent. Europe &  41596 &  6.624427 &  0.766990 &  6.214608 &  6.619581 &  7.036221 \\
2012 & RO \& BG &   9238 &  5.661872 &  0.603479 &  5.375911 &  5.649155 &  5.966160 \\
     & Baltics &  13696 &  6.339866 &  0.831622 &  5.930934 &  6.366572 &  6.778997 \\
     & Cent. Europe &  41243 &  6.676594 &  0.679284 &  6.249765 &  6.661117 &  7.070795 \\
2013 & RO \& BG &   8827 &  5.707954 &  0.546572 &  5.423569 &  5.671162 &  6.000966 \\
     & Baltics &  13626 &  6.425308 &  0.734069 &  6.005043 &  6.427104 &  6.825060 \\
     & Cent. Europe &  39109 &  6.666303 &  0.705582 &  6.239572 &  6.634091 &  7.058328 \\
2014 & RO \& BG &   8689 &  5.714072 &  0.677240 &  5.453236 &  5.709093 &  6.047055 \\
     & Baltics &  13586 &  6.372905 &  1.101866 &  6.018816 &  6.472928 &  6.891700 \\
     & Cent. Europe &  38241 &  6.684074 &  0.689644 &  6.242752 &  6.659294 &  7.082704 \\
2015 & RO \& BG &   8865 &  5.801071 &  0.606776 &  5.516046 &  5.757518 &  6.085880 \\
     & Baltics &  13357 &  6.525265 &  0.803623 &  6.110615 &  6.545443 &  6.968127 \\
     & Cent. Europe &  35515 &  6.680442 &  0.757410 &  6.291835 &  6.675150 &  7.087295 \\
2016 & RO \& BG &  10769 &  5.770345 &  0.795354 &  5.511512 &  5.795422 &  6.141636 \\
     & Baltics &  13751 &  6.590793 &  0.803911 &  6.185563 &  6.605682 &  7.029500 \\
     & Cent. Europe &  35814 &  6.704685 &  0.766409 &  6.309918 &  6.697632 &  7.114769 \\
2017 & RO \& BG &  11022 &  5.944106 &  0.650830 &  5.647826 &  5.956890 &  6.262375 \\
     & Baltics &  13778 &  6.619615 &  0.819248 &  6.230487 &  6.658713 &  7.060399 \\
     & Cent. Europe &  36637 &  6.724185 &  0.851435 &  6.352773 &  6.737042 &  7.151355 \\
2018 & RO \& BG &  11075 &  6.042281 &  0.682581 &  5.753798 &  6.072441 &  6.399620 \\
     & Baltics &  13608 &  6.733360 &  0.808707 &  6.355025 &  6.756988 &  7.159984 \\
     & Cent. Europe &  36745 &  6.820573 &  0.690773 &  6.440738 &  6.829794 &  7.202289 \\
2019 & RO \& BG &  11162 &  6.309858 &  0.715430 &  6.010924 &  6.349876 &  6.703679 \\
     & Baltics &  13382 &  6.799291 &  0.817138 &  6.408943 &  6.831326 &  7.233602 \\
     & Cent. Europe &  38069 &  6.890765 &  0.727646 &  6.533108 &  6.903078 &  7.281287 \\
\bottomrule
\end{tabular}
}
\end{table}

\begin{figure}[!htbp]%
        \centering
        \caption{Relative High/Low Skill Labour Supply - Capital cities and the rest }
        {\includegraphics[scale=0.5]{regional_labour_supplies.png}}
        \label{regional_labour_supplies}
        \caption*{\footnotesize The Figure displays the logarithm of the relative labour supply for a group of clearly identifiable NUTS regions containing a capital city versus regions below 75\% of EU GDP in 2004. The data allows us to distinguish only a limited number of Capital regions from the rest of the countries, effectively the sample work with NUTS1 region for Budapest, Warsaw, Sofia and, Prague is represented by its NUTS2 region. }
\end{figure}


\begin{figure}[!htbp]%
        \centering
        \caption{Composition Adjusted High/Low Skill Wage Premium - Capital cities and the rest}
        {\includegraphics[scale=0.5]{regional_high_low_log_wage_gap.png}}
        \label{regional_high_low_log_wage_gap}
        \caption*{\footnotesize  The Figure displays the logarithm of the skill premium for a group of clearly idenfiable NUTS regions containing a capital city versus regions below 75\% of EU GDP in 2004. The data allows us to distinguish only a limited number of Capital regions from the rest of the countries, effectively the sample work with NUTS1 region for Budapest, Warsaw, Sofia and, Prague is represented by its NUTS2 region.}
\end{figure}


\begin{figure}[!htbp]%
        \centering
        \caption{Composition Adjusted High/Low Skill Wage Premium - Male vs. Female }
        {\includegraphics[scale=0.5]{agg_high_low_log_wage_gap_sex.png}}
        \label{agg_high_low_log_wage_gap_sex}
        \caption*{\footnotesize }
\end{figure}

\begin{figure}[!htbp]%
        \centering
        \caption{Composition Adjusted High/Low Skill Wage Premium - Experience level }
        {\includegraphics[scale=0.5]{agg_high_low_log_wage_gap_exp.png}}
        \label{agg_high_low_log_wage_gap_exp}
        \caption*{\footnotesize }
\end{figure}

\end{document}
