\documentclass{article}
\usepackage[utf8]{inputenc}
\usepackage{graphicx}
\usepackage{booktabs}
\usepackage{caption}
\captionsetup[figure]{font=small,labelfont=small}
%\usepackage[margin=1.5in]{geometry}

\title{Skill-bias and Wage Inequality in CEE: empirical investigation}
\author{Jan Pintera}
\date{}

\begin{document}

\maketitle
\section{Introduction}
The echoes of labour markets turmoil in the developed economies have been heard quite often in recent decades. Fears of unemployment, job-quality deterioration or, more specifically, the "hollowing-out" of the entire middle class appear in the latest government reports (e.g. Rodrik and Stantcheva (2020) in the case of France) and have been long discussed in the academic literature concentrated on effects of technological change in the United States (Acemoglu, 2012; Autor, 2014).

This paper attempts to compare these findings with experience of the so-called new EU member state countries of central and eastern Europe (CEE). These countries play rather different role in the global market value chains than a model developed country (Baldwin and Lopez-Gonzales, 2015). The CEE (including Poland) are typical examples of the "Factory Economies" strongly linked to their headquarter economy - Germany. Note that the labour market theories discussed below are created and tested in context of the headquarter economies (US and Germany in our case) (Baldwin and Lopez-Gonzales, 2015). 
For their favourable unit labour cost and skilled workforce the CEE countries seem to be ideal recipients of offshoring from the high-wage economies, we can therefore assume rather opposite development than that in the old EU states. Indeed, empirical work in case of the income distribution shows signs of declining wage inequality in last decade or so (Magda et al., 2021). This work brings descriptive analysis of this issue using the EU-SILC survey data from 2005 and 2019 and attempts to put this data into context of research on the technological change impact and labour market polarization seen in the US and Western Europe.
\\
The literature generated several testable hypotheses about the impact of technological change on the functioning of the labour market, that despite their early origin (Katz and Murphy, 1992) and empirical critiques by Mishel, Schierholz and Schmitt (2013) and others seem to endure to these days (Aziz and Cortes, 2021; Goldin and Katz, 2020). Among those is especially the STBC (Skill-biased technological change) hypothesis - assuming negative relationship between relative high and low skill labour supply and wages. Further refinement of this hypothesis postulate a job and wage polarization - seen in the US and elsewhere, a phenomena often connected with technological change and also an increase in relative supply of manual non-routine jobs. Using similar methods as statistics as seminal works on the US and German labour market (Katz and Murphy, 1992; Mishel, Schierholz and Schmitt, 2013), this work found that many of the phenomena mentioned above were not confirmed in case of the Central Europe. 
\\
To anticipate our results, the general conclusion reached by the study is a good performance of the lower parts of the wage distribution. Perhaps most notably, we see relative decline of the highest earners in both wages relative to median and employment, contrary to the characteristic U-shaped behaviour documented by Acemoglu (2012) and interpreted as the job and wage polarization. This seems to be in line with the view of the new member states as a semi-periphery of the West, open economies with low unit labour costs which in environment of globalization leads to an inflow on relatively routine-intensive jobs, which drives demand for the low and middle type of jobs and has generally equalizing effect on the labour market. 
We further investigated the question of causality of labour market changes - we utilized a simple supply-demand framework used by Katz and Murphy (1992) and later studies and investigated significance of labour supply and wages relationship in regression inspired by the framework. Due to lack of observation that the micro-surveys can give us, we utilized a panel regression combining multiple CEE countries. One of the motivations of this approach was the significant skill upgrading seen in the region a phenomena visible in the US several decades earlier where the framework is found to perform well Hardy et al., 2018).
\\
The contribution of this work is an investigation of the main hypotheses about the development of labour market inequality in the new EU member states with particular emphasis given to the skill-biased framework and empirical hypotheses stemming directly from it. We also want to overview CEE labour market trends compared to the regularities observed in Germany and the United States. We consider this topic and the use of micro-data in this context as relatively under-researched, even more so as there is a good reason to believe that the observed development in CEE will be different, if not inverse to that found in the West. Compared with previous works on the topic, this work brings a direct test of the skill-biased framework for all the countries of interest instead of testing a particular subsection of the theory (routine-biased technological change as in case of Arendt and Grabowski (2019) or Hardy et al. (2018)) or concentrating on a single country. It will also use a different source of data (EU-SILC) that provides annual data for all countries of interest and due to their annual form and coverage of labour supply are in our view the best fit to the surveys used in the US studies such as Current Population Survey used in Katz and Murphy (1992). \\
The paper is organized as follows. First part introduces the skill bias framework and reviews labour market development in the CEE and US and the Germany. Second part outlines details of the Skill-Bias Technological change and discusses variable construction, third section presents and discusses the results and forth section concludes.

\section{Labour Market developments}
\subsection{Wage inequality hypotheses - the US case}

One of the key hypotheses emerging from the technology debate is so-called skill-biased technological change, which explains changes in relative wages using a simple supply-demand framework (the "Canonical model"; Acemoglu, 2012) focusing on different levels of skills/education. The Canonical model in its original form is a simple and straightforward model that uses relative high/low skill labour supply and time dependent "skill-biased" technological progress as a determinants of relative wages.\\
The skill-bias hypothesis was tested by Katz and Murphy (1992) in the form of regression of US skill premium (college/high-school relative wage) on a time trend and a relative supply of high/low-skilled labour. Despite its simplicity, the model was shown to perform rather well on the US data before 1990s (Katz and Murphy, 1992) and existence of the link between technology and education as a determining factor in wage setting in the long term seems evident (Piketty, 2015). However, Acemoglu and Autor (2011) shows that the Katz and Murphy's model overpredicts the skill premium in the 1990s and the 2000s. It also fails to account for several other stylized facts about the recent developments of wages in the US, most importantly, the job and wage polarization, i.e., strengthening of the tails of income/employment distribution. This process seems strongly connected to the automatization of middle-skill jobs. Based on these findings, Acemoglu (2012) proposes a comprehensive task-based framework (also routine-biased technological change, RBTC) focused on level of routine content of the tasks involved rather than on workers' skills. Formal representation of the task-based framework can be found in Acemoglu and Autor (2011). Most notably, this framework is capable of explaining the wage polarization phenomenon (increasing relative wages of bottom and top of the distribution with respect to the middle part) observed in the 1990s in the US.\\

Mishel, Schierholz and Schmitt (2013) postulate three testable hypotheses derived from the skill-biased technological change literature and its extensions. First, the labour supply and demand interactions determine wage formation. More concretely, technological change causes changes in labour demand which in turn affect wages. This causality can be considered a general feature of the framework common to both the original skill-biased technological change and its subsequent variant, the routinization-biased technological change. Second, from the empirical point of view, the skill-biased technological change leads to job polarization - a phenomenon highlighted by Acemoglu (2012), Howell and Kallenberg (2019) and others when discussing the developments of the Western labour markets in the last decades of the 20th century. Note, that in this point, both variants of the technological framework differ, with the original the SBTC hypothesis predicting a monotonic employment and wage development across the occupational distribution, a phenomena observed in the 1980s.\\
Third, the RBTC hypothesis implies a rise in both employment and wages in specific type of services - namely the low-wage service jobs characterized by manual non-routine content.

Note that the skill-bias framework always faced critiques such as Mishel, Schierholz and Schmitt (2013) who rather than explicitly denying the the underlying "job polarization" trend, map it to much earlier time and thus deny its causal link with inequality rise after 1970s (and thus questions link between job polarization and wage polarization). In their interpretation technological changes has significant impact on occupation composition, not on wage inequality. They also point to a general wage deficit - inability of wages to keep up with productivity growths and rising profits after 2000s.

Regarding the the key part of the framework - job-polarization, we should note that considering evidence outside the US and in longer time period, we can hardly consider it a stylized fact. As shown by Mishel and Shierholz and Schmitt (2013) job-polarization seem to be a phenomena linked firmly to the 1990s and already the early 2000s brought slowdown of education premium and high-occupation rise. We can therefore also formulate the difference between RBTC and SBTC as difference between 1980s and 1990s US labour market.

This work aims to provide an overview of the Central and Eastern European Labour market after the EU accession from the perspective of the above mentioned labour market theories about the impact of technological change in the developed economies. The work will mainly concentrate on labour market developments in the US and Germany as the most relevant labour markets for CEE and compare their development with labour market trends in the CEEs countries. The main theoretical frameworks investigated will be the skill-biased technological change both in its canonical version and its occupation-content oriented variant - routine based technological change. The work aims to provide close comparison of CEE and German trends, especially due to proximity and co-dependence of Germany and CEE in the global value-chains.
We should note that the CEE region has a specific characteristics when compared to the Western Europe, as its comparative advantage in labour costs place it to different side of the globalization demarcation line than countries of the Western Europe. At the same time, giving closer look at the relationship between the West and the CEE region give us even more fascinating picture with the CEE serving as a pool of relatively cheap and qualified labour to Germany, strongly influencing Germany's internal labour market in return (Marin 2004; 2018). Our comparison will mainly focus on Germany, as it is the most deeply connected neighbouring country (see Baldwin, 2015). Following sections outline labour market developments in Germany and CEE.


\subsection{CEE}
As noted by Tyrowicz and Smyk (2019) in their assessment of the impact of structural change on inequality, the micro-data on income and inequality have been used for assessment of the skill-bias hypothesis in a small number of developed countries only. For the CEE countries, these micro data were not available until relatively recently.
So far, there were only few attempts to investigate the role of skills in income inequality among the CEE countries. Arendt and Grabowski (2019) studied the wage premium in Poland, and Hardy et al. (2018) provide an analysis of task-content development in EU following Acemoglu and Autor's (2011) approach and provide analysis of labour supply development in EU-24 with emphasis on the CEE countries. Both papers exploit the task-content division of the labour force (classification of jobs according to a required level of cooperation and creativity), this work attempts for deeper examination of major labour market hypotheses in the CEE countries, it will use with the EU-SILC Eurostat database that provides micro-level information about income and living conditions in the EU countries on an annual basis. We believe this database is the closest to the Current Population Surveys data used by the key works in the field (Katz and Murphy (1992), Acemoglu (2012).
The results of both Hardy et al. (2018) and Arendt and Grabowski (2019)  point towards certain deviations of this region from the rest of Europe in terms of the task distribution. Namely, according to Hardy et al. (2018) we see increase in routine cognitive tasks in CEE countries, which is contrary to both the old-EU countries and routine-replacing technological change hypothesis, similarly Arendt and Grabowski (2019) find relative wages in routine manual jobs in Poland too high for the RBTC hypothesis to hold. Both studies then note significant educational upgrading in the region, especially the rapidly increasing tertiary education attainment (Hardy et al., 2018). We should note that at least in this aspect the CEE seem to differ significantly from the U.S. labour market, where as noted by Acemoglu (2012) high-school attainment is actually stagnant since 1960s and post-secondary attainment decelerated already in 1970s. Specificity of the CEE income inequality with respect to the West is also confirmed other works such as Magda et al. (2021) - who notes decrease of wage inequality in the CEE in 2002-2014 period. Before this period the CEE countries experienced significant inequality rise due to their economic transformation but the inequality leveled since then (Tyrowicz and Smyk, 2019) with evidence of wage inequality staying lower than in the developed countries Mysíková and Večerník (2018). This conclusion is also confirmed by recent study of Magda et el. (2021) who find generally decreasing levels of wage inequality in the CEE using EU-SES datasets for 2002-2014, with the only country with slight increase of wage inequality being the Czech Republic. The authors notes that this finding stands in contrast to development found in the Western countries. From our point of view, this speak in favour of analyzing the skill-bias hypothesis in this region.



\subsection{Germany}

%Germany on the other hand - underwent a profound labour market transformation in recent decades (Marin et al., 2018). Dustmann et al. (2014) attribute its labour market resilience to a unique set of labour market institutions - most notably its decentralized and de-politicized wage bargaining process that allowed for labour market flexibility in face of adverse external macroeconomic conditions. Concretely, Dustmann et al. (2014) concentrate their analysis on wage restrains of German workers which can be reflected in behaviour of German unit labour costs with respect to the other Western countries. Dustman et. al (2014) also notes note decreasing real wages at the lower end of the wage distribution after mid-1990 but not before. He attributes this to the German interaction with the CEE countries - that served as a pool of comparatively cheap skilled labour that helped German businesses and in turn allowed to put pressure on German workers. Marin (2004) also interestingly notes that German were the offshoring skilled rather unskilled work to the new countries, in an attempt to solve its own low human capital endowment shortages.

In general, Germany has been, similarly to the US and other Western economies, experiencing rising income inequality at least since the 1980s (Biewen, Fitzenberger and Lazzer, 2017). However recent development point to certain specificity of the country's development. 

Biewen and Sturm (2021) find that the inequality has been stagnating since 2005 and attribute this phenomenon to recent labour market boom. Schank and Bossler (2020) concentrating on the lower tail of the distribution find a rising wage inequality in 2000s and declining trend after 2010s, furthermore they observe a sharp drop in inequality after 2014, that they attribute to minimum wage introduction.

Giving a closer look at the development of inequality in different parts of the income distribution Biewen and Seckler (2019) do not confirm wage polarization found in the US. They find much more monotonic development with the highest percentiles gaining relatively the most. This is confirmed by Biewen, Fitzenberger and Lazzer (2017) who document rise in inequality limited to the top part of the distribution (development in line with with the SBTC hypothesis) until mid 1990s with the labour market institutions preventing rise of the inequality at cost of higher unemployment  - later it rose across the entire distribution.

%Biewen and Sturm (2021) comments on German development after recent labour market boom (since 2005) and find it having an equalizing effect - it led to income gains across the distribution and the lower part of the distribution experienced bigger gains than the upper parts, despite institutional and external factor dampening this effect.


As far as the causes of the inequality rise is concerned the literature has so far not reached a conclusion. Yet among the most often mentioned reasons are labour market institutions (decline in collective bargaining) and composition changes (such as educational upgrading, labor market history, industry structure, and occupation) (Biewen, Fitzenberger and Lazzer, 2017). Emphasized is also a significant wage restraint by German workers reflected in behaviour of German unit labour costs with respect to the other Western countries (Dustmann et al., 2014) and the role played in the CEE countries that served as a pool of comparatively cheap skilled labour that helped German businesses and in turn allowed to put pressure on German workers (Marin 2004, 2018).
In terms of the SBTC, Biewen, Fitzenberger and Lazzer (2017) find strong influence of composition changes in explaining rising inequality - education (especially in the upper part of the distribution) and changes in recent labor market histories (lower part) and conclude that this finding is in line with SBTC hypothesis.
Last but not least, Glitz and Wissman (2021) show that after breaking German population to three education levels and two age groups and using the procedure developed by Katz and Murphy (1992) and Card and Lemieux (2001), they find that the labour supplies are to large extent able to explain the changes in skill premium in Germany. They find especially pronounced rise in skill premium of medium skilled to low skilled and link it to decline in the share of population with vocational training. Their findings are therefore very much in line with the original SBTC framework and can be seen as reaching a similar results as the seminal work of Goldin and Katz (2009). %TODO: Glitz and Wissmann have some conclusions about Polarization, add those here


\section{Methodology}
\subsection{The Canonical Model}
In the analysis below we will follow a modelling framework ("Canonical Model") developed first in Tinbergen and further elaborated in Katz and Murphy (1992) Goldin and Katz (2009), Card and Lemieux (2001), Acemoglu and Autor (2011) or Glitz and Wissmann (2021).
The fundamental assumption behind the framework is the "skill-bias" of the technological change that causes relative demand for high-skilled labour to permanently rise.
The model departs from an CES production function:
\begin{equation}
\label{eqn:STBC_prod_function}
Y = [\theta(A_{L}L)^{\frac{\gamma - 1}{\gamma}} + (1 - \theta)(A_{H}H)^{\frac{\gamma - 1}{\gamma}}]^\frac{\gamma}{\gamma - 1}
\end{equation}

In this is setting, $H$ denotes high-skilled (university) labour supply, $L$ low-skilled (non-university) labour supply, $\gamma$ is the elasticity of substitution between high skill and low skill labor and $\theta$ determines relative importance the two types of labour in the production function. Primary measure of inequality used is a (log) skill premium between these two types of labour. We can get this premium by first deriving wage for both $L$ and $H$ and obtaining their ratio:
\[\frac{w_{H}}{w_{L}} = \frac{(1 - \theta)}{\theta} \left(\frac{H}{L}\right)^{-\frac{1}{\gamma}}\left(\frac{A_{H}}{A_{L}}\right)^{\frac{\gamma - 1}{\gamma}}\]

and then linearizing the equation by taking logs:
\[\log(\frac{w_{H}}{w_{L}}) = c + \frac{\gamma - 1}{\gamma}\log(\frac{A_{H}}{A_{L}}) - \frac{1}{\gamma}\log(\frac{H}{L})\]
In this function the relative supply of skilled labour $\frac{H}{L}$ decreases the skill premium whereas the unobserved $\frac{A_{H}}{A_{L}}$ parameter measuring relative development of factor augmenting technology for high and low skilled labour represent the skill-biased technological change. We assume that this variable has a log linear trend. This assumption contains a key part of the model - permanently ongoing technological change increasing demand for skilled labour. Thus we obtaining the final version of the equation:
\begin{equation}
\label{eqn:STBC_regression}
\log(\frac{w_{H}}{w_{L}}) = c + \frac{\gamma - 1}{\gamma}\sigma_0 + \frac{\gamma - 1}{\gamma}\sigma_{1}t - \frac{1}{\gamma}\log(\frac{H}{L})
\end{equation}
OLS regression in form of is then estimated by Katz and Murphy (1992), Acemoglu (2012), Goldin and Katz (2020) and others in order to obtain estimate of the elasticity of substitution and an annual change in skilled labour demand.

Card and Lemieux (2001) and Glitz and Wissmann (2021) offer an extension of the model by incorporating middle skill category and distinguishing between young and old workers. Their framework result in a system of equations, allowing to obtain elasticities of substitution between different sub-group using a seemingly unrelated regression framework.

Routine Biased Technological change is then a further extension of the model above (in fact it nests the Canonical model as its specific case). The key idea of this framework is an economic activity primarily consisting of tasks that can be divided between routine and non-routine and further between cognitive and non-cognitive. The technological change is assumed to substitute the routine tasks and strengthen position of non-routine ones. The framework is thus able to explain polarization patterns visible in the US labour market in 1990s (Autor, 2014). This framework is formally elaborated by Acemoglu and Autor (2011).

\subsection{Variables Construction}
We follow proceedings of Katz and Murphy (1992) and Glitz and Wissmann (2021) while calculating variables for equation \ref{eqn:STBC_regression}. For calculation of variables and statistics below we divide our data into groups defined by sex, experience level (6 categories) and highest attained ISCED education category.
We compute an estimate of total hours worked in each group and each year as weeks worked times usual weekly hours and personal sample weight from the survey. In a similar fashion we compute average weekly wages for full-time workers for each of the groups defined above.

We sum over all individuals in the group to get a total labour supply in a given group and year. Subsequently we compute relative share of each group in total labour supply in a given year and use this measure to calculate a vector of average employment shares for each group over all available years in the sample. We use this fixed weights together with the groups' average wages matrix to calculate time series of relative wages by groups. An average of this time series through time can be interpreted as estimate of the average relative wage of a given group. \footnote{Each group's average weekly wages and weighted by the fixed weights, creating a time series of indices for each of the sample by which we then weight wages for all groups in a given year.} % this should be N' W
% Efficiency units 
The relative labour supply and other descriptive statistics below are obtained using efficiency units. Efficiency units are essentially the labour supplies (hours worked) for each education-age-gender-year group multiplied by the group's average relative wage estimate defined above. Result of this operation is then labour supply for each group in each year. We construct more aggregated measured of LS by summing over these groups in each skill group (education category, high and low).
% Labour Supply
We calculate the final supply of high vs. low skill labour supply (more precisely of tertiary vs. non-tertiary education) as a sum of respective cells for tertiary and lower than tertiary education groups.
% Wage premiums = weighted average of education-age-gender-year group's wages
To get the skill wage premium, we use the average wages for each education-age-gender-year group, mentioned above. We calculate the high/low skill group's composition-adjusted wage as a weighted average of the groups' wages with weights defined as each group's average share of the respective skill group's (high/low) total labour supply over all observed years (i.e. we give the highest weight to the wages which provided the highest share of labour supply in the given skill group).
% check this, primary source KM Table II!
In line with the equation \ref{eqn:STBC_regression}, we than use a logarithm of in the regressions below....

% TODO: add estimation info such as efficiency unies, etc.; see Acemoglu and also GLitz and Wissmann. Expand massively, also add much more description to the graphs.

\section{Results}
\subsection{Descriptive analysis}
In our analysis we used the Eurostat EU-SILC database for the CEE countries (Czech Republic, Slovakia, Poland, Hungary, Latvia, Estonia, Lithuania, Romania and Bulgaria) between years 2005 and 2019. EU-SILC collects information about income, poverty, social exclusion and living conditions across the EU countries. Our database contains records of more than 2 million individuals in total.\\ In the analysis below we aggregated the countries into three regional blocks - Central Europe (the "Visegrad"), Baltic (Latvia, Estonia and Lithuania) and two Balkan countries - Romania and Bulgaria. This division was inspired by geographical and socioeconomic closeness of the countries.\footnote{We add Slovenia to the Central European countries, as its macroeconomic performance resembles fits them better than Romania and Bulgaria} % TODO: this part would be worth precising, also maybe move this to another section?
The results below refer to these regional blocks, results for individual countries are presented in the Appendix. We choose 2011-2019 as our main period of interest due to data limitations (data for Bulgaria and Romania and various variable changes during the period of time) and also in order to concentrate on post the 2008 crisis era.

% Table 1 and Table 2 - wage and LS changes tables with commentary
% TODO: compare to the US case (KM 92)
Table \ref{real_wage_changes_agg} shows log changes in real wage for several groups of full-time workers for two time periods - 2007$/$2010 and 2011$/$2019. We calculate the real wages by deflating nominal wages in each period by the country's Harmonised index of consumer prices obtained from Eurostat. Subsequently we calculate the changes for males and females, five education categories and six experience groups. First of all, we note quite strong real wage growth for the highest education category (tertiary education or higher) in all regions apart from the Central Europe where we find small decline between 2011 and 2019. In Baltic and Balkan case is notable growth across all categories. The situation in the Central Europe is somewhat different at least in the later periods - we can see decline in both highest and lowest education category contrasting with wage increases across secondary education. Real wages also declined for the most experienced workers in this region ($>$ 45).\footnote{Note however that the results for primary education category is driven by relatively low number of observations and only two countries - Poland and Slovakia}.
In a similar fashion, table \ref{labour_supply_changes_agg} shows changes in relative labour supply, concretely we depict log changes in each group's share of total labor supply measured in efficiency units. The results show steady rise of female share in the labour supply and also rise of the highest education categories across the regional groups. Among the experience categories we see rising labour supply for the higher experience groups and declines in case of workers with less than five years of working experience, probably a sign of population ageing.

% Figure 11 and 12 - here 
In Figure \ref{low_deciles_vs_min_w} we can see a development of the log minimal wage against the 1st percentile and decile of the wage distribution. We can see that the minimal wage closely copies the first decile of our sample. We can also note a diverging trend for Romania and Bulgaria after 2015. Figure \ref{high_deciles_vs_meam_w} than portrays a similar picture for higher segments of the distribution and log of mean wage for given region for Eurostat (EU-SES).

% Figure 1 - TODO: any interpretation of those numbers
% TODO: 50/10 seems to be significantly more volatile than 90/50 
To sum-up general development of labour market inequality, Figure \ref{agg_wage_gaps_CEE} portrays 50/10 and 90/50 wage gap (ratio of percentiles of an annual log wage distribution) development for full-time workers in the three regions defined above.
Figure \ref{agg_wage_gaps_CEE} shows that the pay gaps developed differently in the three regions, despite being generally different to the US data as found for example in (Mishel, Shierholz and Schmitt, 2013). In case of the central European countries there is a tendency for relatively long-term and monotonic decrease for the 90/50 wage gap, a finding contrary to the US and Western Europe evidence as well as the hypothesis of job polarization and "hollowing-out" of the middle class. We can also note mostly decreasing tendency of the 50/10 ratio a movement more in line with the US evidence as the least paid jobs seem to be catching up with the median. Similar tendency can be observed for the Baltic countries with a notable exception of steep rise of wage inequality after the crisis in 2008, this phenomena is visible yet less pronounced in the Visegrad countries as well. The Balkan countries on the other hand experience rather flat 90/50 ratio after 2012 and reversed trend in the lowest 50/10 in recent years. On the other hand, there is no negative reaction to the 2008 crisis with the ratios continuing to decline around the year 2010.
In general we can see strengthening of the lower part of the income distribution in Central Europe, to lesser degree in the other two regions. The development also contrasts with German experience where we can see a general upward trend for both wage gaps (90/50 and 50/10) at least until 2015 (Biewen and Sturm, 2021).


%Figure 2 - look at Acemoglu, 2012 and so on I consider extending the graph description if necessary.
% TODO magnitudes, compare with Acemoglu
In Figure \ref{agg_wage_changes_percentiles_11_19} we can find changes in log wages percentiles of full-time workers relative to the median between 2011 and 2019. Unlike the U-shaped curve characteristic of wage polarization visible in the similar plot for the US (Acemoglu and Autor, 2011) we can instead observe a monotonic behaviour for the Visegrad counties with a clear tendency for decline of the highest percentiles relative to median. Similar, yet less pronounced picture is visible for the Baltics whereas the same plot for Romania and Bulgaria shows rather contrasting picture with declining lowest percentiles and tendency for an increase for two highest deciles.
\footnote{Note that Figure \ref{agg_wage_changes_percentiles} shows that for 2007-2019 period the graphs for Baltic and Balkan countries, in contrast, show declining tendency from lowest percentiles and rather flat behaviour afterwards.} In general, the graphs confirm the conclusions from figure \ref{wage_gaps_CEE}. In contrast to the Western evidence we do not find relative rise of the highest incomes even though the outcomes differs among regions.


% Figure 3
% TODO magnitudes, compare with Acemoglu
Figure \ref{agg_employ_changes_percentiles} shows changes of employment shares for the ISCO-08 occupations skill rank between 2011-2019 \footnote{The starting year of the analysis is chosen due to changes in ISCO classification in the EU-SILC dataset}, it also depicts a locally-weighted smoothing regression curve. Compared to the US evidence, we most notably never see a characteristic U-shape found in Acemoglu (2012) for the 90s and 00s US labour market interpreted as the job-polarization. We can notice a rather different and diversified, however quite monotonic behaviour across the CEE countries. There is a declining tendency for the high-income occupations in case of Central Europe, whereas for the Balkan states we an opposite tendency for rising employment high-income occupations. Compared to the US benchmark however, we should note that all the curves are rather flat with the magnitude of the changes in employment shares being relatively low, an phenomenon especially pronounced in the Baltics (note this conclusion is not reached in case of Figure \ref{agg_wage_changes_percentiles_11_19}. We can also note an increasing variance of the estimates in Balkan and Baltics in upper half of the distribution. Overall, the results do not confirm the job polarization logic but we also do not confirm any other strong tendency (perhaps with exception of the Central Europe). 

% Figures 4, 5 describing NACE/ILO codes decomposition for wages and employment
Figures \ref{wage_changes_nace} and \ref{employ_changes_nace} show the relative changes in NACE classification of economic activities. We can note especially strong growth of wages in manufacturing and construction in central European countries  - this seems to be in line strong position of these countries in the European manufacturing core (Stöllinger, 2016). On the other hand - key public sectors are falling behind in this region and we see a strong performance of Finance in Baltic countries. In Romania and Bulgaria, the public sectors are on the other hand rising. Figure \ref{employ_changes_nace} then documents that changes in relative wage in the NACE categories are often associated with moves in relative employment in an opposite direction (see Construction in Central Europe or Finance in the Baltics). This seems be in line in one of the fundamental hypotheses about interaction of relative supply and relative wages (Katz and Murphy, 1992).
When we compare similar plots using ILO employment categories \ref{wage_changes_ilo} and \ref{employ_changes_ilo} we again do not find a unified picture among the regions. The Figure \ref{employ_changes_ilo} allows us a basic comparison with trends both US and Western Europe where we, in line with the job polarization hypothesis, find high and low-education occupation growing at the expense of the middle-education occupations such as assemblers or clerks (Acemoglu and Autor, 2011). We see similar behavior for Baltic states with Managers and Professions growing in the relative employment share together with Elementary occupations at the other side of the spectrum. But in the two remaining regions the evidence seems to be less straightforward. Looking at the situation for wages in Figure \ref{wage_changes_ilo}, we can again spot frequent movement of wages and employment in opposite direction. From the particular results, one can note a strong performance of Machine operators in case of Central Europe (rather middle-educated, potentially routine and therefore offshorable occupations), result especially contrasting with the Balkan economies.

In sum, our descriptive analysis has shown differences of CEE labour markets with respect to the stylized facts found in the developed countries - in particular we do not confirm either job or wage polarization in CEE and we never see a monotonic rise in inequality similar to certain periods of the US development. Yet, at the same time there is rather diversified behaviour between the investigated regions themselves. In general, we can contrast generally declining measures of inequality in the the Central European countries to mostly stagnating situation in the Baltic and signs of an increasing inequality in case of the Balkan countries.
% TODO: this is really a stump, expand on it
This result come as a certain surprise to us. All the countries should have similar position in the global supply chain. We should also note that the Central European and Baltic countries has very similar type of labour market. According to OECD (2019) their bargaining systems can be with exception of Slovenia be in all cases classified as fully or largely decentralized (Romania and Bulgaria are not covered).
The differences can perhaps be seen in sectoral specialization as also documented in Figures \ref{wage_changes_nace} and \ref{employ_changes_nace}, where Central European countries show biggest pay rise in manufacturing, Baltics in finance and ICT and Romanian and Bulgaria in Education and Health, result pointing the significantly diversified economies. Stöllinger (2016) for example speaks about the European Manufacturing core which does include central European states yet not other states in our data.
% some sources: https://www.oecd.org/economy/surveys/bulgaria-2021-OECD-economic-survey-overview.pdf
%https://www.oecd-ilibrary.org/sites/bf4a7892-en/index.html?itemId=/content/component/bf4a7892-en#chapter-d1e1343
% overview: https://ec.europa.eu/eurostat/web/products-eurostat-news/-/DDN-20190917-1
%Gini: https://data.worldbank.org/indicator/SI.POV.GINI?end=2018&locations=BG-CZ-RO-HU&start=2002&fbclid=IwAR1PFRKBUVf-Gmq4NdR1r4bjuQrNCrfQXIst_SQcG97p1dcpmO7WNy5heag
Our results so far, despite bringing interesting view of the economies, do not tell us any decisive conclusion about the SBTC hypothesis. We can only tell that our results, especially those for the Central Europe (which are actually inverse to the US scenario), differ from those of the West, they however still could be in line with either SBTC or RBTC theory, given that either high skill labour supply growth is high enough or the countries are recipients of routine jobs from abroad thanks to globalization.


\subsection{Elasticity of substitution estimation}
% Panel Regression
% Main variables
In the next part we get to the crucial part of the canonical model doctrine - (presumably) negative relationship between relative labour supply and relative wages. As can be seen in the plots \ref{agg_labour_supplies} and \ref{agg_high_low_log_wage_gap} a rising tendency in relative high-skill supply rise across CEE on one hand (as high skill category we define individuals with the highest ISCED education level attained greater or equal to 5). On the other hand decline in university/college wage premium can be seen in case of the Central Europe, however the Baltics show rather volatile development with pronounced increase in the premium after 2008 and downward trend afterward. Balkan countries experience a sharp rise in the premium after 2015. These results thus confirm some of the conclusions from the descriptive part. 
The negative correlation between the variables - expected by the theory seems to be evident for the Visegrad but not outside of it. If we, following Acemoglu (2012), de-trend the series as can be visible in Figure \ref{agg_km_vars_detrended} this negative relationship disappears - correlation for all regions as well as countries (Fig \ref{km_vars_detrended} in the Appendix) is actually positive. This is a contrasting result to the US in Acemoglu (2012). The only difference are actually Romania and Bulgaria.
We now take a look at the panel regression for years 2005-2019 to further check if this relationship is statistically significant.\\
% TODO - maybe comment on separate countries???
%%%% Regression results
Following key works in the field such as Katz and Murphy (1992) or Acemoglu (2012), we perform \ref{eqn:STBC_regression} separately for each country - the results confirm the conclusions of the detrended series analysis above - the estimates are insignificant with the exception of Romania and Bulgaria that have significantly negative coefficients with magnitude suggesting higher elasticity than in the US case (-0.27 and -0.34 respectively)  \footnote{We have also done this exercise for the regional groups - elasticities are again insignificant except for the Balkans}. % TODO present the results, maybe put here some comment from Havranek
However, as the regressions above work with short time series we decided to utilize panel regression.
In tabls \ref{panel_regression_comparison} we see the regression results of several regression specifications coming from the basic regression design proposed by Katz and Murphy (1992) in the form of equation \ref{eqn:STBC_regression}, we also add  other explanatory variables inspired by research on determinants of labour market inequality. This regression equation has been used extensively, in studies predominantly concentrated on developed economies in last decades (Havranek et al., 2021) and thus is a good basis for comparison. As suggested by Farber (2018), we added union density for each year as a measure of labour market conditions, we also added average minimum wage in each of the countries.\\
The H/L parameter in table \ref{panel_regression_comparison} can be interpreted as an elasticity of substitution between high and low skilled workers and is therefore of a primary interest . The Random effect model also contains time trend parameter is interpreted as an annual change in relative high skill demand caused by technological change.


The results show that the relative supply coefficient is negative and significant and between (-0.2 and -0.1) this implies elasticity of substitution around 5 or 10 \footnote{We also performed Hausmann test in order to choose prefered model variant, with p-value 0.04551 (for the model including Union density variable) we reject the null hypothesis of RE model.}. The key result is thus that according to these results high and low skill labour are gross substitutes. As Havranek (2020) notes, this implies that high and low skilled workers are relatively interchangeable and crucially, increase in supply of high skill workers decreases demand for the low-skilled ones. The results also suggest significantly higher elasticity of substitution in the CEE compared to the US and German case (Acemoglu, 2012; Glitz and Wissmann, 2021) and at the same time significantly lower pace of technological change. This result could be in line with findings of previous literature such as (Arend and Grabowski, 2019 and Hardy et al., 2018) - who notes significant educational upgrading in the region together with high demand for routine (even though cognitive) jobs in result of CEE being and offshoring destination. This may be the creating excessive supply high-skilled population that is subsequently being pushed into less high-skilled occupations - resulting in a higher substitutibility.
%However, the in many model specifications the the coefficients are insignificant or even positive. This results, contrast with expectations given from the theory -that implies non-negativity of the estimates, as well as many empirical studies (Katz and Murphy has ... for the US), yet seem to be in line with tendency of the literature to present upward biased estimates (Havranek et al., 2021).

Also note that technological change seems to have negative sign despite a miniscule magnitude, this would imply that technological change has an equalizing effect (decrease in skill premium) in CEE.

Overall, we can identify a marked difference of the observed basic labour market patterns in the CEE compared to US and German evidence (and we can probably say Western experience in general) that to bigger or lesser degree experienced and job and wage polarization. We do not confirm such a phenomenon in the CEE and rather see strengthening of the middle and bottom part of the distribution compared to the highest quantiles.

\section{Conclusion}

\section{References}
\begin{enumerate}
\item Acemoglu, Daron. "What does human capital do? A review of Goldin and Katz's The race between education and technology." Journal of Economic Literature 50.2 (2012): 426-63.

\item Acemoglu, Daron, and David Autor. "Skills, tasks and technologies: Implications for employment and earnings." Handbook of labor economics. Vol. 4. Elsevier, 2011. 1043-1171.

\item Arendt, Łukasz, and Wojciech Grabowski. "Technical change and wage premium shifts among task-content groups in Poland." Economic research-Ekonomska istraživanja 32.1 (2019): 3392-3410

\item Autor, David. Polanyi's paradox and the shape of employment growth. Vol. 20485. Cambridge, MA: National Bureau of Economic Research, 2014.

\item Baldwin, Richard, and Javier Lopez‐Gonzalez. "Supply‐chain trade: A portrait of global patterns and several testable hypotheses." The world economy 38.11 (2015): 1682-1721.

\item Biewen, Martin, Bernd Fitzenberger, and Jakob De Lazzer. "Rising wage inequality in Germany: Increasing heterogeneity and changing selection into full-time work." ZEW-Centre for European Economic Research Discussion Paper 17-048 (2017).

\item Biewen, Martin, and Matthias Seckler. "Unions, internationalization, tasks, firms, and worker characteristics: A detailed decomposition analysis of rising wage inequality in Germany." The Journal of Economic Inequality 17.4 (2019): 461-498.

\item Biewen, Martin, and Miriam Sturm. Why a Labour Market Boom Does Not Necessarily Bring Down Inequality: Putting Together Germany's Inequality Puzzle. No. 14357. Institute of Labor Economics (IZA), 2021.

\item Card, David, and Thomas Lemieux. 2001. "Can Falling Supply Explain the Rising Return to College for Younger Men? A Cohort-Based Analysis." Quarterly Journal of Economics 116(2)

\item Schank, Thorsten, and Mario Bossler. "Wage inequality in Germany after the minimum wage introduction." VfS Annual Conference 2020 (Virtual Conference): Gender Economics. No. 224543. Verein für Socialpolitik/German Economic Association, 2020.

\item Rodrik, Dani, and Stefanie Stantcheva. "Economic Inequality and insecurity: Policies for an inclusive economy." Report for the Blanchard-Tirole Commission (2020).

\item Aziz, Imran, and Guido Matias Cortes. "Between-group inequality may decline despite a rising skill premium." Labour Economics 72 (2021)

\item Hardy, Wojciech, Roma Keister, and Piotr Lewandowski. "Educational upgrading, structural change and the task composition of jobs in Europe." Economics of Transition 26.2 (2018): 201-231.

\item Howell, David R., and Arne L. Kalleberg. "Declining job quality in the United States: Explanations and evidence." RSF: The Russell Sage Foundation Journal of the Social Sciences 5.4 (2019): 1-53.

\item Goldin, Claudia, and Lawrence F. Katz. "Extending the Race between Education and Technology." AEA Papers and Proceedings. Vol. 110. 2020

\item Glitz, Albrecht, and Daniel Wissmann. "Skill premiums and the supply of young workers in Germany." Labour Economics 72 (2021): 102034.

\item Dustmann, Christian, et al. "From sick man of Europe to economic superstar: Germany's resurgent economy." Journal of Economic Perspectives 28.1 (2014): 167-88

\item Havranek, Tomas, et al. "The elasticity of substitution between skilled and unskilled labor: A meta-analysis." (2020).

\item Marin, D. (2004): “A Nation of Poets and Thinkers – Less So With Eastern Enlargement? Austria and Germany,” Discussion Paper 4358, Centre for Economic Policy Research, London

\item Marin, Dalia. "Global Value Chains, Product Quality, and the Rise of Eastern Europe." Explaining Germany’s Exceptional Recovery (2018): 41

\item Magda, Iga, Jan Gromadzki, and Simone Moriconi. "Firms and wage inequality in Central and Eastern Europe." Journal of Comparative Economics 49.2 (2021): 499-552.

\item Mishel, Lawrence, Heidi Shierholz, and John Schmitt. 2013. “Don’t Blame the Robots: Assessing the Job Polarization Explanation of Growing Wage Inequality.”

\item Katz, Lawrence F., and Kevin M. Murphy. "Changes in relative wages, 1963–1987: supply and demand factors." The Quarterly Journal of Economics 107.1 (1992): 35-78.

\item Tyrowicz, Joanna, and Magdalena Smyk. "Wage inequality and structural change." Social Indicators Research 141.2 (2019): 503-538.

\item Magda, Iga, Jan Gromadzki, and Simone Moriconi. "Firms and wage inequality in Central and Eastern Europe." Journal of Comparative Economics 49.2 (2021): 499-552.

\item Mysíková, M., and Večerník, J. (2018). Personal Earnings Inequality and Polarization: The Czech Republic in Comparison with Austria and Poland. Eastern European Economics, 56(1), 57–80.

\item OECD (2019), Negotiating Our Way Up: Collective Bargaining in a Changing World of Work

\item Stöllinger, Roman. "Structural change and global value chains in the EU." Empirica 43.4 (2016): 801-829.

\end{enumerate}
\section{Tables and Figures}


\begin{figure}[!htbp]%
    \centering
    {\includegraphics[scale = 0.45]{wage_gaps_agg.png} }
    \caption{Development of (log) wage gaps for fulltime workers in CEE aggregate to regional blocks,  2005–2019}
    \label{agg_wage_gaps_CEE}
\end{figure}



\begin{figure}[!htbp]%
    \centering
    {\includegraphics[scale=0.5]{agg_wage_changes_percentiles_11_19.png} }
    \caption{Changes in Log Hourly Wages by Percentile Relative to the Median (2011-2019}
    \label{agg_wage_changes_percentiles_11_19}
\end{figure}


\begin{figure}[!htbp]%
    \centering
    {\includegraphics[scale=0.5]{agg_employ_changes_percentiles.png} }
    \caption{Changes in employment by occupational skill percentile, 2011–2019. Mean log-wage in 2011 was used for obtaining the occupation skill rank}
    \label{agg_employ_changes_percentiles}
\end{figure}

\begin{figure}[!htbp]%
\centering
    {\includegraphics[scale=0.5]{employ_changes_nace.png} }
    \caption{Employment changes between 2011-2019 by NACE category}
\label{employ_changes_nace}
\end{figure}

\begin{figure}[!htbp]%
\centering
    {\includegraphics[scale=0.5]{wage_changes_nace.png} }
    \caption{Changes in log wages relative to median by NACE category (2011-2019)}
\label{wage_changes_nace}
\end{figure}

\begin{figure}[!htbp]%
\centering
    {\includegraphics[scale=0.5]{employ_changes_ilo.png} }
    \caption{Employment changes between 2011-2019 by ILO major category}
\label{employ_changes_ilo}
\end{figure}

%Figure 8
\begin{figure}[!htbp]%
\centering
    {\includegraphics[scale=0.5]{wage_changes_ilo.png} }
    \caption{Changes in log wages relative to median by ILO major category (2011-2019)}
\label{wage_changes_ilo}
\end{figure}

% Figure 9
\begin{figure}[!htbp]%
    \centering
    {\includegraphics[scale=0.6]{agg_high_low_log_wage_gap.png} }
    \caption{Changes in composition adjusted high/low-skill log wage premium}
    \label{agg_high_low_log_wage_gap}
\end{figure}


\begin{figure}[!htbp]%
    \centering
    {\includegraphics[scale=0.6]{agg_labour_supplies.png} }
    \caption{Changes in relative high/low skill labour supply}
    \label{agg_labour_supplies}
\end{figure}


\begin{figure}[!htbp]%
    \centering
    {\includegraphics[scale=0.6]{agg_km_vars_detrended.png} }
    \caption{Detrended skill wage premium against relative labour supply}
    \label{agg_km_vars_detrended}
\end{figure}



\begin{figure}[!htbp]%
    \centering
    {\includegraphics[scale=0.6]{agg_min_wages.png} }
    \caption{Minimal Wages against the lowest percentiles}
    \label{low_deciles_vs_min_w}
\end{figure}

\begin{figure}[!htbp]%
    \centering
    {\includegraphics[scale=0.6]{agg_high_deciles_against_mean.png} }
    \caption{Mean Wages against the highest percentiles}
    \label{high_deciles_vs_meam_w}
\end{figure}


% Aggregate - Table 1
\begin{table}[!htbp]
\centering 
\caption{Changes of real wages for different groups of countries. The table depicts a log changes in real weekly wages of full-time workers for specified periods. The wages are defined using the sex-education-experience groups defined above, the aggregate categories displayed are then weighted average of relevant groups using average employment shares of given group over the entire sample period as weight}
\label{real_wage_changes_agg}
\begin{center}
\resizebox{\textwidth}{!}{


\begin{tabular}{lrrrrrr}
\toprule
{} &     Ro+Bu &     Ro+Bu &   Central Europe &   Central Europe &     Baltic &     Baltic \\
{} &  2010/2007 &  2019/2011 &  2010/2007 &  2019/2011 &  2010/2007 &  2019/2011 \\
%country\_group &     balkan &     balkan &   visegrad &   visegrad &     baltic &     baltic \\
Groups                                &            &            &            &            &            &            \\
\midrule
%(upper) secondary education           &   9.345806 &  59.931305 &  11.642997 &  13.936923 &  10.444225 &  35.929474 \\
$<$5                                    &  14.797624 &  65.137344 &   7.185108 &   7.891582 &  -2.950337 &  36.497808 \\
5-15                                  &   7.971447 &  63.154173 &   9.118209 &  -0.975471 &  13.526565 &  31.962220 \\
15-25                                 &   8.862724 &  66.525961 &   9.272641 &   4.759290 &  16.791721 &  37.971652 \\
25-35                                 &   8.086467 &  58.721456 &   9.602545 &   0.797051 &  20.272556 &  33.163872 \\
35-45                                 &   4.723792 &  61.367711 &   9.641290 &   0.610153 &  18.121019 &  28.695666 \\
%5-15                                  &   7.971447 &  63.154173 &   9.118209 &  -0.975471 &  13.526565 &  31.962220 \\
%$<$5                                    &  14.797624 &  65.137344 &   7.185108 &   7.891582 &  -2.950337 &  36.497808 \\
$>$45                                   & -76.867792 &  86.427481 &  22.566759 &  -1.274856 &  21.462457 &  33.994379 \\
Female                                &   8.142511 &  66.488347 &   7.900273 &   2.301673 &  16.894605 &  32.871068 \\
Male                                  &   8.070425 &  60.904215 &   9.890782 &   2.572075 &  10.693629 &  35.108801 \\
primary education                     &   4.736836 &  69.037100 &   1.191744 & -50.229431 &  18.815427 &  75.794649 \\
lower secondary education             &   1.236062 &  55.677731 &  13.769496 &  16.239395 &   6.002626 &  27.930662 \\
(upper) secondary education           &   9.345806 &  59.931305 &  11.642997 &  13.936923 &  10.444225 &  35.929474 \\
post-secondary non tertiary education &   6.265056 &  77.476501 &   9.534267 &   2.034679 &  16.617963 &  31.882166 \\
%primary education                     &   4.736836 &  69.037100 &   1.191744 & -50.229431 &  18.815427 &  75.794649 \\
tertiary education                    &   8.404465 &  65.094797 &   7.724367 &  -4.659190 &  14.562024 &  34.033161 \\
\bottomrule
\end{tabular}

}
\end{center}
\end{table}



% Aggregate - Table 2
% TODO - rows ordering
\begin{table}[!htbp]
\centering 
\caption{Changes of labour supply for different groups of countries. The table presents log changes in the share of total labour supply provided by given group in the specified period. The labour supply is measured in the efficiency units.}
\label{labour_supply_changes_agg}
\begin{center}
\resizebox{\textwidth}{!}{


\begin{tabular}{lrrrrrr}
\toprule
{} &  2010/2007 &  2019/2011 &  2010/2007 &  2019/2011 &   2010/2007 &  2019/2011 \\
country\_group &     balkan &     balkan &   visegrad &   visegrad &      baltic &     baltic \\
Groups                                &            &            &            &            &             &            \\
\midrule
%(upper) secondary education           &  -0.856678 &  -1.930573 &  -5.771289 &  -2.719726 &  -12.278409 & -17.813729 \\
$<$5                                    &  -2.383018 & -31.097151 &   2.249316 & -43.438851 &   -4.427445 & -16.999040 \\
15-25                                 &   6.985925 &  -3.933818 &  -7.453282 &   8.174225 &   -8.145433 & -18.880740 \\
25-35                                 & -10.210762 &   8.558053 &  -1.978402 &  -4.703112 &    4.217927 & -15.607869 \\
35-45                                 &  25.112974 &  12.351141 &  21.275877 &  32.029868 &   10.256839 &  38.920874 \\
%5-15                                  &  -4.671684 &  -3.833148 &   0.796531 &  -2.034299 &    5.413153 &  17.826651 \\
%$<$5                                    &  -2.383018 & -31.097151 &   2.249316 & -43.438851 &   -4.427445 & -16.999040 \\
$>$45                                   & -36.640996 &  59.656172 &  14.065841 &  71.537131 &  -19.632433 &  48.323612 \\
Female                                &   1.296522 &  -1.930132 &   4.091550 &   2.007874 &    8.494020 &  -4.176115 \\
Male                                  &  -0.795805 &   1.223618 &  -2.551984 &  -1.325431 &   -6.453240 &   3.177369 \\
primary education                     & -50.254843 & -14.738936 & -20.863042 & -50.979049 &  -12.331598 & -32.205389 \\
lower secondary education             &  -7.603790 & -26.757482 &  -1.851176 &  24.034607 &  -12.826862 & -32.896778 \\
(upper) secondary education           &  -0.856678 &  -1.930573 &  -5.771289 &  -2.719726 &  -12.278409 & -17.813729 \\
post-secondary non tertiary education &  -1.729285 &   5.109752 &   0.797851 & -35.876316 &  -30.872638 &  12.805771 \\
%pre-primary education                 &  59.345873 &       -inf & -91.783450 &       -inf & -117.468738 &       -inf \\
%primary education                     & -50.254843 & -14.738936 & -20.863042 & -50.979049 &  -12.331598 & -32.205389 \\
tertiary education                    &   8.452129 &  10.272401 &  11.712725 &   7.441955 &   20.472786 &  10.584279 \\
\bottomrule
\end{tabular}

}
\end{center}
\end{table}



\begin{table}[!htbp]
\centering 
\caption{Panel Regression Comparison - clustered standard errors reported}
\label{panel_regression_comparison}
\begin{center}
\resizebox{\textwidth}{!}{



\begin{tabular}{lccccccc}
\toprule
                                 & \textbf{FE - entity effects} & \textbf{FE- time effects inc.} & \textbf{FE - UD} &    \textbf{RE}    & \textbf{RE - entity} &  \textbf{RE - UD} & \textbf{RE - UD2}  \\
\midrule
\textbf{Dep. Variable}           &           logW\_HL           &            logW\_HL            &     logW\_HL     &      logW\_HL     &       logW\_HL       &      logW\_HL     &      logW\_HL      \\
\textbf{Estimator}               &           PanelOLS           &            PanelOLS            &     PanelOLS     &   RandomEffects   &    RandomEffects     &   RandomEffects   &   RandomEffects    \\
\textbf{No. Observations}        &             144              &              144               &       144        &        144        &         144          &        144        &        144         \\
\textbf{Cov. Est.}               &          Clustered           &           Clustered            &    Clustered     &     Clustered     &      Clustered       &     Clustered     &     Clustered      \\
\textbf{R-squared}               &            0.3370            &             0.0836             &      0.1608      &       0.3150      &        0.6952        &       0.3689      &       0.7174       \\
\textbf{R-Squared (Within)}      &            0.3370            &             0.3365             &      0.3614      &       0.3366      &        0.3416        &       0.3838      &       0.3897       \\
\textbf{R-Squared (Between)}     &            0.3271            &             0.3375             &      0.8437      &      -0.3251      &        1.0000        &      -0.1149      &       1.0000       \\
\textbf{R-Squared (Overall)}     &            0.3262            &             0.3362             &      0.8307      &      -0.0177      &        0.6952        &       0.1183      &       0.7174       \\
\textbf{F-statistic}             &            67.617            &             10.861             &      7.4737      &       32.422      &        27.366        &       20.314      &       25.392       \\
\textbf{P-value (F-stat)}        &            0.0000            &             0.0013             &      0.0001      &       0.0000      &        0.0000        &       0.0000      &       0.0000       \\
\textbf{=====================}   &         ===========          &          ===========           &   ===========    &  ===============  &   ===============    &  ===============  &  ===============   \\
\textbf{log(H\_L)}               &           -0.2198            &            -0.2285             &     -0.1937      &      -0.1086      &       -0.1685        &      -0.0511      &      -0.1115       \\
\textbf{ }                       &          (-3.6296)           &           (-2.1611)            &    (-1.8663)     &     (-1.6910)     &      (-2.2977)       &     (-0.8928)     &     (-1.6384)      \\
\textbf{UD}                      &                              &                                &      0.0102      &                   &                      &       0.0092      &       0.0098       \\
\textbf{ }                       &                              &                                &     (1.7155)     &                   &                      &      (2.2486)     &      (1.7258)      \\
\textbf{min\_wage}               &                              &                                &      0.0003      &                   &                      &     2.724e-05     &      5.58e-05      \\
\textbf{ }                       &                              &                                &     (0.8571)     &                   &                      &      (0.1870)     &      (0.2645)      \\
\textbf{trend}                   &                              &                                &                  &      -0.0057      &       -0.0029        &      -0.0031      &      -0.0006       \\
\textbf{ }                       &                              &                                &                  &     (-1.0212)     &      (-0.4728)       &     (-0.4082)     &     (-0.0555)      \\
\textbf{const}                   &                              &                                &                  &       0.4819      &        0.3871        &       0.3401      &       0.2467       \\
\textbf{ }                       &                              &                                &                  &      (5.6607)     &       (4.2619)       &      (3.3670)     &      (1.9107)      \\
\textbf{Country.CZ}              &                              &                                &                  &                   &       -0.0265        &                   &      -0.0157       \\
\textbf{ }                       &                              &                                &                  &                   &      (-1.2582)       &                   &     (-0.7569)      \\
\textbf{Country.EE}              &                              &                                &                  &                   &        0.0010        &                   &       0.0447       \\
\textbf{ }                       &                              &                                &                  &                   &       (0.0431)       &                   &      (0.6113)      \\
\textbf{Country.HU}              &                              &                                &                  &                   &        0.1514        &                   &       0.1664       \\
\textbf{ }                       &                              &                                &                  &                   &       (4.4663)       &                   &      (3.1447)      \\
\textbf{Country.LT}              &                              &                                &                  &                   &        0.1589        &                   &       0.1760       \\
\textbf{ }                       &                              &                                &                  &                   &       (3.6748)       &                   &      (2.6030)      \\
\textbf{Country.LV}              &                              &                                &                  &                   &        0.1765        &                   &       0.1540       \\
\textbf{ }                       &                              &                                &                  &                   &       (18.105)       &                   &      (3.3028)      \\
\textbf{Country.PL}              &                              &                                &                  &                   &       -0.0270        &                   &      -0.0512       \\
\textbf{ }                       &                              &                                &                  &                   &      (-0.6965)       &                   &     (-0.8000)      \\
\textbf{Country.RO}              &                              &                                &                  &                   &        0.0361        &                   &      -0.0416       \\
\textbf{ }                       &                              &                                &                  &                   &       (0.9983)       &                   &     (-0.7801)      \\
\textbf{Country.SI}              &                              &                                &                  &                   &        0.1628        &                   &       0.0014       \\
\textbf{ }                       &                              &                                &                  &                   &       (5.7925)       &                   &      (0.0095)      \\
\textbf{Country.SK}              &                              &                                &                  &                   &       -0.1817        &                   &      -0.1726       \\
\textbf{ }                       &                              &                                &                  &                   &      (-6.8356)       &                   &     (-7.1373)      \\
\textbf{=======================} &        =============         &         =============          &  =============   & ================= &  =================   & ================= & =================  \\
\textbf{Effects}                 &            Entity            &             Entity             &      Entity      &                   &                      &                   &                    \\
\bottomrule
\end{tabular}

}
%\caption{Model Comparison}
\end{center}
\end{table}












\newpage

\section{Appendix}

\begin{figure}[!htbp]%
    \centering
    {\includegraphics[scale=0.5]{agg_wage_changes_percentiles.png} }
    \caption{Changes in Log Hourly Wages by Percentile Relative to the Median (2007-2019}
    \label{agg_wage_changes_percentiles}
\end{figure}



\begin{figure}[!htbp]%
    \centering
    {\includegraphics[scale = 0.45]{wage_gaps.png} }
    \caption{Development of (log) wage gaps for fulltime workers in CEE, 2005–2019}
    \label{wage_gaps_CEE}
\end{figure}

\begin{figure}[!htbp]%
    \centering
    {\includegraphics[scale=0.5]{wage_changes_by_percentiles.png} }
    \caption{Changes in Log Hourly Wages by Percentile Relative to the Median}
    \label{wage_changes_percentiles}
\end{figure}

\begin{figure}[!htbp]%
    \centering
    {\includegraphics[scale=0.5]{employ_changes_by_percentiles.png} }
    \caption{Changes in employment by occupational skill percentile, 2011–2019. Mean log-wage in 2011 was used for obtaining the occupation skill rank}
    \label{employ_changes_percentiles}
\end{figure}

\begin{figure}[!htbp]%
        \centering 
        {\includegraphics[scale=0.45]{labour_supplies_cee.png}}
        \caption{Changes in relative high/low skill labour supply in CEE}
        \label{labour_supplies_cee}
\end{figure}

\begin{figure}[!htbp]%
    \centering 
    {\includegraphics[scale=0.45]{high_low_log_wage_gap.png}}
    \caption{Changes in composition adjusted high/low-skill log wage premium}
    \label{high_low_log_wage_gap}
\end{figure}


\begin{figure}[!htbp]%
    \centering
    {\includegraphics[scale=0.6]{km_vars_detrended.png} }
    \caption{Detrended skill wage premium against relative labour supply}
    \label{km_vars_detrended}
\end{figure}


%% Table 1 - real wage comparisons by country
\begin{table}[!htbp]
\centering 
\caption{Changes of real wages changes by country - Central Europe}
\label{Real_Wage_Changes_ce}
\resizebox{\textwidth}{!}{


\begin{tabular}{lrrrrrrrrrr}
\toprule
{} &  2010/2007 &  2019/2011 &  2010/2007 &  2019/2011 &  2010/2007 &  2019/2011 &  2010/2007 &  2019/2011 &  2010/2007 &  2019/2011 \\
Country &         CZ &         CZ &         HU &         HU &         PL &         PL &         SK &         SK &         SI &         SI \\
Groups                                &            &            &            &            &            &            &            &            &            &            \\
\midrule
%(upper) secondary education           &  16.595227 &  17.727918 &  -2.792900 &  17.749656 &   2.079740 &  18.991410 &  37.263530 &  18.773899 &   9.003570 &   4.400152 \\
$<$5                                    &  13.292203 &  12.484792 & -13.301797 &   1.992760 &   1.404850 &   2.158435 &  43.029078 &  14.963351 &   2.532791 &  -4.005816 \\
5-15                                  &  15.490846 &  12.065493 &  -9.345043 &  -4.194209 &   0.206450 &  -6.249807 &  44.631900 &  14.671115 &   6.545352 &  -8.444205 \\
15-25                                 &  15.974291 &  15.734025 &  -5.772473 &   5.305644 &  -2.669173 &   7.435828 &  36.098234 &  13.677924 &   8.370083 &  -4.945813 \\
25-35                                 &  12.657331 &  13.878316 &  -6.565035 &  -4.745099 &  -0.645663 &  -2.863376 &  33.044754 &  10.769683 &   9.746011 &  -2.121099 \\
35-45                                 &  14.679464 &   8.743428 & -10.412439 &  -6.204726 &  -4.694525 &   3.389281 &  34.904562 &  17.388578 &  13.090370 &  -3.692145 \\
%5-15                                  &  15.490846 &  12.065493 &  -9.345043 &  -4.194209 &   0.206450 &  -6.249807 &  44.631900 &  14.671115 &   6.545352 &  -8.444205 \\
%$<$5                                    &  13.292203 &  12.484792 & -13.301797 &   1.992760 &   1.404850 &   2.158435 &  43.029078 &  14.963351 &   2.532791 &  -4.005816 \\
$>$45                                   &  16.969656 &  22.386638 &  40.081575 & -57.108594 & -20.501749 & -13.079941 &  86.422321 &   1.240834 &  89.739821 &   2.103319 \\
Female                                &  16.466768 &  12.904817 & -10.231480 &  -2.504522 &   0.592534 &   1.988888 &  39.512075 &  16.257187 &   4.963890 &  -7.061669 \\
Male                                  &  13.726690 &  13.509837 &  -6.801153 &  -1.191443 &  -2.056478 &  -0.829993 &  38.315302 &  12.038894 &   9.684599 &  -3.691119 \\
primary education                     &        NaN &        NaN &        NaN &        NaN &  -6.431957 &  16.742316 &  14.892138 & -19.080876 &        NaN &        NaN \\
lower secondary education             &  12.661607 &  21.167355 &  -4.838441 &  17.668417 &   1.545489 &  36.688322 &  41.083019 &  19.807928 &   9.676708 &   8.913304 \\
(upper) secondary education           &  16.595227 &  17.727918 &  -2.792900 &  17.749656 &   2.079740 &  18.991410 &  37.263530 &  18.773899 &   9.003570 &   4.400152 \\
post-secondary non tertiary education &        NaN &        NaN &  -6.564271 &  11.965689 &   3.763429 &   8.793980 &        NaN &        NaN &        NaN &        NaN \\
%primary education                     &        NaN &        NaN &        NaN &        NaN &  -6.431957 &  16.742316 &  14.892138 & -19.080876 &        NaN &        NaN \\
tertiary education                    &  13.503349 &  10.458081 & -10.876622 & -13.326144 &  -3.078349 & -17.315706 &  39.555843 &  10.173994 &   6.793411 & -11.944414 \\
\bottomrule
\end{tabular}


}
\end{table}


\begin{table}[!htbp]
\centering 
\caption{Changes of real wages changes by country - Balkans and Baltics}
\label{Real_Wage_Changes_bb}
\resizebox{\textwidth}{!}{
\begin{tabular}{lrrrrrrrrrr}
\toprule
{} &  2010/2007 &  2019/2011 &  2010/2007 &  2019/2011 &  2010/2007 &  2019/2011 &  2010/2007 &  2019/2011 &   2010/2007 &   2019/2011 \\
Country &         BG &         BG &         EE &         EE &         LT &         LT &         LV &         LV &          RO &          RO \\
Groups                                &            &            &            &            &            &            &            &            &             &             \\
\midrule
%(upper) secondary education           &  34.169631 &  48.082543 &  15.294322 &  28.030395 &  -9.759740 &  46.361597 &  20.016842 &  39.274502 &   -9.949148 &   71.059771 \\
$<$5                                    &  47.330489 &  46.999261 &  -2.786063 &  27.692451 & -21.782520 &  43.721896 &  12.934346 &  40.860507 &  -10.227069 &   80.734811 \\
5-15                                  &  39.495721 &  58.990249 &  17.017189 &  23.508816 &  -9.050369 &  44.872473 &  27.472710 &  33.046852 &  -15.907555 &   66.301508 \\
15-25                                 &  38.971626 &  62.226653 &  24.511844 &  29.143427 &   3.417652 &  43.966947 &  19.711511 &  45.794507 &  -11.472192 &   70.141590 \\
25-35                                 &  41.241964 &  47.981114 &  25.221330 &  28.599171 &   0.019808 &  37.385062 &  29.161357 &  37.007500 &  -14.133517 &   67.328592 \\
35-45                                 &  48.925777 &  59.413050 &  23.044023 &  23.929130 &   6.100906 &  25.092660 &  23.943484 &  37.257713 &  -27.240479 &   64.900808 \\
%5-15                                  &  39.495721 &  58.990249 &  17.017189 &  23.508816 &  -9.050369 &  44.872473 &  27.472710 &  33.046852 &  -15.907555 &   66.301508 \\
%$<$5                                    &  47.330489 &  46.999261 &  -2.786063 &  27.692451 & -21.782520 &  43.721896 &  12.934346 &  40.860507 &  -10.227069 &   80.734811 \\
$<$45                                   &  16.741057 &  80.343981 &  45.896240 &  51.709741 & -24.018872 &  23.309385 &  28.600122 &  28.044531 & -170.377774 &  104.299231 \\
Female                                &  42.140465 &  59.076116 &  21.635277 &  30.356567 &   0.618485 &  33.349480 &  26.489896 &  34.992677 &  -14.005371 &   73.227794 \\
Male                                  &  40.929483 &  54.063529 &  16.225103 &  24.603861 &  -9.107269 &  46.648602 &  20.246848 &  41.156555 &  -15.908893 &   66.045212 \\
primary education                     &  18.277370 &  35.193310 &  13.006796 &  52.001774 & -10.855238 &  49.018434 &  75.204162 &  87.603874 &   -8.930614 &   94.121130 \\
lower secondary education             &  25.414493 &  29.361815 &  12.579510 &  15.265605 & -13.902503 &  34.663328 &  14.897043 &  39.491236 &  -18.434239 &   77.152514 \\
(upper) secondary education           &  34.169631 &  48.082543 &  15.294322 &  28.030395 &  -9.759740 &  46.361597 &  20.016842 &  39.274502 &   -9.949148 &   71.059771 \\
post-secondary non tertiary education &  19.765816 &  79.095391 &  18.719176 &  26.865995 &  -4.276447 &  40.013376 &  20.818130 &  38.422394 &   -6.907480 &   73.697912 \\
%primary education                     &  18.277370 &  35.193310 &  13.006796 &  52.001774 & -10.855238 &  49.018434 &  75.204162 &  87.603874 &   -8.930614 &   94.121130 \\
tertiary education                    &  47.536173 &  62.580966 &  20.829847 &  27.525643 &  -2.795276 &  38.959689 &  25.199548 &  37.588198 &  -17.014110 &   67.006217 \\
\bottomrule
\end{tabular}
}
\end{table}





% Table 2 - individual countries a)
\begin{table}[!htbp]
\centering 
\caption{Relative labour supply changes by country - Central Europe}
\label{labour_supply_changes_ce}
\resizebox{\textwidth}{!}{
\begin{tabular}{lrrrrrrrrrr}
\toprule
{} &  2010/2007 &  2019/2011 &  2010/2007 &   2019/2011 &  2010/2007 &  2019/2011 &  2010/2007 &   2019/2011 &  2010/2007 &   2019/2011 \\
Country &         CZ &         CZ &         HU &          HU &         PL &         PL &         SK &          SK &         SI &          SI \\
Groups                                &            &            &            &             &            &            &            &             &            &             \\
\midrule
%(upper) secondary education           &  -4.072809 &  -6.978868 &  -4.258382 &  -14.639163 &  -5.887259 & -12.735159 &  -7.592798 &    0.478960 &  -1.317696 &  -16.957616 \\
$<$5                                    &   8.226937 &  -8.964951 & -10.410914 &   -7.212603 &   1.465125 & -42.976057 &  15.683775 &  -42.556512 &   1.303569 &  -38.138896 \\
15-25                                 &   1.231744 & -12.722249 &  -4.766265 &   -7.567406 & -11.520447 &  18.913838 &  -7.891195 &    4.435811 & -11.531289 &    1.571734 \\
25-35                                 &   2.367041 &   9.069147 &   7.254298 &    1.364161 &  -5.938776 & -24.437601 &  -2.540541 &  -12.859471 &   3.443666 &  -13.001034 \\
35-45                                 &  -2.820841 &  10.661964 &  40.908923 &   28.621185 &  30.080666 &  27.337031 &  14.292898 &   21.478962 &  12.836356 &   64.505852 \\
5-15                                  &  -6.422063 &  -0.039276 & -14.599780 &  -15.640995 &   5.180733 &   8.675418 &  -1.798339 &   16.635153 &   4.815126 &    3.894283 \\
%$<$5                                    &   8.226937 &  -8.964951 & -10.410914 &   -7.212603 &   1.465125 & -42.976057 &  15.683775 &  -42.556512 &   1.303569 &  -38.138896 \\
$>$45                                   &  49.114285 &  53.040895 &  47.969350 &  151.887733 &   4.788782 &  17.825691 &  33.106098 &   41.217003 &  50.342057 &  145.376930 \\
Female                                &   1.896840 &   5.510090 &   2.314666 &   -0.537905 &   6.180807 &   6.427731 &   1.206245 &   -1.029751 &   2.865350 &    1.253471 \\
Male                                  &  -0.930675 &  -2.870336 &  -1.592832 &    0.402593 &  -3.707420 &  -4.259695 &  -0.772012 &    0.659910 &  -2.048722 &   -0.937857 \\
primary education                     &        inf &        inf &        NaN &         NaN & -20.218391 & -53.699785 & -81.802288 & -105.235923 & -62.171101 &        -inf \\
lower secondary education             &  -7.342903 & -14.269409 &  -0.214999 &   17.948349 &  59.969552 &  66.649061 &   5.393534 &    6.398527 & -16.042602 &  -54.422373 \\
post-secondary non tertiary education &  -1.215507 &       -inf &  -6.539873 &    6.072272 &  -0.771252 & -61.832275 &        inf &   -5.867509 &        NaN &         NaN \\
(upper) secondary education           &  -4.072809 &  -6.978868 &  -4.258382 &  -14.639163 &  -5.887259 & -12.735159 &  -7.592798 &    0.478960 &  -1.317696 &  -16.957616 \\
%pre-primary education                 &        NaN &        NaN &        NaN &         NaN & -81.038886 &       -inf &        NaN &         NaN &        NaN &         NaN \\
%primary education                     &        inf &        inf &        NaN &         NaN & -20.218391 & -53.699785 & -81.802288 & -105.235923 & -62.171101 &        -inf \\
tertiary education                    &  11.830884 &  20.047701 &   7.377572 &   12.992954 &  13.578050 &  25.139745 &  11.664222 &   -1.072560 &   8.347106 &   26.038308 \\
\bottomrule
\end{tabular}

}
\end{table}

% Table 2 - individual countries b)
\begin{table}[!htbp]
\centering 
\caption{Relative labour supply changes by country - Balkans and Baltics}
\label{labour_supply_changes_bb}
\resizebox{\textwidth}{!}{
\begin{tabular}{lrrrrrrrrrr}
\toprule
{} &   2010/2007 &  2019/2011 &   2010/2007 &  2019/2011 &  2010/2007 &  2019/2011 &   2010/2007 &  2019/2011 &  2010/2007 &  2019/2011 \\
Country &          BG &         BG &          EE &         EE &         LT &         LT &          LV &         LV &         RO &         RO \\
Groups                                &             &            &             &            &            &            &             &            &            &            \\
\midrule
%(upper) secondary education           &    1.270990 &  -7.472449 &   -4.219765 & -24.563782 & -17.017731 & -11.006235 &  -11.814197 & -21.036536 &  -2.075079 &  -0.143069 \\
$<$5                                    &   -0.951242 &  10.550018 &   -1.107954 &  -3.477220 & -11.578134 & -18.084308 &    5.888761 & -27.601205 &  -3.182683 & -48.302524 \\
15-25                                 &   -0.552386 &  -0.697178 &   -3.466915 & -14.004675 & -11.787495 & -28.160306 &   -8.130503 &  -8.438484 &   9.563295 &  -5.306868 \\
25-35                                 &   -3.163274 &  -3.337190 &    0.920876 &  -9.686528 &   5.593757 & -19.990798 &    1.250872 & -13.213702 & -13.230831 &  12.441937 \\
35-45                                 &   21.015362 &   9.278204 &    5.498243 &  21.481766 &  24.406446 &  57.209064 &   -5.444474 &  26.575962 &  31.079075 &  15.631660 \\
5-15                                  &   -7.183673 &  -8.973120 &    2.123712 &   9.307710 &   8.432220 &  24.372235 &    7.235132 &  16.005090 &  -4.018921 &  -2.376742 \\
%$<$5                                    &   -0.951242 &  10.550018 &   -1.107954 &  -3.477220 & -11.578134 & -18.084308 &    5.888761 & -27.601205 &  -3.182683 & -48.302524 \\
$>$45                                   &  117.220952 &  85.395334 &   -9.337257 &  40.949785 &  -8.184769 &  58.456756 &  -33.811906 &  46.149722 & -71.209834 &  49.650510 \\
Female                                &    2.279623 &  -2.792326 &    7.251265 &  -5.924975 &   7.980956 &  -3.975438 &    9.192610 &  -3.627072 &   1.804030 &  -0.504534 \\
Male                                  &   -1.511782 &   1.978517 &   -4.949785 &   4.096353 &  -6.727155 &   3.277389 &   -7.089635 &   2.914129 &  -1.088929 &   0.314719 \\
primary education                     &  -28.277864 &  -6.329743 &   43.808599 & -38.810912 &   4.774837 & -66.305449 & -134.875227 &  30.864208 & -58.023669 & -20.616426 \\
lower secondary education             &    7.138331 & -16.992489 &  -15.246012 &  16.672097 &   9.909781 & -47.284491 &  -25.265803 & -61.291948 & -14.709289 & -30.352261 \\
(upper) secondary education           &    1.270990 &  -7.472449 &   -4.219765 & -24.563782 & -17.017731 & -11.006235 &  -11.814197 & -21.036536 &  -2.075079 &  -0.143069 \\
post-secondary non tertiary education &  -72.709787 & -18.049389 & -104.662174 &  71.492996 & -20.440135 &  -1.336479 &  -52.989267 &  41.567394 &  -0.935263 &   3.934652 \\
%pre-primary education                 &   72.507726 &       -inf &         NaN &        NaN &        NaN &        NaN & -112.070892 &       -inf &        NaN &        NaN \\
%primary education                     &  -28.277864 &  -6.329743 &   43.808599 & -38.810912 &   4.774837 & -66.305449 & -134.875227 &  30.864208 & -58.023669 & -20.616426 \\
tertiary education                    &   -1.393563 &  14.118566 &   15.750078 &   9.996584 &  17.107347 &   7.413791 &   28.973587 &  13.829629 &  19.051972 &   9.350157 \\
\bottomrule
\end{tabular}
}
\end{table}



\begin{table}[!htbp]
\centering 
\caption{Random Effect model comparison - CEE and Visegrad countries}
\label{RE_models_comparison}

\resizebox{\textwidth}{!}{\begin{tabular}{lccccc}
\toprule
                               &   \textbf{CEE}  & \textbf{CEE - Unions} & \textbf{CEE $t^{2}$} & \textbf{Visegrad} & \textbf{Visegrad $t^{2}$}  \\
\midrule
\textbf{Dep. Variable}         &     logW\textbackslash HL    &        \textbackslash HL       &     logW\textbackslash HL     &      logW\textbackslash HL     &        logW\textbackslash HL        \\
\textbf{No. Observations}      &       118       &          118          &       118        &         60        &           60           \\
\textbf{Cov. Est.}             &    Unadjusted   &       Unadjusted      &    Unadjusted    &     Unadjusted    &       Unadjusted       \\
\textbf{R-squared}             &      0.7631     &         0.7703        &      0.7731      &       0.8058      &         0.8430         \\
\textbf{R-Squared (Within)}    &      0.4455     &         0.4624        &      0.4690      &       0.5993      &         0.6759         \\
\textbf{R-Squared (Between)}   &      1.0000     &         1.0000        &      1.0000      &       1.0000      &         1.0000         \\
\textbf{R-Squared (Overall)}   &      0.7631     &         0.7703        &      0.7731      &       0.8058      &         0.8430         \\
\textbf{F-statistic}           &      38.657     &         35.884        &      36.465      &       44.818      &         47.413         \\
\textbf{P-value (F-stat)}      &      0.0000     &         0.0000        &      0.0000      &       0.0000      &         0.0000         \\
\textbf{=====================} & =============== &    ===============    & ===============  &  ===============  &    ===============     \\
\textbf{log(H\textbackslash L)}             &     -0.1002     &        -0.0739        &     -0.1514      &      -0.1521      &        -0.2496         \\
\textbf{ }                     &    (-1.6702)    &       (-1.2114)       &    (-2.3842)     &     (-1.4363)     &       (-2.4955)        \\
\textbf{Time}                 &     -0.0078     &        -0.0054        &      3.2631      &      -0.0109      &         6.5808         \\
\textbf{ }                     &    (-2.5403)    &       (-1.6454)       &     (2.1696)     &     (-2.4631)     &        (3.5341)        \\
\textbf{EE}            &     -0.0310     &        -0.0052        &      0.0128      &                   &                        \\
\textbf{ }                     &    (-0.5516)    &       (-0.0901)       &     (0.2183)     &                   &                        \\
\textbf{HU}            &      0.1481     &         0.1532        &      0.1704      &       0.1707      &         0.2133         \\
\textbf{ }                     &     (4.2572)    &        (4.4386)       &     (4.7729)     &      (3.3059)     &        (4.4070)        \\
\textbf{LT}            &      0.1197     &         0.1289        &      0.1690      &                   &                        \\
\textbf{ }                     &     (1.9310)    &        (2.0948)       &     (2.5981)     &                   &                        \\
\textbf{LV}            &      0.1550     &         0.1393        &      0.1892      &                   &                        \\
\textbf{ }                     &     (3.1199)    &        (2.7909)       &     (3.6864)     &                   &                        \\
\textbf{PL}            &     -0.0265     &        -0.0436        &     -0.0070      &      -0.0067      &         0.0305         \\
\textbf{ }                     &    (-0.8184)    &       (-1.3087)       &    (-0.2105)     &     (-0.1436)     &        (0.7012)        \\
\textbf{SI}            &      0.1506     &         0.0648        &      0.1796      &                   &                        \\
\textbf{ }                     &     (3.6747)    &        (1.0454)       &     (4.2319)     &                   &                        \\
\textbf{SK}            &     -0.1583     &        -0.1592        &     -0.1559      &      -0.1559      &        -0.1513         \\
\textbf{ }                     &    (-6.8684)    &       (-6.9814)       &    (-6.8721)     &     (-6.6327)     &       (-7.0800)        \\
\textbf{constant}                 &      0.4610     &         0.3808        &      3262.3      &       0.4392      &         6575.2         \\
\textbf{ }                     &     (5.8769)    &        (4.2735)       &     (2.1751)     &      (3.3976)     &        (3.5402)        \\
\textbf{UD}                    &                 &         0.0058        &                  &                   &                        \\
\textbf{ }                     &                 &        (1.8311)       &                  &                   &                        \\
\textbf{$Time^2$}             &                 &                       &     -0.0008      &                   &        -0.0016         \\
\bottomrule
\end{tabular}}
\end{table}




\begin{table}[!htbp]
\centering 
\caption{Fixed effect model comparison - CEE and Visegrad countries (T-stats reported in parentheses)}
\label{FE_models_comparison}

\begin{center}
\resizebox{\textwidth}{!}{\begin{tabular}{lccccc}
\toprule
                               & \textbf{CEE} & \textbf{CEE - Unions} & \textbf{CEE $t^{2}$} & \textbf{Visegrad} & \textbf{Visegrad $t^{2}$}  \\
\midrule
\textbf{Dep. Variable}         &   logW\textbackslash HL   &        logW\textbackslash HL      &     logW\textbackslash HL    &      logW\textbackslash HL     &        logW\textbackslash HL        \\
\textbf{No. Observations}      &     118      &          118          &       118        &         60        &           60           \\
\textbf{Cov. Est.}             &  Unadjusted  &       Unadjusted      &    Unadjusted    &     Unadjusted    &       Unadjusted       \\
\textbf{R-squared}             &    0.2007    &         0.3376        &      0.2044      &       0.3862      &         0.4030         \\
\textbf{R-Squared (Within)}    &    0.4070    &         0.4173        &      0.4128      &       0.5134      &         0.5526         \\
\textbf{R-Squared (Between)}   &    0.0472    &         0.2787        &      0.0498      &       0.2666      &         0.2624         \\
\textbf{R-Squared (Overall)}   &    0.2007    &         0.3376        &      0.2044      &       0.3862      &         0.4030         \\
\textbf{F-statistic}           &    14.434    &         19.369        &      9.7651      &       17.932      &         12.603         \\
\textbf{P-value (F-stat)}      &    0.0000    &         0.0000        &      0.0000      &       0.0000      &         0.0000         \\
\textbf{=====================} & ============ &      ============     &   ============   &    ============   &      ============      \\
\textbf{log(H\textbackslash L)}             &    0.0630    &         0.1080        &      0.0603      &       0.2082      &         0.1996         \\
\textbf{ }                     &   (2.1143)   &        (3.7546)       &     (2.0059)     &      (3.0956)     &        (2.9685)        \\
\textbf{trend}                 &   -0.0149    &        -0.0121        &      1.8574      &      -0.0246      &         4.2630         \\
\textbf{ }                     &  (-5.3300)   &       (-4.6309)       &     (0.7300)     &     (-5.9448)     &        (1.2497)        \\
\textbf{const}                 &    0.6265    &         0.5107        &      1868.9      &       0.7966      &         4279.5         \\
\textbf{ }                     &   (19.494)   &        (13.496)       &     (0.7361)     &      (10.969)     &        (1.2571)        \\
\textbf{UD}                    &              &         0.0079        &                  &                   &                        \\
\textbf{ }                     &              &        (4.8551)       &                  &                   &                        \\
\textbf{$Time^2$}             &              &                       &     -0.0005      &                   &        -0.0011         \\
\bottomrule

\end{tabular}}
\end{center}
\end{table}

\end{document}
